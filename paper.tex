\documentclass[a4paper,fleqn,usenatbib]{mn2e}

\usepackage[hyperfootnotes=false]{hyperref}
\hypersetup{
  colorlinks,
  citecolor=blue,
  linkcolor=red,
  urlcolor=cyan}

% MNRAS is set in Times font. If you don't have this installed (most LaTeX
% installations will be fine) or prefer the old Computer Modern fonts, comment
% out the following line
\usepackage{pgfplotstable}
\usepackage{newtxtext,newtxmath}
\usepackage{caption}
\usepackage{subcaption}
% Depending on your LaTeX fonts installation, you might get better results with one of these:
%\usepackage{mathptmx}
%\usepackage{txfonts}

% Use vector fonts, so it zooms properly in on-screen viewing software
% Don't change these lines unless you know what you are doing
\usepackage[T1]{fontenc}
\usepackage{ae,aecompl}

% fancy section symbol
\usepackage{cleveref}
\crefname{section}{$\S$}{$\S\S$}
\Crefname{section}{$\S$}{$\S\S$}

%%%%% AUTHORS - PLACE YOUR OWN PACKAGES HERE %%%%%

% Only include extra packages if you really need them. Common packages are:
\usepackage{graphicx}   % Including figure files
\graphicspath{{Users/}{owenturner/}{Documents/}{PhD/}{KMOS/}{paper_1/}{Figures/}}
\usepackage{amsmath}    % Advanced maths commands
\usepackage{amssymb}    % Extra maths symbols

%%%%%%%%%%%%%%%%%%%%%%%%%%%%%%%%%%%%%%%%%%%%%%%%%%

%%%%% AUTHORS - PLACE YOUR OWN COMMANDS HERE %%%%%

\newcommand{\Sers}{S\'{e}rsic }
\newcommand{\Lagr}{\mathcal{L}}

% Please keep new commands to a minimum, and use \newcommand not \def to avoid
% overwriting existing commands. Example:
%\newcommand{\pcm}{\,cm$^{-2}$} % per cm-squared

%%%%%%%%%%%%%%%%%%%%%%%%%%%%%%%%%%%%%%%%%%%%%%%%%%

%%%%%%%%%%%%%%%%%%% TITLE PAGE %%%%%%%%%%%%%%%%%%%

% Title of the paper, and the short title which is used in the headers.
% Keep the title short and informative.
\title[Spatially resolved dynamics at $z \sim 3.5$]{The KMOS Deep Survey: Resolved gas dynamics at $z \sim 3.5$}

% The list of authors, and the short list which is used in the headers.
% If you need two or more lines of authors, add an extra line using \newauthor
\author[O.J. Turner et al.]{
Owen J. Turner,$^{1}$\thanks{E-mail: turner@roe.ac.uk (OJT)}
M. Cirasuolo,$^{1}$
J. Dunlop$^{1}$
R. J. McLure$^{1}$
\\
% List of institutions
$^{1}$SUPA\thanks{Scottish Universities Physics Alliance}, Institute for Astronomy, University of Edinburgh, Royal Obervatory, Edinburgh EH9 3HJ\\
$^{2}$Department, Institution, Street Address, City Postal Code, Country\\
$^{3}$Another Department, Different Institution, Street Address, City Postal Code, Country
}

% These dates will be filled out by the publisher
\date{Accepted XXX. Received YYY; in original form ZZZ}

% Enter the current year, for the copyright statements etc.
\pubyear{2016}

% Don't change these lines
\begin{document}
\label{firstpage}
\pagerange{\pageref{firstpage}--\pageref{lastpage}}
\maketitle

% Abstract of the paper
\begin{abstract}
We present first results from the KMOS Deep Survey (KDS), which is an ESO guaranteed time survey of 80 star-forming galaxies in the redshift range $z = 3 - 3.8$.
From this sample we detect spatially resolved [OIII5007] emission in the K-band from 57 galaxies, allowing us to explore the fraction which show ordered rotational motions, presumably in a star-forming disk.
41 galaxies show a clear, smooth velocity gradient along the kinematic axis, meriting them for further investigation via dynamical modelling.
The ratio of maximum rotational velocity to intrinsic velocity dispersion, $v / \sigma _{int}$, is commonly used to indicate the extent to which the gas in disk galaxies is dominated by ordered rotation or random motions.
We find that 16 galaxies have $v / \sigma _{int} > 1$, in broad agreement with results found at similar and low redshift.
The mean values of $v / \sigma _{int}$ are substantially lower in this redshift range than are observed locally; since the measured rotational velocity values have remained roughly constant over cosmic time this suggests that turbulent gas motions within the disk become steadily more important with increasing redshift.
We investigate the causes for this, linking the increase in $\sigma _{int}$ to increasing gas fraction and decreasing galaxy size, following the virial theorem and a $(1 + z)^{-1}$ scaling.
Finally we look into correlations between the maximum rotational velocity of galaxies and their global properties, finding that galaxies with higher stellar mass tend to rotate faster.




\end{abstract}

% Select between one and six entries from the list of approved keywords.
% Don't make up new ones.
\begin{keywords}
Integral Field Unit -- KMOS -- Dynamics -- Turbulence
\end{keywords}

%%%%%%%%%%%%%%%%%%%%%%%%%%%%%%%%%%%%%%%%%%%%%%%%%%

%%%%%%%%%%%%%%%%% BODY OF PAPER %%%%%%%%%%%%%%%%%%

\section{INTRODUCTION}

% TABLES TO INCLUDE
% 1) MORPHOLOGICAL PROPERTIES
% 2) SPECTRAL PROPERTIES
% 3) PROPERTIES INFERRED FROM PHOTOMETRY
% 4) TABLE OF FINALISED KINEMATIC MEASUREMENTS
% SO THAT INCLUDES THE VELOCITY DISPERSION AND THE VELOCITY EXTRACTED
% IN DIFFERENT WAYS FOR THE DIFFERENT POSITION ANGLES AND MODELLING APPROACHES


NO UNIVERSALLY DEFINED MEASURE OF THE CIRCULAR VELOCITY, OR THE POINT AT WHICH TO EXTRACT THAT VALUE.

% Check references here - exactly where does this scenario come from? 
In the hierarchical collapse scenario \citep{Fall1980}, matter density perturbations, which originate from some set of cosmological initial conditions, grow over time before condensing out of the general expansion to form dark matter haloes (Lifshitz 1960) \cite{Silk1968,Peebles1970}.
Mixed amongst this dark matter is baryonic gas, which is initially shock heated to the virial temperature of the halo during the gravitational collapse \citep{Birnboim2003,Keres2005,Dekel2006}.
As the baryons collectively cool, they accumulate at the centre of the potential well of the host halo as dense and cold protogalactic gas.
One crucial ingredient shaping the subsequent evolution of this system is the transfer of angular momentum from the collapsing gas, as a protogalaxy which is rich in angular momentum is likely to form a turbulent, rotating disk, whereas if the angular momentum is low the formation of a spherical system dominated by random motions will ensue.

It is an observationally determined fact that the galaxy population is bimodal in many physical properties at each epoch after galaxies begin to form, with a preference for the most massive galaxies to lie in the red sequence, characterised by red colours, low Star Formation Rates (SFRs) and spherical morphologies, and less massive galaxies in the blue sequence, characterised by blue colours, high star formation rates and disky morphologies. 
\citep{Abramson2016,Abramson2016a}
\citep{Pannella2009a}

\citep{Baldry2006}
simulations explaining the biodality in terms of AGN feedback and a cut at a characteristic mass scale.
\citep{Birnboim2003,Dekel2006,Keres2005,Bower2006,Croton2006}
Explaining the observed number densities of early and late type galaxies at different epochs in terms of their formation mechanisms and evolutionary histories is one of the key goals of modern astronomy.
Two scenarii have emerged. The first highlights the importance of secular evolution, whereby cold gas accreted along cosmic web filaments `tops up' the existing baryon supply which fuels star formation, stellar mass within a galaxy is continually built up with in-situ star formation and the rate of star formation itself is regulated by both stellar and AGN feedback.
In this picture there is a certain mass scale above which a galaxy becomes quenched - EXPAND UPON THAT
The second suggests that environmental effects dominate the evolutionary demographic, with mass buildup predominantly occuring through minor mergers and major mergers triggering the cessation of star formation and building the red sequence at each epoch.  
Throughout the last few years both the results from simulations REF and observations have pushed the community towards the former picture, which downplays the significance of mergers in driving galaxy evolution. 









%,Understanding the evolving dynamical state of disky star-forming galaxies (SFGs) throughout cosmic time is central to understanding the topics of galaxy formation and evolution.
%There are two different evolutionary scenarii.
%The first suggests that the buildup of stellar mass, $M_{*}$, over time within galaxies is hierarchical, with minor mergers steadily contributing to $M_{*}$ and major mergers bringing about the cessation of star formation by shock heating of the cold molecular gas. 
%The second is the equilibrium model of galaxy evolution e.g. \cite{Genel2008,Bouche2010,Dave2011,Dave2011a,Dave2011b,Krumholz2012,Lilly2013}, where the gas content of SFGs regulates their Star Formation Rates (SFRs) and varies smoothly inside individual galaxies throughout their lifetimes.
%In this picture the galactic gas budget is mediated by inflows of cold gas along cosmic web filaments \citep{Keres2005,Dekel2006} processing of the cold gas with steady, secular star formation and by outflows of gas from Active Galactic Nuclei (AGN) and stellar feedback.
%SFGs locked in this cycle of gas consumption form a galactic `Main Sequence', which is a fairly tight relationship between the SFR and stellar mass $M_{*}$  
%Over the last decade there have been considerable advancements 

References for other things
MAIN SEQUENCE
\citep{Daddi2007} \citep{Elbaz2011a} \citep{Elbaz2007} \citep{Noeske2007a}





These are all the references for kinematics, not all the references for other things and other aspects of galaxy evolution. 
\citep{Wisnioski2015} \citep{ForsterSchreiber2009} \citep{ForsterSchreiber2006} \citep{Stott2016} \citep{Gnerucci2011} \citep{Epinat2012} ,\citep{Epinat2009} \citep{Epinat2010} \citep{Lilly2013} \citep{Saintonge2013} \citep{Wisnioski2011} \citep{Wright2007} \citep{Wright2009} \citep{Genzel2008} \citep{Genzel2006} \citep{Shapiro2008} \citep{Jones2010a} \citep{Newman2013} \citep{Genel2008} \citep{Nesvadba2008} \citep{Queyrel2012} \citep{Law2007} \citep{Law2009}
\citep{Simons2016} \citep{Pelliccia2016} \citep{Genzel2011} \citep{Epinat2008} \citep{Epinat2008a} \citep{Puech2008} \citep{Puech2007} \cite{Holmberg1958} \citep{Kassin2012} \citep{Flores2006} \citep{Neichel2008} \citep{Miller2011} \citep{Nesvadba2006} \citep{Nesvadba2007} \citep{Livermore2015} \citep{Swinbank2006} \citep{Swinbank2007} \citep{Swinbank2009} \citep{Cortese2014} \citep{VanderWel2012} \citep{Genzel2014a} \citep{Genzel2014} - for the PSF used when Galfitting F160W. \citep{Rodriguez-Gomez2016} \citep{Qu2016}



MOTIVATING THE FOLLOWING SECTION

The KMOS Deep Survey (KDS) is a guaranteed time programme focussing on the spatially resolved properties of star forming galaxies during a time where the universe is building to peak activity.
There are still many open questions which we can begin to answer by studying the emission, or lack thereof, from regions of ionised gas within individual galaxies at these redshifts.
For example: \textbf{(1)} How do the physical conditions of the ISM such as electron density and ionisation parameter change as a function of spatial position within the galaxy, and what does this tell us about the application of strong line metallicity calibrators at high redshift?
\textbf{(2)} What are the radial gradients in metal enrichment and would evidence for inverse gradients suggest infall of pristine gas to the central regions of these galaxies?
\textbf{(3)} What are the connections between the chemical properties and the kinematics of the gas, particularly in terms of inflows and outflows of material?
\textbf{(4)} What role does formation environment play in shaping the dynamical and chemical evolution of galaxies? 
KMOS benefits from the ability to study many targets simultaneously, with 24 deployable IFUs patrolling a $7^{\prime}$ patch of sky.
Attempting to spatially resolve the flux from emission lines at $z \sim 3.5$ necessarily requires on source integration times $>$ 6 hours, and so the multiplexing capability provides KMOS with an enormous advantage in terms of observing time over other instruments for this type of study.
This has the knock-on effect of enabling a more general galaxy selection criteria, so that the KDS targets will be representative of the main sequence star forming population. 

Many of these questions have already been addressed at lower redshift, as described above \citep{ForsterSchreiber2009,Wisnioski2015,Stott2016}, and also in a similar redshift range \cite{Gnerucci2011}.
The KDS will complement the lower redshift results by providing an additional data point on any plots showing evolutionary sequences of derived physical properties, therefore providing a more complete description of galaxy evolution starting from before cosmic high noon and running up to the present day.
The high redshift results of \cite{Gnerucci2011} will be expanded upon by the KDS, which extends to lower stellar masses and will increase the sample size for which physical properties can be resolved by roughly a factor of three.

In this paper we focus on deriving and interpreting the spatially resolved kinematics of KDS galaxies using the [OIII5007] emission line, discussing what we can learn about the nature of galaxy formation at $z \sim 3.5$, and focussing on forming evolutionary links with lower redshift work.
We combine our KMOS data with the high resolution and sensitivity of the Hubble Space Telescope (HST) WFC3 H-band imaging in a spectro-photometric analysis, so that the kinematics can also be understood in the context of galaxy morphology.
In a future paper, (Turner \textit{et al.} in prep.), we will investigate the radial gradients in galaxy metallicity and hence the spatially resolved physical conditions within the ISM.

The structure of the current paper is as follows. In \cref{sec:Survey_and_data} we present the survey description, sample selection, observation strategy and data reduction.
In \cref{sec:extracting_properties} we explain how the two-dimensional maps of physical properties were extracted from our stacked datacubes and then describe the treatment of these maps and the modelling which leads to the final table of physical properties for the KDS galaxies. 
\cref{sec:results} presents our physical findings which are then discussed in \cref{sec:discussion}.
We then conclude in \cref{sec:conclusions}.
Throughout this work we assume a standard $\Lambda$CDM cosmology with (h, $\Omega_{m}$, $\Omega_{\Lambda}$) = (0.7, 0.3, 0.7). 

\section{SURVEY \& DATA REDUCTION}\label{sec:Survey_and_data}

% note going to assume here that the KMOS targets were selected from 
% from VVDS, COSMOS

%%%%%%%%%%%%%%%%%%%%%%%%%%%%%%%%%%%%%%
% Notes for the sample selection plots
%%%%%%%%%%%%%%%%%%%%%%%%%%%%%%%%%%%%%%
% 1. mstar vs sfr - must have full sample in the backgroud 
% and also lines showing where the main sequence of star formation
% should lie on this plot
% 2. colour vs mass plots - showing again the general population in 
% the background - will need to convert fluxes to magnitudes?
% Maybe a redshift vs. magnitude plot or something similar, showing
% how faint these targets are.

\subsection{The KMOS Deep Survey: sample selection and observations}
Our target selection for the KDS sample is designed to pick out star-forming galaxies at $z = 3-4$, supported by deep multi-wavelength ancillary data and probing both cluster and field environments.
Within this redshift range the [OIII4959,5007] doublet and the $H_{beta}$ emission lines are visible in the K-band and the [OII3727,3729] doublet is visible in the H-band, both of which are observable with KMOS. 
We also want to draw galaxies covering a wide range in the SFR-$M_{*}$ relation, so that we are ultimately tracking the physical properties of a representative sample of the star-forming population at these redshifts.
For these reasons, and to ensure a high detection rate of the ionised gas emission lines in the KDS galaxies, we select galaxies in well studied fields that have a previous spectroscopic detection.
To avoid biasing the sample towards blue and highly star-forming galaxies we make no secondary cuts on the basis of mass and flux. %FIGURE REFERENCE%
Predominantly, the galaxies have been selected for spectroscopic follow up using the Lyman-Break technique \citep{Steidel1996}, which is especially effective at picking out normal starforming galaxies at $z = 3-4$.
However a subset of the selected cluster galaxies in SSA22 were blindly detected in Ly$\alpha$ emission, during a narrowband imaging study of a known overdensity of Lyman Break Galaxies (LBGs) at $z \sim 3.09$ \citep{Steidel2000}.
These `Lyman-Alpha Blobs' (LABs), which are extended and diffuse emission regions associated with but not centred on LBGs \citep{Steidel2000}, and especially LAB01, have been the subject of much previous study, e.g. \citep{Bower2004a}, \citep{Matsuda2004,Matsuda2007}, \citep{Geach2006,Geach2014}, \citep{Weijmans2010}, \citep{Hayes2011a} and LAB01 was the study of a recent campaign \citep{Geach2016} using data from MUSE, \citep{Bacon2010}, ALMA, \citep{Wootten2009}, and MOSFIRE, \citep{McLean2012}.
These studies generally conclude that LABs are observed at the peaks of the cosmic density field and are therefore the progenitors of the most massive galaxy clusters, but the precise origin of the nebulous Ly$\alpha$ line emission is still unknown.

The non-cluster environment galaxies are selected from the GOODS-S field, accessible from the VLT and with excellent multi-wavelength coverage, including deep HST WFC3 F160W imaging with $ 0.13^{\prime\prime}$ pixel scale and $ 0.17^{\prime\prime}$ PSF, which is well suited for constraining galaxy morphology \citep{Grogin2011,Koekemoer2011}.
We select targets from the various spectroscopic campaigns which have targetted GOODS-S, including measurements from VIMOS \citep{Balestra2010,Cassata2014}, FORS2 \citep{Vanzella2005,Vanzella2006,Vanzella2008} and both LRIS and FORS2 as outlined in \cite{Wuyts2009}.
These targets must be within the redshift range $3 < z < 3.8$, have high spectral quality (as quantified by the VIMOS redshift flag equal `3' or `4', and the FORS2 quality flag equal `A') and we carefully exclude those targets for which the [OIII5007] or H$_{\beta}$ emission lines, observable in the K-band at these redshifts, would be shifted into a spectral region plagued by strong OH emission.
The galaxies which remain after imposing these criteria are distributed across the GOODS-S field, and we select two regions of spatial overdensity in the south-east (goodsp1) and north-west (goodsp2) to fill the KMOS IFUs (noting that the IFUs can patrol a $7.2^{\prime}$ diameter patch of sky during a single pointing).

% Add the programme ID for HST WFC3 imaging
The cluster galaxies are all from the SSA22 field, \citep{Steidel1998,Steidel2000,Steidel2003}, \citep{Shapley2003}, which, as mentioned above, is an extreme overdensity of LBG candidates at $z \sim 3.09$.
Thousands of spectroscopic redshifts have been confirmed for these LBGs with follow up observations using LRIS, \citep{Shapley2003}, \citep{Nestor2013}, and when combined with deep B,V,R band imaging with the Subaru Suprime-Cam \citep{Matsuda2004}, deep narrow band imaging at 3640$\AA$ \citep{Matsuda2004} and at 4977$\AA$ \citep{Nestor2011} and archival HST ACS and WFC3 imaging covering large swathes of the field, the ancillary data is in excellent support of integral field spectroscopy.
Spectroscopic data quality assessment and OH feature avoidance criteria are applied in the same way as discussed above, and we fill the KMOS IFUs with galaxies in two pointings spanning the SSA22$\_$a (ssap1) and SSA22$\_$b (ssap2) fields.


% Mentioning the COSMOS field that we did not observe
Additional pointings in the two fields listed above as well as in both the COSMOS and UDS fields were originally planned, but 50$\%$ of the GTO observing time was lost to bad weather.

% Physical properties - Waiting on information from ROSS for writing this
The wealth of ancillary data in both fields allows for a consistent treatment of the SED modelling, providing physical properties which are directly comparable between the cluster and field environments.
We assume solar metallicity stars, the \cite{Calzetti2000} reddening law, and either constant or exponentially declining SFRs.
The stellar masses and SFRs are extracted....

% comparisons with the Holden paper - MUST EXPLAIN DIFFERENCES IN PHYSICAL PR
Ten of the targets in goodsp1 have also been observed with the MOSFIRE spectrograph \citep{McLean2012}, and were the subject of a recent paper by \cite{Holden2016} concerning the ubiquitous nature of elevated [OIII]/H$\beta$ ratios in LBGs, and the consequences this observation has for the physical conditions of the ISM (in terms of electron density, ionisation parameter and hardness of radiation field).
The results from the KDS will complement this study, in the sense that spatially resolved dynamical and chemical information provide key insights into the physical processes driving high redshift LBGs into regions of parameter space which only the most extreme galaxies broach in the local universe.

\subsubsection{Survey Stategy}
The two key science goals of the KDS are to investigate the resolved kinematic properties of high redshift galaxies, particularly the fraction of rotating disks and the degree of disk turbulence, and also to investigate the spatial distribution of metals within these galaxies.
To achieve this we require very deep exposure times in excess of 8 hours on source for all galaxies, to reach the S/N required to detect line emission in the outskirts of the galaxies, where the rotation curves begin to flatten.
We prioritise the kinematics goal, which requires spatially resolving the [OIII5007] emission line, and so plan to first reach the required K-band depth before observing in the H-band to complement the measurements with [OII3727,3729].
After large amounts of time was lost to poor weather conditions we decided to observe the SSA22 galaxies with the HK filter, which has the disadvantage of effectively halving the spectral resolution, but allows for coverage of the H-band and K-band regions simultaneously.
This paper is concerned with the KDS kinematics and so for the remainder we focus only on the K-band observations, which took place over the first two years of the programme. 

\subsubsection{Observations}\label{subsub:Obs}
KMOS is a second generation IFS mounted at Nasmyth focal plane on UT1 at the VLT.
The instrument has 24 moveable pickoff arms, each with an integrated IFU, which patrol a region 7.2$^{\prime}$ in diameter on the sky, providing considerable flexibility when selecting sources for a single pointing.
The light from each set of 8 IFUs is dispersed by a single spectrograph and recorded on a 2k$\times$2k Hawaii-2RGHgCdTe near-IR detector, so that the instrument is comprised of three effectively independent modules.
Each IFU has 14$\times$14 spatial pixels which are 0.2$^{\prime\prime}$ in size, and the central wavelength of the K-band filter has spectral resolution of $R \sim 4200$.

We prepare each pointing using the KARMA tool \citep{Wegner2008}, taking care to allocate at least one IFU to observations of a `control' star closeby on the sky to allow for precise monitoring of the evolution of seeing conditions and the shift of the telescope away from the prescribed dither pattern (see \cref{subsec:datareduction}).
For both all four of the pointings described above we adopt the standard object-sky-object (OSO) nod-to-sky observation pattern, with 300s exposures and alternating full pixel/half pixel dither pattern for increased spatial sampling around each of the target galaxies.
This will later allow for datacube reconstruction with 0.1$^{\prime\prime}$ size spatial-pixels (spaxels) as described in \cref{subsec:datareduction}.

The observations were carried out during August and September of observing semester P92, during August and September of P93 and November of P94.
The seeing conditions were excellent for goodsp1 and goodsp2, $\sim 0.5^{\prime\prime}$ where we observe 21 and 19 targets respectively, and ranged between $\sim 0.65-0.7^{\prime\prime}$ for ssap1 and ssap2 where 21 and 18 galaxies were observed.
Note that these numbers are less than the available 24 arms for each pointing due to the combination of three broken pickoff arms during the P92 observing semester and our requirement to observe at least one control star throughout an Observing Block (OB).
The central coordinates of these pointings, the number of target galaxies observed, N$_{obs}$, the number of galaxies with spatially resolved [OIII5007] emission, N$\_$[OIII]$_{res}$; see \cref{sec:extracting_properties}, the on source exposure time and the averaged seeing conditions are listed in Table \ref{tab:pointings}.
In Figure \ref{fig:distributions} we provide an overview of the sample space spanned by the observed galaxies, in relation to the larger population of LBGs within this redshift range. 

% Figure how to convert both 3dHST flux and ROSS flux to magnitudes, and then can make the sample and parent scatter plots in the same was as the mass sfr ones for this distribution of targets.

\begin{figure}
\centering
\includegraphics[width=0.47\textwidth]{paper_distributions.png}
\caption{Distributions of the physical properties of KDS galaxies in both GOODS-S and SSA22}
\label{fig:distributions}
\end{figure}

\begin{figure}
\centering
\includegraphics[width=0.47\textwidth]{main_sequence.png}
\caption{star forming main sequence}
\label{fig:distributions}
\end{figure}

\begin{figure}
\centering
\includegraphics[width=0.47\textwidth]{KDS-k_mag_mass.png}
\caption{Kmag and mass}
\label{fig:distributions}
\end{figure}

\begin{table*}\label{tab:pointings}
    \centering
\begin{tabular}{ c c c c c c c c }

 \hline
Field & Pointing & N$_{obs}$ & N$\_[OIII]_{res}$ & Coordinates & Band & Exp. Time (s) & Seeing (arcsec)  \\
 \hline
 GOODS-S & goodsp1 & 21 & 16 & 03:32:22 -27:52:51 & K, H & 32400 & 0.5 \\
GOODS-S & goodsp2 & 19 & 14 & 03:32:35 -27:43:15 & K, H & 31800 & 0.52 \\
SSA22 & ssap1 & 21 & 13 & 22:17:28 00:09:54 & HK & 38100 & 0.62 \\
SSA22 & ssap2 & 18 & 11 & 22:17:11 00:15:47 & HK & 27800 & 0.57 \\
 \hline
\end{tabular}
\caption{Summary of KDS statistics}
\end{table*}


\subsection{Data reduction techniques}\label{subsec:datareduction}
The data reduction process relied heavily upon the Software Package for Astronomical Reduction with KMOS, (SPARK; \cite{Davies2013}), implemented using the ESO Recipe Execution Tool (ESOREX) \citep{Freudling2013}.
In addition to the SPARK recipes, custom python scripts were run at different stages of the pipeline and will be described throughout this section.

The SPARK recipes are used to create dark frames, flatfield, illumination correct and wavelength calibrate the raw data.
The four pixel wide gaps between the IFU lenslets in the raw data are then used to correct for readout channel bias across each detector, which if left uncorrected lead to flux bandings over the spatial extent of the reduced cubes.
Standard star observations are processed to provide a flux calibration for each detector, which is necessary to account for varying sensitivity across the three independent KMOS modules. 
Following this pre-processing, each of the object exposures is reconstructed independently, using the closest sky exposure for subtraction, to give more control over the construction of the final stacks for each target galaxy.
Each 300s exposure is reconstructed into a datacube with interpolated $0.1\times0.1^{\prime\prime}$ spaxel size, facilitated by the subpixel dither pattern discussed in \cref{subsub:Obs} which boosts the effective spatial resolution of the observations.
Sky subtraction is enhanced using the SKYTWEAK option within SPARK \citep{Davies2007}, which counters the varying amplitude of OH lines between exposures by scaling `families' of OH lines independently to match the data.
Wavelength miscalibration between exposures due to spectral flexure of the instrument is also accounted for by applying spectral shifts to the OH families during the procedure, and in general we find the quality of the sky subtraction in the K-band to be excellent. % could insert figure to demonstrate
Variations in sky subtraction quality are monitored over the course of the OBs which make up the final stacks, as we find that the subtraction performance becomes poorer as the telescope pointing approaches the zenith.
This is due to the telescope having to track faster whilst at zenith positions, hence scanning more rapidly over regions of sky with spatially and temporally varying OH emission.
In addition to this we monitor the evolution of the atmospheric PSF and the position of the control stars over the OBs, to allow us to reject raw frames with unacceptable seeing and to measure the spatial shifts required for the final stack more precisely.
The telescope tends to drift from its acquired position over the course of an OB, and the difference between the dither pattern shifts and the measured position of the control stars provides the value by which each exposure must be shifted to create the stack. % insert figure here monitoring drift as function of time? Ask Michele here about level of detail
As part of this analysis, we tested whether the drift varies across the three KMOS detectors, finding typically that the difference is negligible, suggesting that it is not necessary to sacrifice more than a single IFU for tracking purposes.

We stack all 300s exposures for each galaxy which pass the sky subtraction and seeing criteria using 3 sigma-clipping, leaving us with a flux and wavelength calibrated datacube for every object in the KDS sample.

\section{MAPPING PHYSICAL PROPERTIES}\label{sec:extracting_properties}
\subsection{Spaxel Fitting}
Having produced stacked and calibrated datacubes for each of the galaxies in the sample, we aim to extract 2-dimensional maps of the flux, velocity and velocity dispersion.
These properties are extracted via modelling of the ionised gas emission line profiles at each spaxel, with a set of acceptance criteria to determine whether the fit quality is high enough to allow the inferred properties into the final 2D maps.
We concentrate solely on the [OIII5007] emission line which always has S/N higher than both [OIII4959] and H$\beta$, and therefore provides a tighter set of model constraints over a wider portion of the physical extent of each galaxy.
As a result, any reference to OIII emission points specifically to the [OIII5007] line unless otherwise stated.

We start by considering each 0.1$^{\prime\prime}$ spaxel across a datacube in turn, and locate the wavelength of the redshifted OIII centre, $\lambda_{obs_{5007}}$, using either the literature redshift value, or the redshift determined from fitting the OIII line in the integrated galaxy spectrum.
We have found that the thermal background is often undersubtracted across the spatial extent of the cube following a first pass through the pipeline, leading to excess flux towards the long wavelength end of the K-band.
To account for this a polynial function is fit, using the python package {\tt LMFIT} \citep{Newville2014} which makes use of the Levenberg-Marquardt algorithm for non-linear curve fitting, to the median stacked spectrum from spaxels in the datacube which contain no object flux, and then subtracted from each spaxel in turn.
This assumes that there is no spatial variation in the required correction, which we have found to be valid for our current sample of galaxies.
A benefit of this polymonial subtraction is that the object continuum is subtracted simultaneously, and so the subsequent fitting of emission lines is performed relative to zero flux. 
During the fit, regions of the spectrum known to contain strong sky emission are masked.
Following the subtraction of excess thermal noise and continuum emission, we search a narrow region around $\lambda_{obs_{5007}}$ for the peak flux, assumed to correspond to the ionised gas emission, and extract a 5$\AA$ region (corresponding to 20 spectral elements) centred on this peak.
This truncated spectrum is then used in the fitting procedure for each spaxel.
The width of the extraction region is large enough to encompass unphysically large velocity shifts, but not so large as to compromise the fitting by potentially encompassing regions of the spectrum plagued by poor subtraction of the sky emission lines.
A single gaussian model is fit to the extraction region, again using {\tt LMFIT} \citep{Newville2014}, returning the values of the fitted parameters flux,$F_{g}$, central wavelength, $\lambda_{g}$ and standard deviation, $\sigma_{g}$.
These parameters, along with the estimate of the noise in the datacube are used to assess whether the fitting of the OIII line is acceptable or not. 
The noise in the cube is estimated in two different ways. 
First, we take the root of the sum of the squares of the flux values in the noise cube over the same extraction region defined for the data cube.
The noise cube is produced as part of the reduction process, taking into account instrumental effects and the noise reduction due to stacking many individual exposures.
This is summarised by Equation~\ref{eq:cube_noise}, where $\sigma _{i}$ are the individual flux points in the noise cube and $N_{c}$ denotes noise estimated from the noise cube.

\begin{equation}\label{eq:cube_noise}
   N_{c} = \sum_{i}\sigma _{i}^{2}
\end{equation}

In the second method the datacube itself is used to estimate the noise.
Again, all spaxels containing object flux are masked and the sum of the squares of the flux values in the extaction region is computed for all remaining spaxels inwards of 3 spaxels from the cube border.
This last constraint is to mitigate edge effects, where the noise increases due to fewer individual exposures constituting the final stack in these regions.
The final noise value is then taken as the standard deviation of the results from each unmasked spaxel, as shown in Equation~\ref{eq:mask_noise} where $F_{i}$ are the flux values across the extraction region and $N_{m}$ denotes noise estimated from masking the datacube.
Both noise estimates are quantitatively similar and the conclusions of the following analysis are not dependent on the choice.
As default we use the noise estimate from masking throughout the rest of the section.

\begin{equation}\label{eq:mask_noise}
    N_{m} = STD\big(\sum_{i}F_{i}^{2}\big)
\end{equation}

The signal in each spaxel is taken as the sum of the flux values across the extraction region, as shown in Equation~\ref{eq:signal} where S denotes the signal.

\begin{equation}\label{eq:signal}
    S = \sum_{i}F_{i}
\end{equation}

The criteria for accepting the fit within a given spaxel are as follows.
1) The uncertainties on the parameters of the gaussian fit must not exceed 30\%
2) The ratio of the flux recovered from the gaussian fit to the OIII line to the sum of the fluxes within the extraction region must exceed 1.3 or drop below 0.7.
This helps to minimise the number of spaxels erroneously accepted due to fitting of non-gaussian profiles.
We found this was an important criterion for `cleaning up' accepted spaxels in the 2D maps which are clearly independent of the main body of the galaxy.
3) The measured S/N must exceed three.
If all three criteria are satisfied, the spaxel is accepted and the gaussian fit is repeated 1000 times with each flux value in the extraction region perturbed by the datacube flux error.
This error is computed as the median of the standard deviations in the extraction regions of the unmasked spaxels, as shown in Equation~\ref{eq:cube_error} with $F_{e}$ denoting the flux error.  

\begin{equation}\label{eq:cube_error}
   F_{e} = \big<STD(F _{i})\big>
\end{equation}

This Monte Carlo procedure gives the fit parameters and associated errors for the accepted spaxels.
The velocity is computed from the gaussian centre, $\lambda_{g}$, using Equation~\ref{eq:velocity_comp} and the velocity dispersion is computed from the gaussian width, $\sigma_{g}$, using Equation~\ref{eq:dispersion_comp}, where $v_{obs}$ and $\sigma_{obs}$ are the observed rotational velocity and velocity dispersion respectively and c is the speed of light.

\begin{equation}\label{eq:velocity_comp}
   v_{obs} = \frac{\lambda_{c} - (1 + z)\lambda_{OIII}}{(1 + z)\lambda_{OIII}} \times c
\end{equation}

\begin{equation}\label{eq:dispersion_comp}
   \sigma_{obs} = \frac{\sigma_{g}}{(1 + z)\lambda_{OIII}} \times c
\end{equation}

If a spaxel does not meet the criteria we bin with neighbouring spaxels to create a box of area $0.3\times0.3^{\prime\prime}$.
This is done by median stacking the extraction region spectra in each of the neighbouring spaxels.
The criteria are then re-examined and if satisfied the Monte-Carlo procedure is carried out. 
If the binned spaxels fail the criteria, one final iteration is carried out to bin to a box of area $0.5\times0.5^{\prime\prime}$.
If after the widest binning the criteria aren't satisfied, the value of the spaxel in the 2D maps is set to nan.
Spaxels which pass the criteria after binning are represented as a single spaxel in the 2D maps, and so a smoothing of the fields are expected in the lowest S/N regions.

The process is automated and applied to each of the KDS galaxies.
We find 54 galaxies with spatially resolved OIII emission, with this defined as being those where the spatial extent of the OIII is equivalent to or greater than the seeing disk, and the distribution of these galaxies across the pointings is shown in Table \ref{tab:pointings}.

\subsection{Deriving Morphological Properties}
For a robust interpretation of the observed velocity fields, it is necessary to separately determine the morphological properties of the galaxies from the available high resolution photometry.
The first of these properties is the galactic axis ratio, $A_{r}$.
As suggested in \cite{Holmberg1958}, by modelling the disk galaxies as an oblate spheroid the inclination angle can be recovered from the observed axial ratio as shown in Equation \ref{eq:holmberg_i}, where $\frac{b}{a} \equiv A_{r}$ is the ratio of minor to major axis of an ellipse fit to the galaxy profile on the sky, $i$ is the inclination angle and $\alpha$ is the axial ratio of an edge-on system, and so is dependent upon an assumption about the `disk thickness'.

\begin{equation}\label{eq:holmberg_i}
   cos^{2}i = \frac{\left(\frac{b}{a}\right)^{2} - \alpha^{2}}{1 - \alpha^{2}}
\end{equation}

A value for this parameter can be constrained by looking at the distribution of derived axis ratios, which for a draw from disk galaxies randomly orientated on the sky should be flat \citep{Sandage1970}.
The value of $A_{r}$, then, at which this cuts off can be thought of a population disk thickness and used universally (ignoring variation in disk thickness throughout the sample).
The inclination angle calculated for each galaxy can be used to correct the observed velocity field to the intrinsic, with this correction factor becoming larger as the galaxy becomes more `face-on'.
The second important parameter is the morphological Position Angle (PA), which is the direction of the photometric major axis of the galaxy on the sky.
This would be expected to coincide with the kinematic axis in the case of a rotating disk galaxy.
The discrepancy between morphologial and kinematic position angles is an important indicator of sub-structure in the morphology \citep{Queyrel2012,Wisnioski2015}.
These values are expected to agree, particularly for low axial ratios, and deviations can indicate clumps or mergers, which may influence the underlying kinematics. 
However for compact, unresolved morphologies there are cases where there is simply not enough information to predict the axis along which velocity gradient will be maximised.

The third essential morphological parameter we will use is the half-light radius, $r_{H}$, as this provides a way to extract circular velocities at some fiducial distance from the centre of the galaxy.
As discussed throughout the introduction, there is no universal definition of what the circular velocity is, and so to compare with previous work it is necessary to extract values consistently from the derived rotation curves.
Connected to this is the issue of extrapolating model rotation curves or two-dimensional rotation maps beyond the last measured datapoints, which may not extend to the rotation curve plateau due to poor S/N in this regions.
Knowledge of the $r_{H}$ gives a fixed point of extrapolation and a clear measure of how far beyond the data this extrapolation is made.
Second, since beam smearing affects all of these measurements and tends to flatten velocity gradients, `Durham' (REFERENCE PROPERLY) have invested time into modelling the extent to which this can be corrected.
The correction is a function of the ratio between $r_{H}$ and PSF size as shown in Figure DURHAM, and so accurately measuring the effective radius of a galaxy vindicates the use of this curve to recover intrinsic circular velocities. 

We use the longest wavelength, highest resolution imaging for the fitting of morphological parameters, which in this case is HST F160W band, and opt to use GALFIT \citep{Peng2010_galfit} to recover the parameters as described in section \ref{subsubsec:galfitting}.

\subsubsection{Applying GALFIT to the imaging data}\label{subsubsec:galfitting}
We use GALFIT to fit 2-D analytic functions, convolved with the system PSF, to the observed HST F160W images in both GOODS-S and SSA22 in a consistent way.
The GOODS-S imaging data used is the latest release of the total field, available via the CANDELS \citep{Grogin2011,Koekemoer2011} data access portal\footnote{\tiny{\url{http://candels.ucolick.org/data_access/Latest_Release.html}}}.
For SSA22 we make use of archival HST imaging\footnote{\tiny{\url{https://archive.stsci.edu/hst/search.php}}} (Proposal ID 13844, P.I. Lehmer and Proposal ID 11735, P.I. Mannucci) covering the SSA22-a field.
In both cases we first run SExtractor \citep{Bertin1996} across the full fields to recover initial input parameters and segmentation maps for running GALFIT, and then extract postage stamp regions around the galaxies in the KDS sample.
We fit both single \Sers profiles with a floating \Sers index (n) and also attempt a bulge/disk decomposition by fitting two components; one with n = 1 and one with n = 4.
The $\chi ^{2}$ of each fit is evaluated and the model with the lowest value is selected as the most suitable representation of the flux profile of the galaxy.
In both cases the adopted PSF is a hybrid between the Tiny Tim H-band model \citep{Krist2011} in the PSF centre and an empirical stack of stars observed in the H-band for the wings \citep{VanderWel2012}, which has been shown to produce more reliable results than either the Tiny Tim model or an empirical stack alone.
Crucially, what we gain from this analysis are our best guesses at the three morphological parameters we require to support the kinematic analysis of \cref{subsec:Kinematic Properties}, namely $r_{H}$, $A_{r}$ and the morphological PA.
Another benefit of studying the high resolution imaging is to be able to interpret peculiarities in the velocity and velocity dispersion fields.
These peculiarities are usually unexpected discontinuities in the rotation fields, or extreme broadening of the dispersion field above that expected from beam smearing (see \cref{subsubsec:beam smearing corr}) or in regions apart from the centre of the galaxy.
Figure \ref{fig:morpho_kinematics} provides two such examples, where the Hubble imaging reveals counterparts to the central galaxy, providing two explanations for the patterns in the observed fields. 
Either the counterpart object is spatially close to the main galaxy, but significantly offset in distance or the objects are close enough to be interacting gravitationally, thus disturbing the gas from circular motion and boosting the dispersion values. 
% is this a valid thing to say? 
% is there any way to distinguish between these two different situations?  
Regardless of which of these situations is really happening, the Hubble imaging enhances our ability to classify each of our targets as multiple component, isolated rotating or isolated disturbed.
Classification into different categories is necessary to recognise the galaxies which have already formed a gravitationally supported and rotating disk of stars and gas.
It is only for these galaxies that a maximum circular velocity has any meaning, the derivation of which is the primary concern of this paper.
The other galaxies have either not yet formed a rotating disk and will do so at a later time, or gravitational interractions with another object have disrupted a disk that was already in place (and we may be witnessing a snapshot of a particular part of that process), or they were destined to never form a disk, probably due to a paucity of initial angular momentum, but yet we still detect OIII emission from ionised regions across the spatial extent of the galaxy.   
We present the morphologcial parameters in Table params, and also provide a classification flag for each object (or group of objects) based upon visual inspection of the Hubble imaging and the velocity fields.
More details of the method used to classify our target galaxies are provided in Appendix \ref{app:morpho-kinematic}.
For those galaxies which are classified as isolated rotators, we carry on to derive the kinematic properties as described in \cref{subsec:Kinematic Properties}.
% Need to answer the question which has been raging for a long time - 
% how do you classify a galaxy for the following section.   


\begin{figure*}
\centering
\includegraphics[width=0.8\textwidth]{morpho_kinematics.png}
\caption{High resolution imaging is important not only for reliably recovering the morphological parameters, but also to explain peculiar features in the observed velocity and velocity dispersion fields.
In the two examples above, multiple components in the HST imaging clearly correspond to marginally different OIII emission wavelengths, which manifest as velocity shifts in individual spaxels and broadens the observed dispersion pattern where the emission lines at different wavelengths merge.}
\label{fig:morpho_kinematics}
\end{figure*}


\begin{figure*}
    \centering
    \begin{subfigure}[t]{0.32\textwidth}
        \centering
        \includegraphics[height=1.5in]{axis_ratio_distributions.png}
        \caption{axis ratio distributions}
    \end{subfigure}%
    ~
    \begin{subfigure}[t]{0.32\textwidth}
        \centering
        \includegraphics[height=1.5in]{size_distributions.png}
        \caption{size distributions}
    \end{subfigure}
    ~
    \begin{subfigure}[t]{0.32\textwidth}
        \centering
        \includegraphics[height=1.5in]{sersic_distributions.png}
        \caption{sersic distributions}
    \end{subfigure}
    \caption{Comparison of physical properties with}
\end{figure*}


\subsection{Kinematic Properties}\label{subsec:Kinematic Properties}
% WHEN WILL THE SAMPLE REFINEMENT HAPPEN? % 
% i.e. should this be written in a way that suggests %
% we attempt to extract the kinematic parameters for all % 
% the galaxies in the sample? Or just the clear rotators? % 
% need to have a clear statement about how that selection is done %
% which is a combination of the morphology and the observed dynamics % 
% we can afford to be eyeballing things here because of the small sample size %
% some people would kill for this data - do the analysis justice and do % 
% something sensible with the galaxies that don't make it into the kinematic % 
% analysis part %


As mentioned in the introduction, there is no universal standard for defining the maximum circular velocity, particularly when that value has been extrapolated from the data using a fitted model, of which there are two types.
The first is something physically motivated, taking into account the potential-well of the galaxy through knowledge of the mass distribution (e.g. \citep{Gnerucci2011,Wisnioski2015,ForsterSchreiber2009}).
The second is a phenomenological model, which assumes a fixed function known to well describe observed galactic rotation curves which flatten at large radii (e.g. \citep{Stott2016}).
Our main concern when presenting the work carried out for this section is to clearly distinguish between observed velocities and those inferred from modelling.
This is particularly important when dealing with galaxies at $z > 3$, since we often do not have enough S/N away from the centre of the galaxy to observe the flattening of the rotation curve and hence infer a maximum velocity.  
There are thus two clear reasons for constructing model velocity fields and fitting these to the observed data for the KDS rotating sample; the first is to extrapolate the observations smoothly to some fiducial radius from which a maximum velocity can be extracted; the second is to estimate the effect of beam smearing on both the velocity and velocity dispersion fields so that we recover the set of intrinsic galaxy parameters which best describe the observations.
We construct these models in two independent ways to understand the differences that arise from changes to the modelling procedure, these being the 1D and 3D fitting methods described in the following subsections. 

%What's the point in writing about the 2D and 1D modelling? 
%Want to mention the extrapolation of the data where the S/N ratio is
%too low to measure a value.
%download barolo and download galpak - compare the two.
%also compare with the Durham results and compare with my
%own results from extracting the data using a gaussian fit.

\subsubsection{3D Modelling}\label{subsubsec:3d modelling}
By simulating a full datacube and fitting the measured quantities to the data, we can constrain both the shape of the velocity field in the spatial direction (in particular the maximum circular velocity) and also derive an estimate for the effect that beam smearing has on the velocity and velocity dispersion fields.
For each galaxy in the isolated rotating sample we construct a model cube with the same spatial dimensions as the data and populate each spaxel with an OIII emission line that has central wavelength determined by the systemic velocity of the galaxy.
This central wavelength is then shifted using the velocity derived at that spaxel from the velocity field model.  
Following the procedure now of numerous authors \citep[e.g.]{Swinbank2012,Stott2016,Mason2016} the velocity field of the gas is modelled as a thin disk with the discrete velocity points along the KA determined by equation \ref{eq:arctangent}, which has been found to well match the rotation curves of galaxies in the local universe:

\begin{equation}\label{eq:arctangent}
   v_{r} = \frac{2}{\pi}v_{asym}arctan\left(\frac{r}{r_{t}}\right)
\end{equation}

\noindent
with r measured as the distance from the centre of rotation, (xcen, ycen).
This model is then projected onto the cube using the inclination angle determined in \cref{subsubsec:galfitting} and sampled with 4 times higher spatial resolution than the KMOS raw data before binning back to the previous resolution.
The increase in model spatial resolution is necessary to capture the steep velocity gradient across the central region of the rotating galaxy, implied by the use of the arctangent function.
If the model is constructed with equivalent spatial resolution to the data, the rotation curve has a flatter central profile which mimics the effect of beam smearing, which is exactly what we want to correct for. 
The velocity decreases with a factor of cos($\phi$) from the KA, where $\phi$ is the angle measured clockwise from the KA, so that the KA points to the positive side of the velocity field.
In principle then the model has 5 free parameters; $\{(xcen, ycen), KA, rt, v_{asym}$, rather than the 6 free parameters in previous works \citep[e.g.]{Stott2016} since the inclination to the morphological value.
The intrinsic flux profile of these simulated OIII lines in the spatial direction is determined using the galfit model derived in section \cref{subsubsec:galfitting}, also constructed at 4 times higher spatial resolution than the KMOS cube and rebinned for the same reasons, and the intrinsic velocity dispersion of the line is set to follow a flat distribution with a default width of 50kms$^{-1}$.
This simulated intrinsic OIII cube is then convolved slice by slice with the atmospheric seeing profile determined from fitting the collapsed, stacked standard star cube in each pointing with an elliptical gaussian (the recovered seeing values are reported in table \ref{tab:pointings}).
The velocity and velocity dispersion values are then re-measured in each spaxel and compared with the intrinsic model values to assess the beam smearing impact.
Convolution with the seeing profile produces a peak in the modelled velocity dispersion field at the centre of rotation and flattens the observed velocity field perpendicular to the KA.
This provides a beam smearing correction for the velocity dispersion in every spaxel which is equal to the difference between the assumed intrinsic velocity dispersion and the re-measured velocity dispersion after convolution, i.e. $\sigma_{bs} = \sigma_{model} - \sigma_{assumed}$.
This is computed for each spaxel to provide the 2D $\sigma_{bs}$ correction map.
The impact of the value of $\sigma_{assumed}$ on the magnitude of $\sigma_{bs}$ correction is discussed further in section \cref{subsubsec:sigma_errors}.
We fit the beam smeared velocity field to the observed velocity field, using MCMC sampling with the python `emcee' algorithm \citep{Foreman-Mackey2013} to vary the intrinsic model parameters, and seek the combination of parameters which maximises the log-Likelihood function given by equation \ref{eq:likelihood}, which fully accounts for the errors on the observed velocity field.

\begin{equation}\label{eq:likelihood}
   ln\Lagr = \frac{-0.5\sum_{i=1}^{N}(d_{i} - M_{i})^{2}}{\sigma_{v}^{2}} - ln\left(\frac{1}{\sigma_{v}^{2}}\right)
\end{equation}

In practice we fix the (xcen,ycen) parameters of the galaxy to the location of the stellar continuum, a proxy for the centre of the gravitational potential well, recovered from the collapsed KMOS cube, leaving only 3 free parameters to vary in the MCMC sampling.
This assumes that the gas and the stars in the galaxy share a common centre of mass, an assumption which is at least verified for the vast majority of SFGs in the local and intermediate redshift universe.
Due to the faintness of the continuum, there are several galaxies for which we cannot reliably estimate the rotation centre using this method, and instead opt to set (xcen,ycen) to coincide with the peak OIII flux location.
This differs from the method of fixing the rotation centre at the midpoint between the velocity extrema, described in \cite{Wisnioski2015} and \cite{Rodrigues2016}.
In general the S/N of our sample is low at the galaxy outskirts and for many of the isolated rotating sample we don't observe the flattening of the velocity field.
This suggests that the velocity midpoint may be biased towards the side of the galaxy which shows the higher velocity value and so we instead use the physically motivated continuum centre where possible.

The MCMC sampling provides a distribution of parameter values around those which provide the maximum likelihood, an example of which is shown in Figure \ref{fig:8543_mcmc}.
The 16th and 84th percentiles of these distributions will be used to assess the model uncertainties as described in \cref{subsubsec:model_errors}.


\begin{figure}
\centering
\includegraphics[width=0.45\textwidth]{8543_mcmc.png}
\caption{Example parameter grid for 8543}
\label{fig:8543_mcmc}
\end{figure}

The best fitting smeared and intrinsic models for the isolated rotating sample extracted along the KA, along with shaded uncertainty ranges, are plotted in Appendix A1 for the GOODS-S sample and Appendix A2 for the SSA sample.
Overplotted onto the models are the data values and uncertainties extracted along the KA.
This figure is accompanied by the HST imaging for each galaxy (HST F160W in GOODS-S and ACS F814W in SSA22\footnote{There are 2 galaxies in the SSA sample which did not have ACS F814W imaging, but had F160W imaging with lower integration time, and 3 galaxies not covered by either F160W or ACS 814W imaging, in which case we must resort to ground based SUBARU i-band imaging}), the best fit galfit model and residuals, the narrowband OIII image, 2D maps of the OIII Flux, observed velocity, observed dispersion, best fitting smeared model velocity, data - model residuals, intrinsic velocity dispersion (observed dispersion corrected for beam smearing and instrumental resolution as described in \cref{subsubsec:param_extraction}) as well as a plot of 1D extractions along the observed, intrinsic and beam smeared dispersion maps.
After the modelling, the kinematic parameters must be extracted from the 2D intrinsic fields, as described in \cref{subsubsec:param_extraction}.   

\subsubsection{Kinematic Parameters Extraction}\label{subsubsec:param_extraction}
The crucial information to extract for the model grids described above for each galaxy, both for this study and for comparison to previous studies, is a measure of the maximum intrinsic circular velocity, $V_{max}$, and a single measurement of the intrinsic velocity dispersion, $\sigma_{int}$.
Previous studies have measured these quantities in several different ways:
MENTION GNERUCCI, FS, WISNIOSKI, STOTT, LAW etc. 
In some of the isolated rotating sample, the extracted $V_{max}$ value involves significant extrapolation ($> 40kms^{-1}$) above the last observed data point.
This is a combination of the beam smearing effect leading to a flatter observed velocity field at small radii and the low S/N of this very high redshift sample limiting the radius at which the final observed data point can be extracted.
We hope to be transparent about when large extrapolations are happening, usually as a consequence of a smaller ratio $\frac{R_{KMOS}}{R_{e}}$, where $R_{KMOS}$ is the radius of the last extracted KMOS datapoint and $R_{e}$ is the effective radius measured with GALFIT, by presenting the observed $V_{max}$ values, the beam smeared model $V_{max}$ and the intrinsic model $V_{max}$ in the table of kinematic properties.

\subsubsection{3D modelling error estimates}\label{subsubsec:model_errors}


\subsubsection{1D Modelling}\label{subsubsec:1d modelling}
\subsubsection{Beam Smearing Correction}\label{subsubsec:beam smearing corr}




  




\subsection{Sample Statistics}



Fig.~\ref{fig:distributions}, Table~\ref{tab:KDS}.



\section{ANALYSIS \& RESULTS}\label{sec:results}

\begin{figure}
\centering
\includegraphics[width=0.4\textwidth]{tf_binned.png}
\caption{star forming main sequence}
\label{fig:distributions}
\end{figure}


\section{DISCUSSION}\label{sec:discussion}

\begin{itemize}
    \item Surface brightness dimming
    \item beam smearing
    \item Different modelling techniques
    \item 
\end{itemize}

\section{CONCLUSIONS}\label{sec:conclusions}



\section*{Acknowledgements}

This work is based on observations taken by the CANDELS Multi-Cycle Treasury Program with the NASA/ESA HST, which is operated by the Association of Universities for Research in Astronomy, Inc., under NASA contract NAS5-26555.

%%%%%%%%%%%%%%%%%%%%%%%%%%%%%%%%%%%%%%%%%%%%%%%%%%

%%%%%%%%%%%%%%%%%%%% REFERENCES %%%%%%%%%%%%%%%%%%

% The best way to enter references is to use BibTeX:

%\bibliographystyle{mnras}
%\bibliography{example} % if your bibtex file is called example.bib


% Alternatively you could enter them by hand, like this:
% This method is tedious and prone to error if you have lots of references
\clearpage 
\bibliographystyle{apj.bst}
%\bibliography{/usr/local/texlive/texmf-local/bibtex/bib/ojt.bib}
\bibliography{/Users/owenturner/Documents/PhD/KMOS/Latex/Bibtex/library.bib}

%%%%%%%%%%%%%%%%%%%%%%%%%%%%%%%%%%%%%%%%%%%%%%%%%%

%%%%%%%%%%%%%%%%% APPENDICES %%%%%%%%%%%%%%%%%%%%%

\appendix

\section{Rotating Galaxies plots}

\begin{figure*}
\centering
\includegraphics[width=\textwidth]{combine_sci_reconstructed_bs006516_grid_fixed.png}
\includegraphics[width=\textwidth]{combine_sci_reconstructed_bs006541_grid_fixed.png}
\includegraphics[width=\textwidth]{combine_sci_reconstructed_bs008543_grid_fixed.png}
\caption{Distributions of the physical properties of KDS galaxies in both GOODS-S and SSA22}
\label{fig:grids}
\end{figure*}

\section{Morpho-Kinematic Classification}\label{app:morpho-kinematic}

%%%%%%%%%%%%%%%%%%%%%%%%%%%%%%%%%%%%%%%%%%%%%%%%%%


% Don't change these lines
\bsp    % typesetting comment
\label{lastpage}
\end{document}

% End of mnras_template.tex