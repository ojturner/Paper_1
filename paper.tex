\documentclass[fleqn,usenatbib]{mn2e}

\usepackage[draft]{hyperref}
\hypersetup{
  colorlinks,
  citecolor=blue,
  linkcolor=red,
  urlcolor=cyan}

% MNRAS is set in Times font. If you don't have this installed (most LaTeX
% installations will be fine) or prefer the old Computer Modern fonts, comment
% out the following line
\usepackage{newtxtext,newtxmath}
\usepackage{subcaption}
\usepackage{lscape}
\usepackage[LGRgreek]{mathastext}

% Depending on your LaTeX fonts installation, you might get better results with one of these:
%\usepackage{mathptmx}
%\usepackage{txfonts}

% Use vector fonts, so it zooms properly in on-screen viewing software
% Don't change these lines unless you know what you are doing
\usepackage[T1]{fontenc}
\usepackage{ae,aecompl}

% fancy section symbol
\usepackage{cleveref}
\crefname{section}{$\S$}{$\S\S$}
\Crefname{section}{$\S$}{$\S\S$}

%%%%% AUTHORS - PLACE YOUR OWN PACKAGES HERE %%%%%

% Only include extra packages if you really need them. Common packages are:
\usepackage{graphicx}   % Including figure files
\graphicspath{{Users/}{owenturner/}{Documents/}{PhD/}{KMOS/}{paper_1/}{Figures/}}
\usepackage{amsmath}    % Advanced maths commands
\usepackage{amssymb}    % Extra maths symbols
\usepackage[flushleft]{threeparttable}

%%%%%%%%%%%%%%%%%%%%%%%%%%%%%%%%%%%%%%%%%%%%%%%%%%

%%%%% AUTHORS - PLACE YOUR OWN COMMANDS HERE %%%%%

\newcommand{\Sers}{S\'{e}rsic }
\newcommand{\Lagr}{\mathcal{L}}
\newcommand\tab[1][1cm]{\hspace*{#1}}

% Please keep new commands to a minimum, and use \newcommand not \def to avoid
% overwriting existing commands. Example:
%\newcommand{\pcm}{\,cm$^{-2}$} % per cm-squared

%%%%%%%%%%%%%%%%%%%%%%%%%%%%%%%%%%%%%%%%%%%%%%%%%%
%%%%%%%%%%%%%%%%%%%  PAGE %%%%%%%%%%%%%%%%%%%

% Title of the paper, and the short title which is used in the headers.
% Keep the title short and informative.
\title[KDS: dynamical properties of z$\sim3.5$ galaxies]{The KMOS Deep Survey (KDS): dynamical measurements of $z\sim3.5$ main sequence galaxies\thanks{Based on observations obtained at the Very Large Telescope of the European Southern Observatory. Programme IDs: }}

% The list of authors, and the short list which is used in the headers.
% If you need two or more lines of authors, add an extra line using \newauthor
%\author[O.J. Turner et al.]{
%O. J. Turner,$^{1,2,\thanks{E-mail: turner@roe.ac.uk (OJT)}}$
%M. Cirasuolo,$^{1,2}$
%C. M. Harrison,$^{2,3}$
%J. Dunlop,$^{1}$
%R. J. McLure,$^{1}$\newauthor
%A. M. Swinbank$^{4,3}$
%H. L. Johnson$^{3,4}$
%\\
%$^{1}$SUPA\thanks{Scottish Universities Physics Alliance}, Institute for Astronomy, University of Edinburgh, Royal Obervatory, Edinburgh EH9 3HJ\\
%$^{2}$European Southern Observatory, Karl-Schwarzschild-Str. 2, 85748 Garching b. M{\"u}nchen, Germany\\
%$^{3}$Centre for Extragalactic Astronomy, Durham University, South Road, Durham, DH1 3LE, U.K.\\
%$^{4}$Institute for Computational Cosmology, Durham University, South Road, Durham, DH1 3LE, U.K.
%}

\author[O.J. Turner et al.]{
O. J. Turner,$^{1,2,\thanks{E-mail: turner@roe.ac.uk (OJT)}}$
\\
$^{1}$SUPA\thanks{Scottish Universities Physics Alliance}, Institute for Astronomy, University of Edinburgh, Royal Obervatory, Edinburgh EH9 3HJ\\
$^{2}$European Southern Observatory, Karl-Schwarzschild-Str. 2, 85748 Garching b. M{\"u}nchen, Germany
}

% These dates will be filled out by the publisher
\date{Accepted XXX. Received YYY; in original form ZZZ}

% Enter the current year, for the copyright statements etc.
\pubyear{2017}

% Don't change these lines
\begin{document}
\label{firstpage}
\pagerange{\pageref{firstpage}--\pageref{lastpage}}
\maketitle

% Abstract of the paper
\begin{abstract}
We present dynamical measurements of 77 [O~{\sc III}]$\lambda5007$ resolved main sequence star-forming galaxies at $z\sim3.5$ in the mass range $9.5 < log(M_{\star}[M_{\odot}]) < 10.5$ from the KMOS (K-band Multi-Object Spectrograph) Deep Survey (KDS).
These measurements constitute the largest sample at $z > 3$ utilising Integral-Field Spectroscopy (IFS), and provide robust constraints on the internal dynamics, via the intrinsic velocity dispersion $\sigma_{int}$ and rotation velocities, $V_{C}$, of galaxies in the high redshift universe.
We have made careful comparisons to other IFS samples which have been selected consistently, span a wide redshift baseline and use ionised gas emission to trace galaxy dynamics. 
The KDS mean dispersion value is $\sigma_{int} = 70.8^{+3.3}_{-3.1} kms^{-1}$, in agreement with the increase of $\sigma_{int}$ with redshift reported across the comparison samples.
However we find that only $39 \pm 8\%$ of our galaxies are rotationally-dominated ($V_{C}/\sigma_{int}$ > 1), with the sample average $V_{C}/\sigma_{int}$ value much less than at lower redshift.
The rotation dominated KDS galaxies show no clear offset from the local rotation velocity-stellar mass (i.e. $V_{C}-M_{\star}$) relation (the inverse of the stellar mass Tully-Fisher relation), although a smaller fraction of the galaxies sit on the relation due to the increase in the dispersion dominated fraction.
The galaxies show a decrease in specific angular momentum (angular momentum divided by stellar mass) of $\sim0.6-0.7$dex in comparison to $z=0$ disk galaxies, with this evolution the result of the gradual size evolution of the star-forming population over cosmic time.
Due to these more compact sizes, the KDS galaxies show a paucity of virial mass in comparison to stellar mass, which may be explained through an additional virial component originating in random motions, as traced by $\sigma_{int}$.
These observations are in line with a simple equilibrium model picture, in which random motions are boosted in high redshift galaxies by the increase in gas fractions, accretion efficiency, specific star-formation rate and stellar feedback.
\end{abstract}

% Select between one and six entries from the list of approved keywords.
% Don't make up new ones.
\begin{keywords}
galaxies:kinematics and dynamics -- galaxies:evolution -- galaxies:integral field spectroscopy
\end{keywords}

%%%%%%%%%%%%%%%%%%%%%%%%%%%%%%%%%%%%%%%%%%%%%%%%%%

%%%%%%%%%%%%%%%%% BODY OF PAPER %%%%%%%%%%%%%%%%%%

\section{INTRODUCTION}

In the hierarchical collapse scenario \citep{Rees1977,Fall1980}, matter density perturbations, which originate from some set of cosmological initial conditions, grow over time before condensing out of the general expansion to form dark matter halos \cite{Silk1968,Peebles1970}.
It is at the centre of these halos that baryons collect, cool and begin to form stars (\cite{Birnboim2003,Keres2005,Dekel2006} and see e.g. \cite{Mo2010} for a recent review), with these systems constituting the first galaxies.
One crucial ingredient shaping the subsequent evolution of this system is the transfer of angular momentum from the collapsing gas, as a protogalaxy which is rich in angular momentum is likely to form a turbulent, rotating disk, whereas if the angular momentum is low the formation of a spherical system dominated by random motions will ensue (e.g. \cite{Fall1983}). 

It has been observationally determined that the galaxy population is bimodal in many physical properties at each redshift slice at which it is studied (e.g. as listed in \cite{Dekel2006}), with a preference for the most massive galaxies to lie in the red sequence, characterised by red optical colours, low Star Formation Rates (SFRs) and spherical morphologies, and less massive galaxies in the blue sequence, characterised by blue colours, high star formation rates and disky morphologies. 
For these blue, star-forming galaxies (SFGs) there is a roughly linear correlation between SFR and stellar mass ($M_{\star}$) \citep[e.g.][]{Daddi2007,Noeske2007,Elbaz2007}, in the sense that galaxies which have already accumulated a larger stellar population tend to have higher SFRs.
This correlation, or `main sequence' underpins what is known as the equilibrium model, in which the SFR of galaxies is regulated by the availability of gas, and outflows and accretion events maintain a rough equilibrium as the galaxy evolves \citep[e.g.][]{Dave2012,Lilly2013,Saintonge2013}.
The main sequence has been studied comprehensively, using multi-wavelength SFR tracers, between $0 < z < 3$ \citep[e.g.][]{Rodighiero2011,Karim2011,Whitaker2012,Behroozi2013b,Whitaker2014,Rodighiero2014,Speagle2014,Pannella2014,Sparre2015,Lee2015,Schreiber2015,Renzini2015,Nelson2016}, finding evolution of the relation towards higher SFRs at fixed $M_{\star}$ with increasing redshift, reflecting the increase of the cosmic star formation rate density in this redshift range \citep{Madau_2014}.
At each redshift slice, galaxies on the main sequence are undergoing secular evolution regulated by their gas reservoirs, and so selecting such populations offers the chance to explore the evolution of the physical properties of typical SFGs across cosmic time.
This assumes that high redshift main sequence galaxies are the progenitors of their lower redshift counterparts, which may not be the case \citep[e.g.][]{Gladders2013,Kelson2014,Abramson2016b}, and also assumes that something can be learned about galaxy evolution (i.e. how individual galaxies develop in physical properties over time) by studying the mean properties of populations at different epochs.
The picture is complicated by the addition of major and minor mergers which can rapidly change the physical properties of galaxies \citep[e.g.][]{Toomre1977,Lotz2008,Conselice2011,Conselice2014}, and the relative importance of in-situ, secular stellar mass growth vs. stellar mass aggregation via mergers is the subject of much work involving both observations and simulations \citep[e.g.][]{Robaina2009,Kaviraj2012,Stott2013,Lofthouse2016,Qu2016}. 
To account for the growing number density of quiescent galaxies from z$\sim 2.5$ to the present day \citep[e.g.][]{Bell2004,Faber2007,Brown2007,Ilbert2010,Brammer2011,Muzzin2013,Buitrago2013} there must also be some process which shuts off star-formation within main sequence galaxies (i.e. quenching) which, to explain observations, is a function of both mass and environment \citep{Peng2010}.

Recent cosmological volume simulations are able to resolve the complex interplay of baryonic processes which are at work as galaxies evolve, and can track the development of individual galaxies from early stages, to maturity and through quenching \citep{Dubois2014a,Vogelsberger2014b,Schaye2015}.
Observations can aid the predictive power of such simulations by providing constraints on the evolving physical properties of galaxy populations which must be reproduced across cosmic time.
The observed dynamical properties of galaxies contain information about the transfer of angular momentum between dark matter halos and the baryons, and the dissipation of this angular momentum thereafter (through gravitational collpase, mergers and outflows \cite[e.g.][]{Fall1983,Romanowsky2012,Fall2013}), constituting an important set of quantities for simulators to attempt to match.
Developments in both Integral-Field Spectroscopy (IFS) instrumentation and data analysis tools over the last decade have led to the observation of two-dimensional velocity and velocity dispersion fields for large samples of galaxies of different morphological types, spanning a wide redshift range \citep[e.g.][]{Sarzi2005,Flores2006,Epinat2008,ForsterSchreiber2009,Cappellari2011,Gnerucci2011,Epinat2012,Croom2012,Bundy2015,Wisnioski2015,Stott2016} and when interpreted in tandem with high-resolution imaging data from the Hubble Space Telescope (HST) these data provide exquisite information about the range of physical processes which are driving galaxy evolution.
In particular, in recent years the multiplexing capabilities of KMOS have allowed for IFS kinematic observations for large galaxy samples to build up over much shorter timescales \citep{Sobral2013,Wisnioski2015,Stott2016,Harrison2016,Mason2016} providing an order of magnitude boost in statistical power over previous high-redshift campaigns.    

SFGs generally appear to be increasing in random motions with increasing redshift between $0 < z < 3$, as traced by their observed velocity dispersions, $\sigma_{obs}$ \citep{Genzel2008,ForsterSchreiber2009,Law2009,Cresci2009,Gnerucci2011,Epinat2012,Kassin2012,Green2014,Wisnioski2015,Stott2016}.
This has been explained in terms of increased `activity' in galaxies during and before the global peak in cosmic Star Formation Rate Density (SFRD) \citep{Madau_2014}, in the form of higher specific star formation rates (sSFR) \citep{Wisnioski2015}, larger gas reservoirs \citep{Law2009,ForsterSchreiber2009,Wisnioski2015,Stott2016}, more efficient accretion \citep{Law2009}, increased stellar feedback from supernovae \citep{Kassin2012} and turbulent disk instabilities \citep{Law2009,Bournaud2007,Bournaud2016}, all of which combine to increase $\sigma_{obs}$ and complicate its interpretation.
There is also an increasing body of work measuring the relationship between the observed maximum rotational velocity of a galaxy, a tracer for the total dynamical mass, and its stellar mass, known as the stellar mass Tully-Fisher Relation (smTFR) \citep{Tully1977}, with surveys reporting disparate results for the evolution of this relation with redshift \citep[e.g.][]{Puech2008,Miller2011,Gnerucci2011,Tiley2016a,Harrison2017}.
In both cases, systematic differences in measurement and modelling techniques at high redshift, especially with regards to beam smearing corrections, combine with our poor understanding of progenitors and descendants to blur the evolutionary picture which these surveys paint.

In this paper we present results from the KMOS Deep Survey (KDS), which is a guaranteed time programme focussing on the spatially resolved properties of main sequence SFGs at $z\sim3.5$, a time when the universe was building to peak activity.
There are still many open questions which we can begin to answer by studying the emission from regions of ionised gas within individual galaxies at these redshifts.
For example: \textbf{(1)} How do the physical conditions of the ISM such as electron density and ionisation parameter change as a function of spatial position within the galaxy, and what does this tell us about the application of strong line metallicity calibrators at high redshift?
\textbf{(2)} What are the radial gradients in metal enrichment within these galaxies and what can this tell us about the physical mechanisms responsible for redistributing metals?
\textbf{(3)} What are the dynamical properties of main sequence galaxies at this early stage in their lifetimes? 
\textbf{(4)} What are the connections between the chemical properties and the kinematics of the gas, particularly in terms of inflows and outflows of material?

This study focusses on the final question by deriving and interpreting the spatially resolved kinematics of the KDS galaxies, particularly the rotational velocities and velocity dispersions, using the [O~{\sc III}]$\lambda$5007 emission line, discussing what we can learn about the nature of galaxy formation at $z \sim 3.5$ and forming evolutionary links with lower redshift work.
We seek also to minimise the impact of systematic differences introduced by the differing approaches in defining and extracting kinematic parameters when interpreting these trends.  
By making use of KMOS we have been able to spatially resolve [O~{\sc III}]$\lambda$5007 emission in 44 galaxies with integration times ranging between 7.5-9 hours on source, with these galaxies, which span the mass range $9.5 < log(M_{\star}[M_{\odot}]) < 10.5$, marking a leap in the statistics of galaxies observed in IFS at $z > 3$.

The structure of the paper is as follows. In \cref{sec:Survey_and_data} we present the survey description, sample selection, observation strategy and data reduction, leading to stacked datacubes for each of the KDS galaxies.
In \cref{sec:extracting_properties} we explain how two-dimensional maps of physical properties were extracted from our stacked datacubes and the treatment of the HST imaging to derive the morphological properties of our galaxies.
We also define a classification procedure which separates the KDS galaxies into a merger and modelling sample.
In \cref{sec:Kinematic_Properties} we present the kinematic modelling applied to the KDS galaxies, including the beam smearing correction, and the method chosen to extract the kinematic properties following this modelling procedure.
\cref{sec:results} presents an analysis of the derived kinematic parameters, comparing with lower redshift work where possible and drawing conclusions about the evolutionary trends and possible physical mechanisms driving these.
We present our conclusions in \cref{sec:conclusion}.
Throughout this work we assume a standard $\Lambda$CDM cosmology with (h, $\Omega_{m}$, $\Omega_{\Lambda}$) = (0.7, 0.3, 0.7). 

\section{SURVEY DESCRIPTION, SAMPLE SELECTION AND OBSERVATIONS}\label{sec:Survey_and_data}
\subsection{The KDS survey description and sample selection}\label{subsec:survey_intro}
The KDS is a KMOS study of the gas kinematics and chemical composition in 77 SFGs with a median redshift of $z\sim3.5$, probing a representative section of the galaxy main sequence.
The addition of this data approximately triples the number of galaxies observed in IFS at this redshift \citep{Cresci2010,Lemoine-Busserolle2010,Gnerucci2011}, and will allow for a statistically significant investigation of the dynamics and metal content of SFGs during a crucial period of galaxy evolution. 
The two key science goals of the KDS are to investigate the resolved kinematic properties of high redshift galaxies in the epoch of galaxy formation, particularly the fraction of rotating disks and the degree of disk turbulence, and also to study the spatial distribution of metals within these galaxies in the context of their observed dynamics.
To achieve this we require very deep exposure times in excess of 7 hours on source for all galaxies, to reach the S/N required to detect line emission in the outskirts of the galaxies, where the rotation curves begin to flatten, and to achieve adequate S/N across several ionised emission lines within individual spatial pixels (spaxels).

\subsubsection{Sample selection}\label{subsubsec:sample_selection}
Target selection for the KDS sample is designed to pick out SFGs at $z = 3-4$, supported by deep multi-wavelength ancillary data and probing both cluster and field environments.
Within this redshift range the [O~{\sc III}]$\lambda$4959,$\lambda$5007 doublet and the $H\beta$ emission lines are visible in the K-band and the [O~{\sc II}]$\lambda$3727,$\lambda$3729 doublet is visible in the H-band, both of which are observable with KMOS. 
We also want to select a sample of galaxies spanning a representative region of the galaxy main sequence to minimise biases towards a particular subset of the overall star-forming population at these redshifts.
For these reasons, and to ensure a high detection rate of the ionised gas emission lines in the KDS galaxies, we select galaxies in well studied fields that have a previous spectroscopic detection.
To avoid biasing the sample towards blue and highly star-forming galaxies we make no secondary cuts on the basis of mass and flux.
Predominantly, the galaxies have been selected for spectroscopic follow up using the Lyman-Break technique \citep{Steidel1996}, which is especially effective at picking out normal starforming galaxies at $z = 3-4$.
However, a subset of the selected cluster galaxies in the SSA22 field were blindly detected in Ly$\alpha$ emission during a narrowband imaging study of a known overdensity of Lyman Break Galaxies (LBGs) at $z \sim 3.09$ \citep{Steidel2000}.

\subsubsection{GOODS-S}\label{subsubsec:sample_selection_goods}
We select two `field' environment pointings with KMOS in the GOODS-S field, accessible from the VLT and with excellent multi-wavelength coverage, including deep HST WFC3 F160W imaging with $ 0.06^{\prime\prime}$ pixel scale and $ \sim 0.2^{\prime\prime}$ PSF, which is well suited for constraining galaxy morphology \citep{Grogin2011,Koekemoer2011}.
We select targets from the various spectroscopic campaigns which have targetted GOODS-S, including measurements from VIMOS \citep{Balestra2010,Cassata2014}, FORS2 \citep{Vanzella2005,Vanzella2006,Vanzella2008} and both LRIS and FORS2 as outlined in \cite{Wuyts2009}.
These targets must be within the redshift range $3 < z < 3.8$, have high spectral quality (as quantified by the VIMOS redshift flag equal `3' or `4', and the FORS2 quality flag equal `A') and we carefully exclude those targets for which the [O~{\sc III}]$\lambda$5007 or H$_{\beta}$ emission lines, observable in the K-band at these redshifts, would be shifted into a spectral region plagued by strong OH emission.
The galaxies which remain after imposing these criteria are distributed across the GOODS-S field, and we select two regions of spatial overdensity in the south-east (GOODS-S-P1) and north-west (GOODS-S-P2) to fill the KMOS IFUs (noting that the IFUs can patrol a $7.2^{\prime}$ diameter patch of sky during a single pointing).

\subsubsection{SSA22}\label{subsubsec:sample_selection_ssa}
We select a single `cluster' environment pointing from the SSA22 field, \citep{Steidel1998,Steidel2000,Steidel2003,Shapley2003}, which, as mentioned above, is an extreme overdensity of LBG candidates at $z \sim 3.09$.
Thousands of spectroscopic redshifts have been confirmed for these LBGs with follow up observations using LRIS, \citep{Shapley2003}, \citep{Nestor2013}, and when combined with deep B,V,R band imaging with the Subaru Suprime-Cam \citep{Matsuda2004}, deep narrow band imaging at 3640$\AA$ \citep{Matsuda2004} and at 4977$\AA$ \citep{Nestor2011} and archival HST ACS and WFC3 imaging covering large swathes of the field, the ancillary data is in excellent support of integral field spectroscopy, albeit over a shorter wavelength baseline and with shallower exposures than in the GOODS-S field.
Fortunately at $z \sim 3.09$ the [O~{\sc III}]$\lambda$5007 line is shifted into a region of the K-band which is free from OH features and so we fill the KMOS IFUs with galaxies located towards the centre of the SSA22 protocluster (SSA22-CLUSTER). We also add a further field environment pointing to the south of the main galaxy spatial overdensity (SSA22-FIELD), leaving us with three field environment pointings across GOODS-S and SSA22 and a single cluster environment pointing, as summarised in table \ref{tab:pointings}. \footnote{Additional pointings in the COSMOS and UDS fields were originally scheduled as part of the GTO project, however 50$\%$ of the observing time was lost to bad weather during a poor year for Paranal.}

\subsection{Observations and data reduction}\label{subsubsec:observations_and_dr}
\subsubsection{Observations}\label{subsubsec:Obs}
KMOS is a second generation IFS mounted at Nasmyth focal plane on UT1 at the VLT.
The instrument has 24 moveable pickoff arms, each with an integrated IFU, which patrol a region 7.2$^{\prime}$ in diameter on the sky, providing considerable flexibility when selecting sources for a single pointing.
The light from each set of 8 IFUs is dispersed by a single spectrograph and recorded on a 2k$\times$2k Hawaii-2RGHgCdTe near-IR detector, so that the instrument is comprised of three effectively independent modules.
Each IFU has 14$\times$14 spatial pixels which are 0.2$^{\prime\prime}$ in size, and the central wavelength of the K-band filter has spectral resolution of $R \sim 4200$.
To achieve the science goals of the KDS the target galaxies at $3 < z < 3.8$ must be observed in both the K-band, into which the [O~{\sc III}]$\lambda$5007 and $H\beta$ lines are redshifted, and the H-band, into which the [O~{\sc II}]$\lambda$3727,$\lambda$3729 doublet is redshifted, to facilitate both dynamical and abundance measurements.
To this end we decided to observe the SSA22 galaxies with the KMOS HK filter, which has the disadvantage of effectively halving the spectral resolution, but allows for coverage of the H-band and K-band regions simultaneously.
This paper is concerned with the spatially resolved kinematics of the KDS galaxies, with the spatially resolved metallicities reserved for a future work, and so we now focus exclusively on the spatially resolved [O~{\sc III}]$\lambda$5007 measurements in the K-band spectral window.

We prepare each pointing using the KARMA tool \citep{Wegner2008}, taking care to allocate at least one IFU to observations of a `control' star closeby on the sky to allow for precise monitoring of the evolution of seeing conditions and the shift of the telescope away from the prescribed dither pattern (see \cref{subsubsec:datareduction}).
For the four pointings described above and summarised in table \ref{tab:pointings}, we adopt the standard object-sky-object (OSO) nod-to-sky observation pattern, with 300s exposures and alternating $0.2^{\prime\prime}$/$0.1^{\prime\prime}$ dither pattern for increased spatial sampling around each of the target galaxies.
This will later allow for datacube reconstruction with 0.1$^{\prime\prime}$ size spaxels as described in \cref{subsubsec:datareduction}.

The observations were carried out during ESO observing periods P92-P95 using Guaranteed Time Observations (Programme IDs: xxx xxx xxx)
The seeing conditions were excellent for GOODS-S-P1 and GOODS-S-P2 with median K-band seeing of $\sim 0.5^{\prime\prime}$ and for the SSA22-CLUSTER and SSA22-FIELD pointings the K-band seeing ranged between $\sim 0.55-0.65^{\prime\prime}$.
We observed 19-21 targets in each field (\ref{tab:pointings}), with these numbers less than the available 24 arms for each pointing due to the combination of three broken pickoff arms during the P92 observing semester and our requirement to observe at least one control star throughout an Observing Block (OB).

\subsubsection{Data reduction}\label{subsubsec:datareduction}
The data reduction process relied heavily upon the Software Package for Astronomical Reduction with KMOS, (SPARK; \cite{Davies2013}), implemented using the ESO Recipe Execution Tool (ESOREX) \citep{Freudling2013}.
In addition to the SPARK recipes, custom python scripts were run at different stages of the pipeline and will be described throughout this section.

The SPARK recipes are used to create dark frames, flatfield, illumination correct and wavelength calibrate the raw data.
Readout channel bias is addressed using the `LCAL\_XXX.fits' calibration product to identify pixels which are not illuminated during exposures.
The median values of these pixels are used to correct for varying flux baselines between each 64 pixel wide detector readout column, which if left uncorrected leads to flux bandings over the spatial extent of the reduced cubes.
Standard star observations are processed to provide a flux calibration for each detector, which is necessary to account for varying sensitivity across the three independent KMOS modules. 
Following this pre-processing, each of the object exposures is reconstructed independently, using the closest sky exposure for subtraction, to give more control over the construction of the final stacks for each target galaxy.
Each 300s exposure is reconstructed into a datacube with interpolated $0.1\times0.1^{\prime\prime}$ spaxel size, facilitated by the subpixel dither pattern discussed in \cref{subsub:Obs} which boosts the effective spatial resolution of the observations.

Sky subtraction is enhanced using the SKYTWEAK option within SPARK \citep{Davies2007}, which counters the varying amplitude of OH lines between exposures by scaling `families' of OH lines independently to match the data.
Wavelength miscalibration between exposures due to spectral flexure of the instrument is also accounted for by applying spectral shifts to the OH families during the procedure, and in general we that the use of the SKYTWEAK option in the K-band greatly reduces the sky-line residuals. 

Variations in sky-subtraction quality are monitored over the course of the OBs which constitute the final stacks, as we find that the sky-subtraction performance becomes poorer as the telescope pointing approaches the zenith.
This is due to the telescope tracking faster whilst looking through very low airmasses, hence scanning more rapidly over regions of sky with spatially and temporally varying OH emission.
In addition to this we monitor the evolution of the atmospheric PSF and the position of the control stars over the OBs, to allow us to reject raw frames where the averaged K-band seeing rises above $0.8^{\prime\prime}$ and to measure the spatial shifts required for the final stack more precisely.
The telescope tends to drift from its acquired position over the course of an OB, and the difference between the dither pattern shifts and the measured position of the control stars provides the value by which each exposure must be shifted to create the stack. 
As part of this analysis, we tested whether the drift varies across the three KMOS detectors, finding typically that the difference is negligible, suggesting that it is not necessary to sacrifice more than a single IFU for tracking purposes.

We stack all 300s exposures for each galaxy which pass the sky-subtraction and seeing criteria using 3 sigma-clipping, leaving us with a flux and wavelength calibrated datacube for every object in the KDS sample.
We have found that the thermal background is often under-subtracted across the spatial extent of the cube following a first pass through the pipeline, leading to excess flux towards the long wavelength end of the K-band.
To account for this a polynial function is fit, using the python package {\tt LMFIT} \citep{Newville2014} which makes use of the Levenberg-Marquardt algorithm for non-linear curve fitting, to the median stacked spectrum from spaxels in the datacube which contain no object flux, and then subtracted from each spaxel in turn.
This assumes that there is no spatial variation in the required correction, which we have found to be valid for our current sample of galaxies.

The central coordinates of each pointing, the number of target galaxies observed, N$_{obs}$, the number of galaxies with [O~{\sc III}]$\lambda$5007 detected as measured by attempting to fit the integrated galaxy spectrum, N$_{[O~{\sc III}]_{Det}}$, the number with spatially resolved [O~{\sc III}]$\lambda$5007 emission, N$_{[O~{\sc III}]_{Res}}$ (see \cref{sec:extracting_properties}), the on source exposure time and the averaged seeing conditions are listed in Table \ref{tab:pointings}. \footnote{We note that in GOODS-S-P2 we later found that the redshifts for two of our targets were $z < 0.5$ and we omit these from all columns in table \ref{tab:pointings}.}

\begin{table*}
\centering
\begin{threeparttable}
\caption{Summary of KDS pointing statistics}
\label{tab:pointings}
\begin{tabular}{c c c c c c c c c c c c c}

 \hline
Pointing & RA & DEC & N$_{obs}$ & N$_{Det}$ & $\%$ Det. & N$_{Res}$ & $\%$ Res & N$_{Merg}$ & $\%$ Merg$^{*}$ & Band(s) & Exp. Time (ks) & Seeing ($^{\prime\prime}$)  \\
 \hline
 GOODS-S-P1 & 03:32:22 & -27:52:51 & 20 & 16 & 80 & 12 & 60 & 2 & 17 & K, H & 32.4 & 0.50 \\
GOODS-S-P2 & 03:32:35 & -27:43:15 & 17 & 14 & 82 & 11 & 65 & 2 & 19 & K, H & 31.8 & 0.52 \\
SSA22-FIELD & 22:17:11 & 00:15:47 & 19 & 17 & 89 & 11 & 58 & 1 & 9 & HK & 27.8 & 0.57 \\
SSA22-CLUSTER & 22:17:28 & 00:09:54 & 21 & 15 & 71 & 10 & 46 & 8 & 80 & HK & 38.1 & 0.62 \\
 \hline
\end{tabular}
\begin{tablenotes}
      \small
      \item $^{*}$ Note that the Merger percentage is computed with respect to the number of resolved galaxies; the other percentages are computed with respect to the total number of galaxies observed in that pointing.
    \end{tablenotes}
  \end{threeparttable}
  \end{table*}

\subsection{Stellar masses and SFRs}\label{subsec:stellar_masses_and_sfrs}
The wealth of ancillary data in both fields allows for a consistent treatment of the SED modelling, providing physical properties which are directly comparable between the cluster and field environments.
These derived properties are considered in the context of the galaxy main sequence, to verify that the KDS sample contains typical SFGs at $z\sim3.5$.   

\subsubsection{SED fitting and main sequence}\label{subsubsec:sed_fitting}
We use SED modelling to constrain simultaneously the $M_{\star}$ and SFR of the KDS.
During the SED modelling we assume solar metallicity stellar templates, and make use of \cite{Calzetti2000} reddening law with exponentially declining SFRs in both GOODS-S and SSA22.
We have also computed the quantities using 0.2Z$\odot$ templates, finding stellar masses for the galaxies which are on average 0.1 dex lower, but this change does not affect the conclusions of this work.
The median stellar mass for the full observed samples in both GOODS-S and SSA22 is $log(M_{\star}[M_{\odot}]) = 10.0$.

In Figure \ref{fig:main_sequence} we plot the stellar mass (M$_{\star}$) and SFR `main sequence' of the KDS galaxies, in combination with the derived physical properties of $\sim 10,000$ $2.5 < z < 3.5$ SFGs (based upon the `z\_best classification flag') from the 3D-HST survey \citep{Brammer2012,Momcheva2016}.
We also plot both the linear-break and quadratic $z\sim 2.5$ main sequence fits to the 3D-HST data described in \cite{Whitaker2014} with the solid and dashed black lines as well as the MS relation described in \cite{Speagle2014}: $logSFR(M_{\star}, t) = (0.84 - 0.026 \times t)logM_{\star} - (6.51 - 0.11 \times t)$, evaluated at $z\sim3.5$ where the age of the universe is 1.823Gyr, with the green line.
The difference in position of these relations highlights the MS evolution towards higher SFRs at fixed $M_{\star}$ between $z\sim2.5-3.5$.
The KDS galaxies scatter, within the errors, consistently above and below these lines, suggesting that they are mostly regular SFGs in the process of converting their gas into stars (i.e. they are not especially bursty, or devoid of star formation).
There are fewer galaxies plotted in Figure \ref{fig:main_sequence} than observed due to the lack of deep, ground-based J and K-band photometry in SSA22 leaving the SFR unconstrained for several objects.
We omit from this main sequence plot all objects in SSA22 that have J and K band detections lower than 2-$\sigma$ significance, leaving 64\% of the sample across all pointings, and note that the derived (M$_{\star}$) values for the galaxies which are not plotted are consistent with the distribution shown in figure \ref{fig:main_sequence}.
\begin{figure}
\centering
\includegraphics[width=0.49\textwidth]{main_sequence.png}
\caption{We plot the location of the KDS galaxies across the four pointings on the star forming `main sequence', with filled symbols showing galaxies detected in the [O~{\sc III}]$\lambda$5007 emission line galaxies and open symbols those in which no detection has been made. 
Also plotted with the small grey points are individual SFGs from the 3D-HST survey  \protect\citep{Brammer2012,Momcheva2016} at 2 $<$ z $<$ 3 as a reference for the typical relationship between SFR and $M_{\star}$.
The solid black lines and the dashed line show the $z\sim2.5$ broken power-law and quadratic fit to the main sequence respectively, described in \protect\cite{Whitaker2014}.
We include the MS relation described in \protect\cite{Speagle2014}: $logSFR(M_{\star}, t) = (0.84 - 0.026 \times t)logM_{\star} - (6.51 - 0.11 \times t)$, evaluated at $z\sim3.5$ where the age of the universe is 1.823Gyr, with the green line. 
Given the errors associated with the computed SFRs and M$_{\star}$ values, the KDS galaxies scatter evenly above and below the plotted relations, suggesting that these are typical SFGs which sample the main sequence of star formation between 9.5 $< log(M_{\star}[M_{\odot}]) <$ 10.5 and have mean $log(M_{\star}[M_{\odot}]) = 10.0$.
29 of the SSA22 targets do not have SFR measurements due to limited photometry and are not plotted here.
For these galaxies it is still possible to derive an $M_{\star}$ value and we note that the distribution is consistent with the galaxies plotted in this figure.}
\label{fig:main_sequence}
\end{figure}

\subsubsection{Sample summary}\label{subsubsec:sample_summary}
We have observed 77 SFGs spanning $3.0 < z < 3.8$ with the KMOS IFUs as part of the KMOS Deep Survey, and have derived $M_{\star}$ values for all, and SFRs for 64\% of the sample from SED modelling.
Three KMOS pointings cover the field environment, and one pointing covers a cluster environment at the heart of the SSA22 protocluster.
This paper presents the dynamical properties of the KDS \textit{field} sample, with the cluster sample showing a very high merger fraction (as listed in table \ref{tab:pointings} and discussed in \cref{subsec:morpho-kin-class}).
We omit this cluster pointing, SSA22-CLUSTER, from the dynamical analysis presented in the following sections due to the complexity added by galaxy interactions to the interpretation of cluster galaxy velocity fields.
Abundance measurements for the KDS galaxies in both the field and cluster environments will be presented in a future work.

\section{ANALYSIS}\label{sec:analysis}

\begin{table*}
\centering
\begin{threeparttable}
\caption{Physical properties of the KDS field galaxies as measured from SED fitting and from applying GALFIT \protect\citep{Peng2010_galfit}}
\label{tab:phys-props}
\begin{tabular}{ccccccccccc}


 \hline
ID              & RA       & Dec       & z     & K$_{AB}$     & $log(M_{\star}[M_{\odot}])$ & SFR$_{SED}$($M_{\odot}yr^{-1}$) & $A_{r}$ & i$^{\circ}$ & PA$_{morph}^{\circ}$ & R$_{1/2}$(kpc) \\
 \hline
 GOODS-S & & & & & & & & & & \\
 \hline
bs009818        & 03:32:20 & -27:53:01 & 3.706        & 24.18  & 9.79  & 33.0 & 0.8         & 37.0        & 148.0   & 1.24      \\
bs014828        & 03:32:27 & -27:52:26 & 3.562        & 23.58  & 10.15 & 30.0 & 0.31        & 76.0        & 63.0    & 1.61      \\
lbg\_20         & 03:32:41 & -27:52:21 & 3.225        & 24.97  & 9.61  & 4.7  & 0.64        & 52.0        & 1.0     & 1.28      \\
lbg\_105        & 03:32:24 & -27:52:16 & 3.092      & 23.79  & 9.79  & 30.0 & 0.56        & 57.0        & 128.0   & 1.72      \\
bs006541        & 03:32:15 & -27:52:05 & 3.475       & 23.44  & 10.19 & 18.0 & 0.44        & 66.0        & 168.0   & 1.83      \\
b012141\_012208 & 03:32:23 & -27:51:57 & 3.471        & 24.12  & 10.16 & 59.0 & 0.36        & 72.0        & 9.0     & 1.57      \\
b15573          & 03:32:28 & -27:51:00 & 3.583        & 23.6   & 10.37 & 27.0 & 0.28        & 78.0        & 146.0   & 0.52      \\
bs008543        & 03:32:18 & -27:50:50 & 3.474        & 22.73  & 10.67 & 42.0 & 0.5         & 61.0        & 67.0    & 1.59      \\
bs006516        & 03:32:15 & -27:50:46 & 3.215        & 23.94  & 9.85  & 14.0 & 0.5         & 61.0        & 146.0   & 1.91      \\
bs016759        & 03:32:29 & -27:48:53 & 3.602       & 23.85  & 9.99  & 8.8  & 0.65        & 50.0        & 49.0    & 0.87      \\
lbg\_111        & 03:32:42 & -27:45:52 & 3.609       & 24.01  & 9.74  & 27.0 & 0.74        & 42.0        & 80.0    & 0.64      \\
lbg\_38         & 03:32:22 & -27:44:38 & 3.488       & 24.58  & 10.03 & 7.5  & 0.58        & 56.0        & 137.0   & 0.92      \\
lbg\_94         & 03:32:29 & -27:44:12 & 3.367       & 24.54  & 10.01 & 8.1  & 0.22        & 84.0        & 81.0    & 1.16      \\
lbg\_109        & 03:32:21 & -27:43:46 & 3.600       & 24.63  & 9.73  & 9.4  & 0.6         & 54.0        & 119.0   & 1.98      \\
lbg\_24         & 03:32:40 & -27:39:57 & 3.279       & 24.67  & 9.71  & 5.9  & 0.53        & 60.0        & 34.0    & 1.27      \\
lbg\_25         & 03:32:29 & -27:40:22 & 3.322        & 24.95  & 9.44  & 5.8  & 0.3         & 76.0        & 78.0    & 1.18      \\
lbg\_32         & 03:32:34 & -27:41:24 & 3.417      & 23.84  & 10.17 & 52.0 & 0.6         & 54.0        & 40.0    & 1.88      \\
lbg\_113        & 03:32:36 & -27:41:50 & 3.622      & 24.01  & 9.83  & 41.0 & 0.52        & 60.0        & 15.0    & 0.87      \\
lbg\_91         & 03:32:27 & -27:41:52 & 3.170       & 24.65  & 9.8   & 5.7  & 0.58        & 56.0        & 79.0    & 0.89 \\
 \hline
 SSA22 & & & & & & & & & & \\
 \hline
s\_sa22a-d3   & 22:17:32 & +00:11:33 & 3.069 & 23.4618 & 9.88  & -            & 0.39        & 70.0        & 125.0   & 1.78      \\
n3\_009       & 22:17:28 & +00:12:12 & 3.069 & 99.0    & 8.72  & -            & 0.75        & 42.0        & 84.0    & 1.06      \\
lab18         & 22:17:29 & +00:07:52 & 3.101           & 99.0    & 8.66  & -            & 0.85        & 32.0        & 27.0    & 0.46      \\
s\_sa22b-c20  & 22:17:49 & +00:10:14 & 3.196     & 23.9103 & 9.74  & -            & 0.57        & 57.0        & 76.0    & 1.59      \\
n\_c3         & 22:17:33 & +00:10:57 & 3.096       & 24.9637 & 9.94  & -            & 0.71        & 46.0        & 94.0    & 0.56      \\
s\_sa22b-d9   & 22:17:22 & +00:08:04 & 3.084    & 24.2538 & 10.05 & -            & 0.65        & 50.0        & 60.0    & 0.5       \\
lab7\_top     & 22:17:41 & +00:11:28 & 3.095      & 23.8919 & 9.89  & -            & 0.51        & 61.0        & 139.0   & 1.47 \\
$^{*}$lab25         & 22:17:23 & +00:15:51 & 3.067      & 99.0    & 8.78  & -            & 0.57         & 57.3        & -    & 2.4      \\
$^{*}$s\_sa22b-d5  & 22:17:36 & +00:06:10 & 3.175 & 23.7182 & 10.32 & -            & 0.57         & 57.3        & -    & 2.14      \\
$^{*}$s\_sa22b-md25 & 22:17:42 & +00:06:20 & 3.304     & 24.6174 & 9.05  & -            & 0.57         & 57.3        & -    & 2.14      \\
$^{*}$n3\_006       & 22:17:25 & +00:11:18 & 3.069     & 22.9847 & 10.64 & 50.0 & 0.57         & 57.3        & -    & 2.14      

\end{tabular}
\begin{tablenotes}
      \small
      \item $^{*}$ No HST coverage: $PA_{morph}$ unconstrained; $A_{r}$ and $R_{1/2}$ estimated as explained throughout the text.
    \end{tablenotes}
  \end{threeparttable}
  \end{table*}


\subsection{Morphological Measurements}
For a robust interpretation of the observed velocity fields, it is necessary to separately determine the morphological properties of the galaxies from high resolution photometry.
This imaging is used primarily to determine morphological parameters which characterise the size (quantified here through the half-light radius, $R_{1/2}$,), morphological position angle, $PA_{morph}$, and axis ratio, $A_{r}$, of the galaxies and helps to distinguish multiple component objects with small angular separations from objects which are morphologically isolated.
In the following sections we describe the approach we have chosen to recover these parameters.
In section \cref{subsubsection:morph_param_summary} we compare the KDS morphological parameters with those described in \cite{VanderWel2012}, and also with spectroscopically confirmed galaxies in different redshift intervals in order to study the size evolution of SFGs across cosmic time.

\subsubsection{Applying GALFIT to the imaging data}\label{subsubsec:galfitting}
We use GALFIT \citep{Peng2010_galfit} to fit 2D analytic functions, convolved with the PSF, to the observed HST images of the KDS field galaxies across GOODS-S and SSA22 in a consistent way.
The GOODS-S imaging data used is the latest release of the total field in WFC3 F160W band, which traces the rest-frame near-UV at $z \sim 3.5$, available via the CANDELS \citep{Grogin2011,Koekemoer2011} data access portal\footnote{\tiny{\url{http://candels.ucolick.org/data_access/Latest_Release.html}}}.
For SSA22 we make use of archival HST imaging\footnote{\tiny{\url{https://archive.stsci.edu/hst/search.php}}} data in the WFC3 F160W band (P.I. Lehmer: PID 13844; P.I. Mannucci: PID 11735) and the ACS F814W band, tracing $\sim$ 2000$\AA$ light at $z\sim 3.5$ (P.I. Chapman: PID 10405; P.I. Abraham: PID 9760; P.I. Siana: PID 12527).
The HST coverage is shallower in SSA22 (exposure times of $\sim 5$ ks) and the F160W coverage is incomplete, and so we must resort to the bluer ACS F814W data which traces younger stellar populations and may not be sensitive to the majority of the mass distribution within these galaxies.

We first run SExtractor \citep{Bertin1996} across the full fields to recover initial input parameters and segmentation maps for running GALFIT, and then extract postage stamp regions around the galaxies in the KDS sample.
At this redshift, the galaxies are more compact and generally we cannot resolve more complicated morphological features such as spiral arms and bars and so we follow the simple method of fitting \Sers profiles, with the \Sers index fixed to the exponential disk value of n = 1.
During the fitting process all other morphological parameters, including $R_{1/2}$, the central x and y coordinates, $PA_{morph}$ and inclination, are free to vary.
This method has been verified in GOODS-S by comparing the $\chi ^{2}$ recovered from fitting floating \Sers index models, bulge/disk models with both an n = 1 and n = 4 component and from the fixed exponential disk value of n = 1 following the procedure described in \cite{Bruce2012} (with the median ratio of $\chi ^{2}$ values equal unity in all cases). Additionally, for the corresponding objects in the \cite{VanderWel2012} catalogue the median \Sers index value is 1.2.
The use of exponential profiles also facilitates comparison with recent large surveys such as KROSS \citep{Harrison2017}, in which beam smearing correction factors are applied to the derived kinematic parameters as function of exponential disk scale length, $R_{D}$, defined as $R_{1/2} \sim 1.68 R_{D}$ (see \cref{subsubsec:param_extraction}), and KMOS$^{3D}$ \citep{Wisnioski2015}.

In the F160W band the adopted PSF is a hybrid between the Tiny Tim H-band model \citep{Krist2011} in the PSF centre and an empirical stack of stars observed in the H-band for the wings \citep{VanderWel2012}, which has been shown to produce more reliable results than either the Tiny Tim model or an empirical stack alone.
In the F814W band we use the pure Tiny Tim i-band model.

This analysis provides us with three crucial morphological parameters required to support the kinematic analysis of \cref{sec:Kinematic_Properties}, namely the axis ratio $A_{r}$, half-light radius $R_{1/2}$ and the $PA_{morph}$. The errorbars produced by GALFIT are purely statistical and are determined from the covariance matrix used in the fitting, and this results in unrealistically small uncertainties in the derived galaxy parameters \citep{Hausler2007,Bruce2012}.
Throughout the following subsections we discuss adopted errors on each of the morphological parameters.

\subsubsection{Inclination angle}\label{subsubsection:inclination_angle}
We used the derived $A_{r}$ values to determine galaxy inclination angles.
As suggested in \cite{Holmberg1958}, by modelling the disk galaxies as an oblate spheroid the inclination angle can be recovered from the observed axial ratio as shown in Equation \ref{eq:holmberg_i}, where $\frac{b}{a} \equiv A_{r}$ is the ratio of minor to major axis of an ellipse fit to the galaxy profile on the sky, $i$ is the inclination angle and $q_{0}$ is the axial ratio of an edge-on system.
To derive the inclination, we must select a value for $q_{0}$, and following the discussion in \citep{Law2012a} we choose a value appropriate for thick disks, $q_{0} = 0.2$ \citep[e.g.][]{Epinat2012,Harrison2017}.
However, as discussed in \cite{Harrison2017}, varying the $q_{0}$ value by a factor of 2 makes only a small change to the final inclination corrected velocity values, with the difference $<10\%$ in the case of the KDS galaxies.

\begin{equation}\label{eq:holmberg_i}
   cos^{2}i = \frac{\left(\frac{b}{a}\right)^{2} - q_{0}^{2}}{1 - q_{0}^{2}}
\end{equation}

The inclination angle calculated for each galaxy is used to correct the observed velocity field, which is the line of sight component of the intrinsic velocity field, with the correction factor increasing with increasing $A_{r}$.

For the inclination error we adopt a conservative nominal uncertainty of 10$\%$ also, i.e. $\Delta i = i / 10$, after examining the typical magnitude of the uncertainties reported following the Monte Carlo procedure described in \cite{Epinat2012}.
We also note that generally this inclination uncertainty is a function of the recovered axis ratio, since it is more difficult to determine the inclination when the galaxy is more `face-on', but detailed inclination angle uncertainty modelling is beyond the scope of this paper.

\subsubsection{Position angle}\label{subsubsection:position_angle}
The second GALFIT parameter is $PA_{morph}$, which is the direction of the photometric major axis of the galaxy on the sky.
This is expected to coincide with the kinematic position axis, $PA_{kin}$, in the case of a rotating disk galaxy, where the stars and the gas are both rotating around the centre of the gravitational potential well.
Discrepancy between $PA_{morph}$ and $PA_{kin}$ is an important indicator of sub-structure in the morphology \citep[e.g.][]{Queyrel2012,Wisnioski2015,Rodrigues2016}, and deviations can indicate clumps or mergers which may influence the underlying kinematics or bias the derived $PA_{morph}$ towards a particular direction. 
The discrepancy, often labelled $\Delta PA \equiv |PA_{morph} - PA_{kin}|$, is a function of $A_{r}$, as it becomes increasingly difficult to accurately measure $PA_{morph}$ for very inclined galaxies \citep[e.g.][]{Wisnioski2015,Harrison2017}.
We have visually inspected each of the GALFIT maps to check that the derived $PA_{morph}$ follows the galaxy light distribution, and when this is not the case it is usually an indication of multiple distinct components or tidal streams (see \cref{subsec:morpho-kin-class}).

$PA_{morph}$ is well constrained by the fits at small axis ratios, but becomes increasingly uncertain as the galaxy morphology becomes closer to spherical.
We note that typically the $PA_{morph}$ uncertainties are large in the regime of high axis ratios, but make no attempt to quote a formal uncertainty on this parameter.

\subsubsection{Half-light radii}\label{subsubsection:half-light_radii}
The third morphological parameter is the half-light radius, $R_{1/2}$, which provides an indication of the disk sizes and hence gives us a common fiducial distance from the centre of the galaxy at which to extract rotation velocities.
As discussed throughout the introduction and in \cref{subsubsec:param_extraction}, methods used to derive the intrinsic maximum circular velocity, $V_{C}$, vary widely between studies, and so to compare with previous work it is necessary to extract kinematic parameters consistently from the derived rotation curves.
Knowledge of $R_{1/2}$ gives a fixed point of extrapolation and a clear measure of how far beyond the data this extrapolation is made, both of which are important when making comparisons to previous IFU studies \citep[e.g.][]{ForsterSchreiber2009,Epinat2012,Wisnioski2015,Stott2016,Harrison2017,Swinbank2017}.
Throughout our dynamical modelling section we will assume a flux profile when constructing mock datacubes (see \cref{subsec:3d modelling}), which requires the $R_{1/2}$ value for each galaxy.
A final use of the derived $R_{1/2}$ values is to verify our beam smearing correction results through comparison with systematic correction factors presented in Johnson et al. (in prep), which are a function $R_{d} / R_{PSF}$ (see \cref{subsubsec:param_extraction}).

\cite{Bruce2012} presents a detailed error analysis for recovering the morphological parameters of $1 < z < 3$ galaxies in CANDELS, reporting that the magnitude of the error on $R_{1/2}$ is typically at the 10$\%$ level.
We adopt a nominal error of 10$\%$ for our recovered $R_{1/2}$ values and note that the errors on $V_{C}$ recovered from extracting at $R_{1/2}$ and the errors on the intrinsic velocity dispersion, $\sigma_{int}$, the derivation of which requires knowledge of $R_{1/2}$ (\cref{subsec:3d modelling}) are dominated by measurement errors and uncertainties connected with assumptions about the velocity dispersion distribution in high redshift galaxies (see \cref{app:kin_error_estimates}).

\subsubsection{Lack of HST Coverage}\label{subsubsection:lack_of_hst}
In SSA22, 4/21 of the spatially resolved galaxies do not have any HST coverage.
For these galaxies the inclination angle is set to the theoretical mean of 57.3$^{\circ}$ calculated for a population of galaxies with randomly drawn viewing angles (see e.g. the appendix in \cite{Law2009}).
This value is in good agreement with the KDS sample median of 57.4$^{\circ}$, see \cref{subsubsection:morph_param_summary}), and for the inclination uncertainty we use the standard deviation of the KDS inclinations, 12.4$^{\circ}$.
This corresponds to a factor $\sim1.3$ uncertainty in the derived velocities for these galaxies.
We fix $PA_{morph}$ equal to $PA_{kin}$ for the analysis described in \cref{subsec:3d modelling} and to measure $R_{1/2}$ we first fit the continuum light in the median stacked HK datacubes with a exponential profile, and subtract the PSF from the recovered $R_{1/2}$ value.
For these galaxies we also adopt a nominal $R_{1/2}$ error of 10$\%$ following the description above, and note that the derived $V_{C}$ values are not sensitive to the precise $R_{1/2}$ values, since we measure at $2R_{1/2}$ where the rotation curves have already flattened.
Owing to the small number of galaxies without HST imaging, we do not expect this to impact the bulk dynamical properties of the KDS sample.

\subsubsection{Morphological parameter summary}\label{subsubsection:morph_param_summary}

\begin{figure*}
    \centering \hspace{-2.0cm}
    \begin{subfigure}[h!]{0.50\textwidth}
        \centering
        \includegraphics[height=3.5in]{axis_ratio_distributions.png}
        %\caption{axis ratio distributions}
    \end{subfigure} \hspace{0.4cm}
    \begin{subfigure}[h!]{0.50\textwidth}
        \centering
        \includegraphics[height=3.6in]{size_distributions.png}
        %\caption{size distributions}
    \end{subfigure}
    \caption{We crossmatch publicly available spectroscopic redshifts for SFGs, as discussed in the text, with the morphological catalogue presented in \protect\cite{VanderWel2012} in two different redshift slices.
    This provides a set of reference morphological properties for typical SFGs at different evolutionary stages to which the derived KDS values can be compared.
    We find that the KDS galaxies are much more compact than those at low redshift, as traced by $R_{1/2}$ parameter, which could partially be explained by the F160W filter tracing a younger stellar population at $z\sim3.5$.
    The distribution of KDS $R_{1/2}$ values are consistent with those in the higher redshift reference sample.
    The $A_{r}$ distribution appears to be constant with time as traced by the reference samples, and the KDS values are in good agreement with a relatively uniform distribution spanning $0.3 < A_{r} < 0.9$.
    This shows that we have not been biased towards deriving low or high $A_{r}$ values.}
    \label{fig:morpho-distributions}
\end{figure*}

Having derived $R_{1/2}$, $A_{r}$ and $PA_{morph}$ for each of the KDS field galaxies \ref{fig:morpho-distributions} we plot the distributions of these properties, using KDS galaxies with HST coverage, along with distributions of morphological parameters determined in \cite{VanderWel2012} at low and high redshift.
The KDS galaxies have been identified in the \cite{VanderWel2012} catalogue and the morphological properties show excellent agreement, with median $\Delta A_{r} = 0.0002$, median $\Delta R_{1/2} = 0.0054^{\prime\prime}$ (corresponding to 0.04kpc at $z\sim3.5$) and the median \cite{VanderWel2012} $n = 1.2$ value for the KDS values close to our adopted n = 1 exponential disk fits.

At $3 < z < 4$ and $0.1 < z < 1$ we have made use of secure spectroscopic redshifts obtained for SFGs during the ESO public surveys zCOSMOS \citep{Lilly2007}, VUDS \citep{Tasca2016}, GOODS\_FORS2 \citep{Vanzella2005,Vanzella2006,Vanzella2008} and GOODS\_VIMOS \citep{Balestra2010} to cross-match with the morphological catalogue of \cite{VanderWel2012}.
This allows us to look at the bulk morphological properties of the SFG population in the two redshift slices in comparison with those determined for the KDS sample.
The derived KDS $R_{1/2}$ distribution at $z \sim 3.5$ is similar to the wider sample distribution at $3 < z  < 4$, i.e. compact when compared to galaxies at lower redshifts and the KDS $A_{r}$ are consistent with being generally uniform between $0.3 < A_{r} < 0.9$, with median value of 0.57, corresponding to i = 57.4$^{\circ}$ using the conversion given in equation \ref{eq:holmberg_i} with $q_{0}=0.2$, which is close to the statistical mean for a random galaxy sample.
This reassures us that we are not biased towards deriving either particularly low or particularly high inclination angles.

\subsection{Kinematic Measurements}\label{subsection:kinematic_measurements}

We proceed now to extract 2D measurements from the KDS datacubes by fitting the [O~{\sc III}]$\lambda$5007 emission line profiles within individual spaxels.
We then describe our dynamical modelling and the approach we have followed to extract $V_{C}$ and $\sigma_{int}$, using the models to derive beam smearing corrections for each galaxy.  

\subsubsection{Spaxel Fitting}\label{subsubsection:spaxel_fitting}
Having produced stacked and calibrated datacubes for each of the 77 galaxies in the full KDS sample, we aim to extract 2-dimensional maps of the flux, velocity and velocity dispersion.
These properties are extracted via modelling of the ionised gas emission line profiles within each spaxel, with a set of acceptance criteria imposed to determine whether the fit quality is high enough to allow the inferred properties to pass into the final 2D maps of the flux and kinematics.
We concentrate solely on the [O~{\sc III}]$\lambda$5007 emission line which always has S/N higher than both [O~{\sc III}]$\lambda$4959 and H$\beta$ and typically extends over a larger area of each IFU.
Each 0.1$^{\prime\prime}$ spaxel across a datacube is considered in turn, within which we locate the systemic wavelength of the redshifted [O~{\sc III}]$\lambda$5007 centre, $\lambda_{obs_{5007}}$, using the redshift value determined from fitting the [O~{\sc III}]$\lambda$5007 line in the integrated galaxy spectrum.

We search a narrow region around $\lambda_{obs_{5007}}$ for the peak flux, assumed to correspond to the ionised gas emission, and extract a 5$\AA$ region (corresponding to 20 spectral elements) centred on this peak.
This truncated spectrum is then used in the fitting procedure for each spaxel.
The width of the extraction region is large enough to encompass unphysically large velocity shifts, but not so large as to compromise the fitting by potentially encompassing regions of the spectrum plagued by poor subtraction of the sky emission lines.
A single gaussian model is fit to the extraction region, again using {\tt LMFIT} \citep{Newville2014}, returning the values of the best fitting flux, $F_{g}$, central wavelength, $\lambda_{g}$, and dispersion, $\sigma_{g}$.
These parameters, along with the estimate of the noise in the datacube, are used to assess whether the fitting of the [O~{\sc III}]$\lambda$5007 line is acceptable or not. 
The noise is estimated by masking all datacube spaxels containing object flux and computing the sum of the squares of the flux values in the spectral extraction region for all remaining spaxels inwards of 0.3$^{\prime\prime}$ from the cube boundary.
This boundary constraint is to mitigate edge effects, where the noise increases sharply due to fewer individual exposures constituting the final stack in these regions.
The final noise value is then taken as the standard deviation of the results from each unmasked spaxel, as shown in Equation~\ref{eq:mask_noise} where $F_{i}$ are the flux values across the extraction region and $N_{m}$ denotes noise estimated from masking the datacube.

\begin{equation}\label{eq:noise}
    N_{m} = STD\left(\sum_{i}F_{i}^{2}\right)
\end{equation}

The signal in each spaxel is taken as the sum of the flux values across the extraction region, as shown in Equation~\ref{eq:signal} where S denotes the signal.

\begin{equation}\label{eq:signal}
    S = \sum_{i}F_{i}
\end{equation}

The criteria for accepting the fit within a given spaxel are as follows:

\begin{enumerate}
\item The uncertainties on any of the parameters of the gaussian fit must not exceed 30\%
\item $\frac{S}{N} > 3$
\end{enumerate}

We impose a further criterion to help remove fits to spectral features unrelated to the galaxy (e.g. fits to skyline residuals), which is important for `cleaning up' accepted spaxels in the 2D maps which are clearly independent of the main body of the galaxy. For this test we subtract the galaxy continuum and impose that the $0.7 < F_{g}/{S} < 1.3$, where $F_{g}$ is the flux from the gaussian fit to the [O~{\sc III}]$\lambda$5007 line and S is the signal in the extraction region from equation \ref{eq:signal}.

If all three criteria are satisfied, the spaxel is accepted and the gaussian fit is repeated 1000 times with each flux value in the extraction region perturbed by the datacube flux error.
This error is computed as the median of the standard deviations in the extraction regions of the unmasked spaxels, as shown in equation \ref{eq:cube_error} with $F_{e}$ denoting the flux error.  

\begin{equation}\label{eq:cube_error}
   F_{e} = \big<STD(F _{i})\big>
\end{equation}

This Monte Carlo procedure gives the fit parameters and associated errors for the accepted spaxels.
The velocity is computed from the gaussian centre, $\lambda_{g}$, using Equation~\ref{eq:velocity_comp} and the velocity dispersion is computed from the gaussian width, $\sigma_{g}$, using Equation~\ref{eq:dispersion_comp}, where $v_{obs}$ and $\sigma_{obs}$ are the observed rotational velocity and velocity dispersion respectively, c is the speed of light and $\lambda_{5007} = 0.500824\mu m$ is the rest wavelength of [O~{\sc III}]$\lambda$5007 in vacuum.

\begin{equation}\label{eq:velocity_comp}
   v_{obs} = \frac{\lambda_{g} - (1 + z)\lambda_{5007}}{(1 + z)\lambda_{5007}} \times c
\end{equation}

\begin{equation}\label{eq:dispersion_comp}
   \sigma_{obs} = \frac{\sigma_{g}}{(1 + z)\lambda_{5007}} \times c
\end{equation}

If a spaxel does not meet the criteria we follow the procedure described in \cite{Stott2016} and bin with neighbouring spaxels to create a box of area $0.3\times0.3^{\prime\prime}$.
This is carried out by median stacking the extraction region spectra in each of the neighbouring spaxels.
The criteria are then re-examined and if satisfied the Monte-Carlo procedure is carried out. 
If the binned spaxels fail the criteria, one final iteration is carried out to bin to a box of area $0.5\times0.5^{\prime\prime}$, roughly equivalent to the size of the seeing disk.
If after the widest binning the criteria aren't satisfied, no value is assigned to that spaxel.

The process is automated and applied to each of the KDS galaxies that have an [O~{\sc III}]$\lambda$5007 detection.
Following the fitting procedure the galaxies are clasified as spatially resolved in [O~{\sc III}]$\lambda$5007 emission or not, with this defined as  the spatial extent of the accepted spaxels in the [O~{\sc III}]$\lambda$5007 flux map equivalent to or greater than the seeing disk.
For each pointing, the number of resolved galaxies, $N_{res}$, is added to table \ref{tab:pointings}.
29\% of the detected KDS galaxies are not spatially resolved, and we do not analyse these further, despite several showing velocity gradients within the seeing disk, due to the uncertainties associated with deriving kinematic properties and classifications from a single resolution element.
The physical properties of all spatially resolved galaxies in both GOODS-S and SSA22 are listed in table \ref{tab:phys-props}.

\subsubsection{3D modelling and beam smearing corrections}\label{subsec:3d_modelling}

As mentioned in the introduction, there is no universal standard for defining $V_{C}$ and $\sigma_{int}$, particularly when it is extrapolated from the data using a fitted model, of which there are several types.
Physically motivated models take into account the potential-well of the galaxy through knowledge of the mass distribution \citep[e.g.][]{Genzel2008,ForsterSchreiber2009,Gnerucci2011,Wisnioski2015}.
Phenomenological models assume a fixed function known to well describe observed galactic rotation curves which flatten at large radii \citep[e.g.][]{Epinat2010,Epinat2012,Swinbank2012,Stott2016,Harrison2017}.
Both of the above examples fit to 2D fields extracted from the datacubes using line-fitting software.
Recently, algorithms such as Galpak$^{3D}$ \citep{Bouche2015} and 3D-Barolo \citep{DiTeodoro2015} have emerged, fitting directly to the flux in the datacube in order to constrain the kinematic properties.

In this analysis we minimise the number of fitted parameters by making use of information from HST imaging and by assuming a fixed function to describe the rotation curves as described in the following subsection.    
We seek also to clearly distinguish between observed velocities and those inferred from modelling.
This is particularly important when dealing with galaxies at $z > 3$, since we often do not have enough S/N away from the centre of the galaxy to observe the flattening of the rotation curve, consequently requiring extrapolation to determine $V_{C}$.  
There are thus two clear reasons for constructing model velocity fields; the first is to extrapolate the observations smoothly to some fiducial radius from which $V_{C}$ can be extracted; the second is to estimate the effect of beam smearing on both the velocity and velocity dispersion fields so that we recover the set of intrinsic galaxy parameters which best describe the observations.
The following section describes our 3D modelling procedure, and the validation of our results through comparison with the techniques used to determine the dynamical properties presented in \cite{Harrison2017}.

\subsubsection{Constructing model datacubes}\label{subsubec:model_cube}
For each galaxy with spatially resolved [O~{\sc III}]$\lambda$5007 emission, we construct a model datacube with the same spatial dimensions as the observed datacube, and populate each spaxel with an [O~{\sc III}]$\lambda$5007 emission line that has central wavelength determined by the systemic velocity of the galaxy.
This central wavelength is then shifted using the velocity derived at each spaxel from the velocity field model described below.  
Following the procedure now of numerous authors \citep[e.g.][]{Epinat2010,Epinat2012,Swinbank2012,Stott2016,Mason2016} the velocity field of the gas is modelled as a thin disk with the discrete velocity points along $PA_{kin}$ determined by equation \ref{eq:arctangent}, which has been found to well match the rotation curves of galaxies in the local universe:

\begin{equation}\label{eq:arctangent}
   v_{r} = \frac{2}{\pi}v_{asym}arctan\left(\frac{r}{r_{t}}\right)
\end{equation}

\noindent
with `r' measured as the distance from the centre of rotation, (xcen, ycen).
This model is then projected onto the cube using the inclination angle determined in \cref{subsubsec:galfitting} and sampled with 4 times higher spatial resolution than the KMOS raw data before binning back to the previous resolution.
The increase in model spatial resolution is necessary to capture the steep velocity gradient across the central region of the rotating galaxy, implied by the use of the arctangent function.
If the model is constructed with equivalent spatial resolution to the data, the rotation curve has a flatter central profile which mimics the effect of beam smearing.
The velocity decreases with a factor of cos($\phi$) from $PA_{kin}$, where $\phi$ is the angle measured clockwise from $PA_{kin}$, so that $PA_{kin}$ points to the positive side of the velocity field.
In principle then the model has 5 free parameters; $\{(xcen, ycen), PA_{kin}, rt, v_{asym}$, rather than the 6 free parameters in some previous works \citep[e.g.][]{Stott2016}, since we have determined the inclination has been determined from the HST axis ratio.
The intrinsic flux profile of these simulated [O~{\sc III}]$\lambda$5007 lines in the spatial direction is determined using the galfit model derived in section \cref{subsubsec:galfitting} (which assumes that the [O~{\sc III}]$\lambda$5007 emission profile follows the stellar profile), also constructed at 4 times higher spatial resolution than the KMOS cube and rebinned for the same reasons, and the intrinsic velocity dispersion of the line is set to follow a uniform distribution with a default width of 50kms$^{-1}$.
This default line width is selected after examining Figure 10 of \cite{Wisnioski2015}, which shows predictions for the cosmic evolution of velocity dispersion.

This simulated intrinsic [O~{\sc III}]$\lambda$5007 cube is then convolved slice by slice using fast Fourier transform libraries, with the atmospheric seeing profile determined from fitting the collapsed, stacked standard star cube in each pointing with an elliptical gaussian (the recovered seeing values are reported in table \ref{tab:pointings}).
The velocity and velocity dispersion values are then re-measured in each spaxel to produced `beam-smeared' 2D maps of the kinematic parameters.

We fit the beam-smeared velocity field to the observed velocity field, using MCMC sampling with the python package `emcee' \citep{Foreman-Mackey2013} to vary the intrinsic model parameters, and seek the combination of parameters which maximises the log-Likelihood function given by equation \ref{eq:likelihood}, which fully accounts for the errors on the observed velocity field.

\begin{equation}\label{eq:likelihood}
   ln\Lagr = \frac{-0.5\sum_{i=1}^{N}(d_{i} - M_{i})^{2}}{\delta_{v_{i}}^{2}} - ln\left(\frac{1}{\delta_{v_{i}}^{2}}\right)
\end{equation}

\noindent
where $\delta_{v_{i}}$ are the observed velocity errors, $d_{i}$ are the datapoints and $M_{i}$ are the convolved model velocity values.
In practice we fix the (xcen,ycen) parameters of the galaxy to the location of the stellar continuum, a proxy for the centre of the gravitational potential well, recovered from the collapsed KMOS cube, leaving only 3 free parameters to vary in the MCMC sampling.
This assumes that the gas and the stars in the galaxy share a common centre of mass, an assumption which is verified for the vast majority of SFGs in the local and intermediate redshift universe.
Due to the faintness of the continuum, there are several galaxies for which we cannot reliably estimate the rotation centre using this method, and instead set (xcen,ycen) to coincide with the peak [O~{\sc III}]$\lambda$5007 flux location.
Using the continuum to fix the rotation centre is generally more successful for the SSA22 galaxies, since the use of the HK filter effectively doubles the S/N of this component by collapsing over two wavebands.
This differs from the method of fixing the rotation centre at the midpoint between the velocity extrema, described in \cite{Wisnioski2015} and \cite{Rodrigues2016}.
In general the S/N of our sample is low at the galaxy outskirts and for many of the isolated rotating sample we don't observe the flattening of the velocity field.
This suggests that the velocity midpoint may be biased towards the side of the galaxy which shows the higher velocity value and so we instead use the physically motivated continuum centre where possible, which also matches the approach described in \cite{Harrison2017}.

The MCMC sampling provides a distribution of parameter values around those which give the maximum likelihood.
We have confirmed also by examining the individual parameter chains that the MCMC run has been properly `burned-in' and that the parameter estimates have converged around the 50th percentile values.
The 16th and 84th percentiles of these distributions will be used to assess the model uncertainties as described in \cref{appsubsec:model_errors}.

Following the construction of a convolved cube and the measurement of the beam-smeared maps we can extract beam smearing corrected values for $V_{C}$ and $\sigma_{int}$, as described throughout the following subsections.
There have been several different methods used to extract $V_{C}$ and $\sigma_{int}$ in previous surveys at different redshift.
We provide an overview of these surveys in section \cref{subsubsec:comparison_samples} along with the methods used to extract kinematic parameters and the derivation of average values of the parameters used for comparison to the KDS throughout results section \cref{subsec:redshift_evolution}).

\subsubsection{KDS beam smearing corrected rotation velocities}\label{subsubsec:beam_smearing_corrected_velocities}
We extract $V_{smeared}$ and $V_{C}$ by reading off from the beam-smeared and intrinsic best fitting 2D model grids respectively at $2R_{1/2}$.
The $R_{1/2}$ value used to extract velocities from the KMOS data is found by convolving the intrinsic $R_{1/2}$ value with the PSF of the observations.
The $V_{C}$ value extracted at this radius is a commonly used measure of the `peak' of the rotation curve \citep[e.g.][]{Miller2011,Pelliccia2016,Stott2016,Harrison2017} and is similar also to the $v_{80}$ measurement of the velocity at the radius enclosing 80$\%$ of the total light \citep{Tiley2016a}, and extracting the velocity at $\geqslant 2R_{1/2}$ can be crucial for measuring the majority of the total angular momentum \citep[e.g.]{Obreschkow2015,Harrison2017}.
The beam smearing correction factor is then given by the ratio $V_{C}/V_{smeared}$, with mean and median values of 1.29 and 1.21 for the KDS galaxies classified as morphologically isolated (see \cref{subsec:morpho-kin-class}).
The distributions of $V_{obs}$ and $V_{C}$ are discussed in more detail in section \cref{subsubsec:observed_and_intrinsic}.
The $V_{C}$ errors are discussed in \cref{appsubsubsec:model_errors}.

In most of the KDS field sample galaxies, the $2R_{1/2}$ fiducial radius used to extract $V_{C}$ from the model is greater than the last observed radius, with 2 galaxies requiring extrapolation > 0.4$^{\prime\prime}$ with a mean extrapolation of 0.17$^{\prime\prime}$.
This extrapolation is a requirement imposed by the low S/N at the outskirts of the KDS galaxies, limiting the extent to which the ionised gas disk can be observed, with the majority of the observed velocity curves not reaching the point of flattening. 
The extent of the extrapolation for each galaxy is made clear throughout the kinematics plots in \cref{app:kinematics_plots}, with the $V_{C}$ extraction radius marked on each velocity thumbnail.
We present the maximum observed velocity values, $V_{obs}$, and the intrinsic, inclination and beam smearing corrected model velocity values, $V_{C}$, in table \ref{tab:kin_props}.

\subsubsection{KDS beam smearing corrected intrinsic dispersions}\label{subsubsec:beam_smearing_corrected_dispersions}
Convolution of the model velocity field with the seeing profile produces a peak in the beam-smeared velocity dispersion map at the centre of rotation.
The model therefore provides a beam smearing correction for the velocity dispersion in every spaxel, dependent mainly on the magnitude of the velocity gradient, which is equal to the difference between the assumed intrinsic velocity dispersion and the beam-smeared dispersion i.e. $\sigma_{bs} = \sigma_{model} - \sigma_{assumed}$.
This is computed for each spaxel to provide the 2D $\sigma_{bs}$ correction map.
The impact of the value of $\sigma_{assumed}$ on the magnitude of $\sigma_{bs}$ correction is discussed further in section \cref{appsubsubsec:model_errors}.
We note also that not all of the observed $\sigma_{int}$ profiles peak in the centre, and since we do not attempt to fit the velocity dispersions, this is a feature the model cannot reproduce. 

To measure the $\sigma_{int}$ value for each galaxy in our sample we first use equation \ref{eq:owen_sigma} to derive an intrinsic sigma map, where $\sigma_{inst}$ is a measure of the KMOS instrumental resolution value averaged across gaussian fits to several skylines (equal to 31.1kms$^{-1}$ in the K-band and 55.4kms$^{-1}$ in the HK-band).
This follows the approach of \cite{Stott2016} which shows that linear subtraction of the beam smearing correction best recovers the $\sigma_{int}$ values. 

\begin{equation}\label{eq:owen_sigma}
   \sigma_{int} = \sqrt{\left(\sigma_{obs} - \sigma_{bs} \right)^{2} - \sigma_{inst}^{2}}
\end{equation}

The median value of this map is taken as the $\sigma_{int}$ value for that galaxy and the median value of the observed map is the $\sigma_{obs}$ value, with both of these presented in table \ref{tab:kin_props}.

The beam smearing correction factor is then given by the ratio of these two medians, i.e. $\sigma_{obs}/\sigma_{int}$, with mean and median values of 0.77 and 0.79 respectively for the morphologically isolated KDS galaxies.


To test the robustness of our beam smearing correction approach we have also computed $V_{C}$ and $\sigma_{int}$ values using the independently derived correction factors, which are a function of the velocity gradient and the ratio $R_{d}/R_{psf}$, detailed in Johnson et al. (in prep), used to evaluate the final kinematic parameters in \cite{Harrison2017}.
These correction factors are applied to $V_{smeared}$ to reach $V_{C}$ and to $\sigma_{obs}$ to reach $\sigma_{int}$.
The $V_{C}$ values agree to within $\sim10\%$, and the $\sigma_{int}$ values to within $\sim5\%$ which is within the errors on these quantities. 
This confirms that the choice of modelling and kinematic parameter extraction described above is consistent with the $z\sim0.9$ KROSS results.


\begin{table*}
\centering
\begin{threeparttable}
\caption{Dynamical Properties}
\label{tab:kin_props}
\begin{tabular}{cccccccc}

 \hline
ID & $V_{obs}$(kms$^{-1}$) & $V_{C}$(kms$^{-1}$) & $\sigma_{obs}$(kms$^{-1}$) & $\sigma_{int}$(kms$^{-1}$) & $V_{C}/\sigma_{int}$ & $R_{1/2}$(kpc) & PA$_{kin}^{\circ}$ \\
 \hline
 GOODS-S & & & & & & & \\
 \hline
bs009818        & 65.0                    & 84.0                     & 95.0                  & 79.0                   & 1.06       & 1.25        & 72.0 \\
bs014828        & 32.0                    & 46.0                     & 97.0                  & 88.0                   & 0.53       & 1.65      & 44.0   \\
lbg\_20         & 26.0                    & 56.0                     & 55.0                  & 40.0                   & 1.39      & 1.30        & 41.0  \\
lbg\_105        & 13.0                    & 32.0                     & 68.0                  & 59.0                   & 0.54       & 1.75       & 170.0 \\
bs006541        & 36.0                    & 65.0                     & 96.0                  & 83.0                   & 0.78        & 1.85       & 23.0 \\
b012141\_012208 & 34.0                    & 93.0                     & 94.0                  & 63.0                   & 1.47       & 1.60       & 122.0 \\
b15573          & 62.0                    & 81.0                     & 111.0                 & 84.0                   & 0.97        & 0.50       & 118.0 \\
bs008543        & 77.0                    & 111.0                    & 98.0                  & 71.0                   & 1.56       & 1.60       & 114.0 \\
bs006516        & 56.0                    & 61.0                     & 64.0                  & 45.0                   & 1.35      & 1.95       & 144.0  \\
bs016759        & 71.0                    & 110.0                    & 82.0                  & 57.0                   & 1.93        & 0.90       & 75.0 \\
lbg\_111        & 53.0                    & 58.0                     & 90.0                  & 81.0                   & 0.71      & 0.65        & 168.0 \\
lbg\_38         & 31.0                    & 97.0                     & 76.0                  & 53.0                   & 1.82        & 0.95       & 47.0 \\
lbg\_94         & 18.0                    & 34.0                     & 65.0                  & 55.0                   & 0.62       & 1.20       & 41.0 \\
lbg\_109        & 32.0                    & 53.0                     & 101.0                 & 91.0                   & 0.58      & 2.00        & 157.0 \\
lbg\_24         & 30.0                    & 42.0                     & 64.0                  & 51.0                   & 0.82      & 1.30       & 64.0  \\
lbg\_25         & 14.0                    & 33.0                     & 53.0                  & 41.0                   & 0.8       & 1.20        & 87.0  \\
lbg\_32         & 79.0                    & 121.0                    & 104.0                 & 76.0                   & 1.58       & 1.90       & 11.0  \\
lbg\_91         & 58.0                    & 91.0                     & 98.0                  & 71.0                   & 1.28       & 0.90        & 40.0 \\
$^{*}$lbg\_113        & 0.0                    & 0.0                      & 116.0                 & 112.0                  & 0.0         & 0.90       & 28.0 \\
 \hline
 \hline
 SSA22 & & & & & & & \\
 \hline
 \hline
s\_sa22a-d3   & 57.0                    & 109.0                    & 127.0                 & 93.0                   & 1.16        & 1.80       & 30.0 \\
n3\_009       & 32.0                    & 47.0                     & 77.0                  & 48.0                   & 1.0       & 1.07       & 20.0 \\
lab18         & 44.0                    & 41.0                     & 87.0                  & 62.0                   & 0.67     & 0.45       & 54.0  \\
s\_sa22b-c20  & 79.0                    & 139.0                    & 112.0                 & 48.0                   & 2.87       & 1.60       & 37.0 \\
$^{*}$n\_c3         & -                    & -                      & 96.0                  & 78.0                   & -        & 0.55       & -  \\
$^{*}$s\_sa22b-d9   & -                    & -                      & 99.0                  & 82.0                   & -      & 0.50       & 163.0  \\
$^{*}$lab7\_top     & -                    & -                      & 83.0                  & 62.0                   & -      & 1.50       & 85.0  \\
$^{**}$lab25         & 42.0                    & 48.0                     & 80.0                  & 49.0                   & 0.98       & -       & 72.0 \\
$^{**}$s\_sa22b-d5   & 31.0                    & 57.0                     & 67.0                  & 23.0                   & 2.45      & -       & 60.0 \\
$^{**}$s\_sa22b-md25 & 50.0                    & 57.0                     & 102.0                 & 75.0                   & 0.76       & -       & 9.0  \\
$^{**}$n3\_006       & 41.0                    & 81.0                     & 144.0                 & 125.0                  & 0.65      & -       & 156.0 

\end{tabular}
\begin{tablenotes}
      \small
      \item $^{*}$ No clearly defined kinematic axis; $V_{obs}$ and $V_{C}$ not defined.
      \item $^{**}$ No HST coverage; see table \ref{tab:phys-props}
    \end{tablenotes}
  \end{threeparttable}
  \end{table*}


\subsection{Comparison samples}\label{subsubsec:comparison_samples}
Throughout section \cref{subsec:redshift_evolution} we compare with the results of surveys tracing dynamics with ionised gas emission across a wide redshift baseline, to determine the evolving physical state of SFGs as the age of the universe increases.
The galaxy selection criteria in these surveys, with the exception of \cite{Green2014} as discussed below, consistently pick out main sequence SFGs, and trace the dynamical properties of these by observing either the [O~{\sc III}]$\lambda$5007 or $H\alpha$ ionised gas emission line.
Given the range of modelling and kinematic parameter extraction methods it is important to verify the extent to which the results from these surveys can be directly compared and treated as forming an evolutionary sequence, which we do in the following subsections by considering each survey in turn.
We make use of tabulated data from the surveys deemed directly comparable, where available, to compute sample averages using the following method. 
The fractional error weighted mean of $V_{C}/\sigma_{int}$, $\sigma_{int}$ and $V_{C}$ in each of the samples is computed (i.e. we do not want the derived values with extremely large errors to dominate the averages).
The errors on these mean values are computed in a statistical sense, by generating bootstrapped samples, with replacement, with size equivalent to the original survey sample size and with values perturbed by a random number drawn from a gaussian distribution with width given by the error on the original point.
The same process is applied to compute the errors on the sample medians, and we report the 16th and 84th percentiles of the distributions of each of the above quantities as an indicator of the distribution width, and hence galaxy diversity, at each redshift slice.
In section \cref{subsec:redshift_evolution} we will make statements about dynamical evolution by connecting the dots of these different surveys, assuming that on average they are tracing a population of SFGs which evolve across cosmic time. 


\subsubsection{GHASP}\label{subsubsec:GHASP}

\subsubsection{DYNAMO}\label{subsubsec:DYNAMO}

\subsubsection{KROSS}\label{subsubsec:KROSS}

\subsubsection{KMOS$^{3D}$}\label{subsubsec:kmos_3d}

\subsubsection{MASSIV}\label{subsubsec:MASSIV}

\subsubsection{SINS$^{3D}$}\label{subsubsec:SINS}

\subsubsection{LAW\_09}\label{subsubsec:law_09}

\subsubsection{AMAZE}\label{subsubsec:AMAZE}








The SINS sample uses a Kinemetry approach described for \citep{Shapiro2008,ForsterSchreiber2009,Cresci2009} which quantifies asymmetries in both the velocity and velocity dispersion maps to empirically differentiate between rotating and non-rotating systems.
In \cite{Gnerucci2011} an inclined plane is fit to the velocity map and the $\chi^{2}$ of this fit is evaluated in order to test for the presence of smooth velocity gradients, with the authors also commenting that the S/N of the data is not sufficient to accurately determine the required kinemetry parameters.
In \cite{Epinat2012}, a detailed description of a morpho-kinematic classification in outlined throughout their section 4, based on the proximity of counterparts in the imaging, $PA_{kin}$ and $PA_{morph}$ misalignment (i.e.$\Delta PA$) and the degree of rotational support as quantified with $V_{C}/\sigma_{int}$.
In \cite{Wisnioski2015} a set of 5 criteria are described in their section 4 which assess the smoothness of the velocity gradient, the degree of rotational support through the ratio of the velocity to velocity dispersion, $V_{rot}/\sigma$, $\Delta PA$ and the position of the kinematic centre in relation to both the peak in the velocity dispersion map and the centroid of the continuum centre (also see \cite{Rodrigues2016} for a re-analysis of KMOS$^{3D}$ data at z$\sim 1$ using the same criteria).

The crucial information to extract for the model grids described above for each galaxy, both for this study and for comparison to previous studies, is a measure of the maximum intrinsic circular velocity, $V_{C}$, and a single measurement of the intrinsic velocity dispersion, $\sigma_{int}$, used as a diagnostic for the local random motions of the gas free from large scale velocity gradients \citep[e.g.]{ForsterSchreiber2006,Genzel2008,Gnerucci2011}.
Previous studies have measured these quantities in several different ways, and it is important to keep this in mind when trying to establish the dynamical evolution of star forming galaxies between different slices of cosmic time.
Here we briefly mention the measurement techniques of the major surveys we aim to make comparisons to at high redshift, as an attempt to motivate our own method of kinematic parameter extraction.
The SINS sample \citep{ForsterSchreiber2009} is refined to a high-z robust rotating disk sample \citep{Cresci2009} using a kinemetry approach \citep{Krajnovic2006,Shapiro2008} which searches the moment maps extracted from the SINFONI datacubes for asymmetries which quantify the degree of rotational support against gravity in each galaxy.
\cite{Cresci2009} model the robust rotators in three dimensions using the IDL code DYSMAL, which derives rotation curves given an input radial mass distribution.
In this approach the $V_{C}$ value comes from the best fitting model parameter, and $\sigma_{int}$ is calculated using equation \ref{eq:sins_sigma}, with the best fitting $\sigma_{01}$ used (reflecting thin and thick disks; see their text for more detail) and with $\sigma_{02}$, an additional component of isotropic velocity dispersion throughout the disk, left as a free parameter in the fitting.

\begin{equation}\label{eq:sins_sigma}
\begin{split}
\sigma_{01} = \sqrt{\frac{v^{2}(R)h_{z}}{R}} \quad OR \quad \sigma_{01} = \frac{v(R)h_{z}}{R} \\
& \sigma_{int} = \sqrt{\sigma_{01}^{2} + \sigma_{02}^{2}}
\end{split}
\end{equation}

In \cite{Law2009}, a study using OSIRIS, the `velocity shear' is computed as $v_{shear} = \frac{1}{2}(v_{max} - v_{min})$ and $\sigma_{int}$ is the flux-weighted mean of $\sigma_{obs}$.

In their survey at z $\sim 3$, closest in redshift to the current work, \cite{Gnerucci2011} et al. derive rotation curves and intrinsic velocity dispersions from a modelled exponential mass distribution, varying the parameters of the model to best fit the observed velocity field.
The extracted $V_{C}$ value is then the large radius limit of the rotation curve and the $\sigma_{int}$ is the maximum of the difference in quadrature between the $\sigma_{obs}$ map and the $\sigma_{model}$ map, which also takes into account instrumental resolution and beam smearing (see their Eq. 8).

In the MASSIV sample, \citep{Epinat2012}, spanning $0.9 < z < 1.6$, the authors use a 2D arctangent fit to the data and extract $V_{C}$ from the model at $R_{last}$, this being the radius where the S/N of the data drops below 3.
The $\sigma_{int}$ value is derived from the $\sigma_{obs}$ map, by subtracting in quadrature the $\sigma_{model}$ value as per \cite{Epinat2010}, which is the contribution from beam smeaing assuming the best-fitting model parameters. 

In KMOS$^{3D}$, covering two redshift regions at $z \sim 1.0$ and $z \sim 2.3$, \cite{Wisnioski2015} compute $V_{C} = \frac{1}{2sin(i)}(v_{obs_{max}} - v_{obs_{min}})$ and $\sigma_{int}$ is computed from the $\sigma_{obs}$ map far away from the kinematic centre, where the effects of beam smearing should be negligible.
In the KROSS sample at $z \sim 1$, \cite{Stott2016} also model the 2D velocity field using an arctangent function, and after finding the best fitting parameters using a genetic algorithm, extract $V_{C}$ at a fiducial radius, which they choose to be $2.2R_{d} \sim 1.3R_{1/2}$, where $R_{d}$ is the galaxy disk radius in the case of an exponential light profile.
The $\sigma_{int}$ value is computed in two different ways; the first facilitates comparison with KMOS$^{3D}$ results by extracting the instrumental resolution corrected $\sigma_{obs}$ far away from the kinematic centre; the second uses Equation \ref{eq:stott_sigma} to first linearly remove the effects of the local beam smeared velocity gradients, and then correct for instrumental resolution in quadrature (an approach which they verify in their Appendix A).

\begin{equation}\label{eq:stott_sigma}
   \sigma_{int} = \sqrt{\left(\sigma_{obs} - \frac{\Delta V}{\Delta R}\right)^{2} - \sigma_{inst}^{2}}
\end{equation}

The derived median $\sigma_{int}$ from these two approaches varies considerably, as highlighted in Figure 8 of their paper.

In the most recent KROSS paper, \citep{Harrison2017}, exponential disk models, \cite{Freeman1970}, are fit to the data extracted along $PA_{kin}$ in 1D to facilitate extraction of the observed velocity at $2R_{1/2}$ (where $2R_{1/2} \sim 3.4R_{d}$).
The $\sigma_{int_{obs}}$ is also extracted at $2R_{1/2}$ when the data extend to this radius, and when they do not $\sigma_{int_{obs}}$ is taken to be the median of the $\sigma_{obs}$ map.
Using a suite of mock KMOS data with exponential intrinsic flux distributions and $PA_{morph}$ coincident with $PA_{kin}$s, convolved with various atmospheric seeing profiles, beam smearing correction factors are derived as a function of both the velocity gradient and the ratio $R_{d}/R_{PSF}$ Johnson et al. in prep.
This correction factors are applied to compute $V_{C}$, with a median correction factor of 1.2 at $3.4R_{1/2}$.
The same treatment is applied to the outer $\sigma_{int_{obs}}$ values, with a median correction factor of 0.97, and when unavailable a seperate correction factor is applied to the median of the $\sigma_{obs}$ map with a median value of 0.8 (see also \citep{Burkert2016a} for similar simulations of mock data and derivations of beam smearing correction factors).

Clearly there are many different approaches for computing the same intrinsic kinematic parameters which we hope to compare across cosmic time intervals, but despite the diversity over the past decade the approaches recently appear to be converging on `modelling out' the effects of beam smearing using similar approaches.
This trend towards consistency is encouraging, and as described in \cref{subsec:3d modelling} we have also shaped our approach towards extracting intrinsic kinematic parameters after correcting for beam smearing effects quantified in the modelling.
Our small sample size has allowed us to compute beam smeared models, and the corresponding intrinsic models, for each galaxy.


\subsection{Morpho-Kinematic classification}\label{subsec:morpho-kin-class}
% chancing my luck a bit here with this paragraph
Classification into different kinematic categories at $z \sim 3.5$ is necessary for quantifying the fraction of KDS galaxies which are isolated `rotation dominated' with $V_{C}/\sigma_{int} > 1$, isolated `dispersion dominated' with $V_{C}/\sigma_{int} < 1$ and the fraction taking part in interactions with other galaxies.
In the local universe, main sequence galaxies are completely dominated by rotational motions, with the ratio of $V_{C}/\sigma_{int} \sim 10$ or higher \citep{Epinat2008,Epinat2008a}, with this ratio observed to decline over time as random motions increase \citep[e.g.][]{ForsterSchreiber2009,Law2009,Epinat2012,Wisnioski2015,Stott2016}.
By studying the Rotation Dominated Fraction (RDF) of KDS galaxies, and the mean value of the ratio $V_{C}/\sigma_{int}$ we hope to gain an understanding of balance of gravitational support provided by rotational and random motions.
$V_{C}$ and $\sigma_{int}$ are both outputs of the dynamical model, so we describe here a pre-modelling morpho-kinematic classification for all KDS galaxies which are spatially resolved in the [O~{\sc III}]$\lambda$5007 emission line, to inform us which galaxies it is appropriate to attempt dynamical modelling. 

In this work we are in a similar S/N and redshift regime as \cite{Gnerucci2011} and so we aim to keep the classification criteria simple.
This involves a joint visual inspection of the velocity fields and associated HST imaging for all spatially resolved KDS galaxies.
One benefit of studying the high resolution imaging in tandem is to aid the interpretation of peculiarities in the velocity and velocity dispersion fields.
These peculiarities are usually unexpected discontinuities in the rotation fields, or extreme broadening of the dispersion field above that expected from beam smearing or in regions spatially offset from the centre of the galaxy.
A second benefit is to uncover `rotation doppelg{\"a}ngers', which mimic disk rotation from a purely kinematic perspective, but are clearly two or more potentially interacting components at slightly different redshift as revealed by higher resolution HST photometry.
It is difficult to distinguish in some cases between two HST components rotating together in a single disk (i.e. clumpy disk galaxies \citep[e.g.][]{Elmegreen2004,Bournaud2007}) with a receding and approaching side and two [O~{\sc III}]$\lambda$5007 emitting blobs which are offset along the line of sight.
Perhaps the most telling diagnostic is a double peak in the object spectrum at the kinematic centre, which allows for the classification of IFU data with 2 HST flux peaks as mergers.

In 4/23 of the resolved GOODS-S galaxies (one count per merger, 2 mergers in each of the GOODS-S pointings) we clearly identify signs of multiple components or merging in the HST high-resolution imaging.
In 9/20 of the resolved SSA22 galaxies we identify clear mergers from the HST imaging, noting that 8 of these are in SSA22-a (80$\%$ of the spatially resolved sample in that pointing, although potentially higher as one of the galaxies is without HST coverage), which is the pointing close to the centre of the $z \sim 3.1$ proto-cluster.
We note that the 1/11 mergers reported in SSA22-b is a lower limit, since there are 3 galaxies which do not have HST coverage in this pointing, although the pointing is spatially offset to the south of the protocluster centre in SSA22-a, and consequently is less dense.
This observation in SSA22-a is perhaps unsurprising in such a high-density environment and these interactions will inevitably have major consequences for the consumption and stripping of gas and the morphological evolution of these galaxies.
The observation of frequent mergers which cause major kinematic disturbances is also in support of the evidence that galaxies living at the peak of the cosmic density field at lower redshift tend to be redder, less active and morphologically early-type, having presumably rapidly exhausted their gas supply throughout these violent interactions at high redshifts \citep[e.g.][]{Steidel1998,White2007,Kodama2007,Zheng2009}.  
We reject these merger candidates from the following dynamical analysis presented in \cref{sec:Kinematic_Properties}, keeping the KDS galaxies with HST coverage with the caveat that we do not have a clear morphological classification.
In Appendix \cref{app:mergers} we present the HST imaging, kinematic maps and 1-dimensional kinematic extractions of the merger candidates for the KDS galaxies.
Many of these are rotation doppelg{\"a}ngers and can be fit with the arctangent model (see \cref{subsec:3d modelling}) with a low $\chi ^{2}$ value, highlighting again the importance of the HST photometry for determining true rotation.
The best fitting smeared and intrinsic models for the modelling sample in both are plotted in \cref{app:kinematics_plots}, where the galaxies have been divided into dispersion and rotation dominated categories as determined by the ratio $V_{C}/\sigma_{int}$, described in more detail throughout \cref{subsubsec:rotation_and_mergers}.

The galaxies classified as mergers are also plotted in \cref{app:mergers}, as well as the dispersion dominated galaxies which have not been d with the full 3D modelling in \cref{app:dispersion_dominated,app:dispersion_dominated}.

For the spatially resolved galaxies which are not merger candidates, we extract in synthetic apertures along an axis joining the velocity extrema to search for the presence of a smooth velocity gradient indicative of rotation within a disk.
In 1/23 and 3/20 resolved galaxies in GOODS-S and SSA22 respectively we find no such gradient, and do not attempt to fully model the kinematics of the galaxies to determine $V_{C}$ and $\sigma_{int}$.
We assume that there is very little rotational support in these galaxies and take the $\sigma_{int}$ value as the median of the intrumental resolution corrected velocity dispersion map.
To sumamrise, galaxies are passed to the full dynamical modelling of \cref{sec:Kinematic_Properties} if:
\begin{itemize}
    \item They are spatially resolved in the [O~{\sc III}]$\lambda5007$ emission line
    \item They have a clear, monotonic velocity gradient along a single kinematic axis
    \item They are isolated as revealed by high-resolution HST imaging
\end{itemize}

The galaxies passing these criteria shall be referred to as the modelling sample throughout the remainder of the paper, and we make no further distinction between galaxies in the SSA22 and GOODS-S fields.
It is only these galaxies for which we can measure $V_{C}$, and to which the `$V_{C}$ measured' legend titles refer throughout \ref{sec:results}.

\subsubsection{Kinematic sample summary}\label{subsubsec:kin_sample_summary}



\section{RESULTS \& DISCUSSION}\label{sec:results}

\begin{figure*}
    \centering
    \begin{subfigure}[h!]{0.32\textwidth}
        \centering
        \includegraphics[height=2.25in]{vobs_vs_vmax.png}
        %\caption{V$_{obs}$ vs. intrinsic $V_{C}$}
    \end{subfigure}%
    \begin{subfigure}[h!]{0.32\textwidth}
        \centering
        \includegraphics[height=2.25in]{sigobs_vs_sigint.png}
        %\caption{$\sigma_{obs}$ vs. intrinsic $\sigma_{int}$}
    \end{subfigure}
    \begin{subfigure}[h!]{0.32\textwidth}
        \centering
        \includegraphics[height=2.25in]{sigobs_vs_vmax.png}
        %\caption{$\sigma_{obs}$ vs. $V_{C}$}
    \end{subfigure}
    \caption{Correlations between observed KDS properties (y-axis) and the intrinsic properties (x-axis) recovered following the beam smearing analysis.
    Plots (a) and (b) highlight the strong positive correlation between the observed and intrinsic velocities and velocity dispersions, verifying that the beam smearing corrections are not generally over-extrapolating beyond the values recovered from the data themselves.
    Plot (c) shows the positive correlation between velocity dispersion and velocity, demonstrating that the broadening of $\sigma$ in the beam smearing is largely a function of the strength of the velocity gradient in the galaxy.}
    \label{fig:observed_and_intrinsic}
\end{figure*}

\subsection{Derived Kinematic Properties}
Equipped with $PA_{Kin}$, $V_{C}$ and $\sigma_{int}$ for each of the resolved and non-interacting galaxies in the KDS, we first interpret these at $z \sim 3.5$, comparing with the \cite{Gnerucci2011} (G11) results at similar redshift throughout \cref{subsubsec:kin_misalignment}-\cref{subsubsec:dynamical_masses} and then in the context of galaxy evolution by comparing with samples across a wide redshift baseline in \cref{subsubsec:sigma_evolution} - \cref{subsubsec:v_evolution}.
\subsubsection{Kinematic and Photometric Misalignment}\label{subsubsec:kin_misalignment}
The dynamical model parameter $PA_{Kin}$ can be compared with $PA_{morph}$ computed by GALFIT as described in \cref{subsubsec:galfitting}, to indicate the degree of misalignment between the gaseous and stellar components of the galaxies.
In figure \ref{fig:delta_pa} (a) the absolute difference between the position angles, $\Delta PA = |PA_{Kin}-PA_{morph}|$, is plotted as a function of the axis ratio, $\frac{b}{a}$, also determined by GALFIT from the HST photometry.
The mean and median $\Delta PA$ values are 46$^{\circ}$ and 26$^{\circ}$ respectively, and $\Delta PA$ tends to increase as the axis ratio increases.
Similar analysis is described in \citep[e.g.][]{Epinat2012,Wisnioski2015,Harrison2017} wherein rotating galaxies generally have excellent agreement between the position angles, with $\Delta PA$ increasing as a function of axis ratio due to $PA_{morph}$ becoming more difficult to measure as the galaxy becomes more `face-on'.
This is due to the connection between galaxy inclination and axis ratio, with the lowest inclination galaxies showing the most morphological elongation, and the direction of this elongation on the sky is indicative of the galactic plane of rotation.

\begin{figure*}
    \centering \hspace{-1.3cm}
    \begin{subfigure}[h!]{0.5\textwidth}
        \centering
        \includegraphics[height=3.5in]{delta_pa_vs_axis_ratio.png}
        %\caption{$\Delta PA$ vs. axis ratio}
    \end{subfigure} \hspace{0.4cm}
    \begin{subfigure}[h!]{0.5\textwidth}
        \centering
        \includegraphics[height=3.5in]{delta_pa_vs_sigma.png}
        %\caption{$\Delta PA$ vs. $\sigma_{obs}$}
    \end{subfigure}
    \caption{The absolute difference between $PA_{morph}$ and $PA_{kin}$, $\Delta PA$, is plotted in panel (a) against the GALFIT determined galactic minor to major axis ratio for all KDS galaxies with both HST coverage and dynamical modelling.
The median difference computed in bins of 0.2 x-axis width and the standard deviation in each of these bins are also plotted as the black line and light blue shaded region respectively.
The misalignment clearly increases as a function of axis ratio, but is also generally larger than reported in previous IFU studies at lower redshift, as described throughout the text.
In panel (b) we explore the possibility that the photometric and kinematic misalignment may be a function of the observed velocity dispersion.
There appears to be some suggestion that with increasing $\sigma_{obs}$, larger $\Delta PA$ are observed but we do not quantify this connection and conclude only that the elevated velocity dispersions at high redshift could be a contributing factor in increasing the misalignment between the two axes.}
    \label{fig:delta_pa}
\end{figure*}

\cite{Wisnioski2015} report mean and median $\Delta PA$ values of 18$^{\circ}$ and 12$^{\circ}$ respectively and exclude galaxies from their rotating disk sample when $\Delta PA > 30^{\circ}$.
In the local universe the agreement between morphological and kinematic position angles is generally $\Delta PA < 20^{\circ}$ \citep[e.g.][]{Epinat2008,Barrera-Ballesteros2014,Barrera-Ballesteros2015} and it is shown that significant misalignment is often a signature of galaxy interaction.
In a recent paper by \cite{Rodrigues2016}, a subset of the KMOS$^{3D}$ galaxies at $z \sim 1$ have been re-analysed to establish the fraction of disk-like galaxies at this redshift given the criteria presented in section 4.1 of \cite{Wisnioski2015}.
In this paper the authors note that single \Sers fits to the HST photometry can introduce biases when the galaxies show prominent morphological features such as bars, spiral arms and tidal tails, and instead measure $PA_{morph}$ using the outer-most isophote in a series of concentric elliptical isophotes with the IRAF task ELLIPSE.
The resolved KDS galaxies at $z \sim 3.5$ have generally larger $\Delta PA$ values than those at lower redshift, and this observation has several potential explanations.
For the majority of galaxies we do not believe we are in a regime where prominent morphological features are resolved in the HST imaging, so that $PA_{morph}$ is biased by single \Sers fits as described in \cite{Rodrigues2016}.
However there are some cases where morphological tidal features could affect $PA_{morph}$ such as the GOODS-S galaxies bs006541, bs008543, cdfs\_lbg\_105, cdfs\_lbg\_109 and cdfs\_lbg\_111.
Another factor which could be partially driving increased misalignment in the KDS is the use of the full observed 2D velocity field to determine the kinematic position angle.
This generally does not give the same answer as simply connecting the velocity extrema, as we have verified by rotating $PA_{Kin}$ in 1$^{\circ}$ increments and choosing the value which maximises the velocity gradient, calling this $PA_{rot}$.
If the direction of the `zero-velocity strip' across the centre of the velocity maps is not perpendicular to the line connecting the velocity extrema, the $PA_{Kin}$ determined from MCMC will be different to the $PA_{Kin}$ determined from rotation (e.g. GOODS-S galaxy bs006541).
The observed velocity map contains meaningful information about $PA_{Kin}$ beyond just a consideration of the velocity extremes, and we believe it is more physically meaningful to allow the $PA_{Kin}$ to explore and choose values which best fit the full 2D observed velocity field.
As a test, extracting the $V_{C}$ values along $PA_{rot}$ makes xxx difference to the values on average for the sample. 
The $1-\sigma$ errorbars in Figure \ref{fig:delta_pa} are determined by the 16th and 84th percentile distribution of the MCMC samples.
For SSA22 the situation is complicated by the lower integration times of the HST data, and the use of ACS 814W imaging in which the galaxies have smaller apparent $R_{1/2}$ values due to observing light from younger stars.
This is a limitation of this study and the errorbars in $\Delta PA$ reflect the poorer imaging data in this field.
There do appear to be some galaxies that have low axis ratios, so that $PA_{morph}$ should be well constrained, and $\Delta PA$ values of $\sim$ 90$^{\circ}$ (e.g. GOODS-S galaxy cdfs\_lbg\_38 and SSA22 galaxy s\_sa22a-d3).
There does not appear to be any obvious explanation for why this is the case.
In practice, since the intrinsic [O~{\sc III}]$\lambda5007$ profile is determined by the GALFIT model throughout the modelling in \cref{subsec:3d modelling}, the $\Delta PA$ value has an impact on the derived $V_{C}$.
As $\Delta PA$ increases, the model beam smeared velocity reaches maximum value at larger radii, and so the inferred intrinsic $V_{C}$ becomes larger to provide an adequate fit to the data.
This effect is small until $\Delta PA$ reaches $\sim 60^{\circ}$, in which case the $V_{C}$ can be up to 30kms$^{-1}$ larger, with this an interesting consequence of modelling the intrinsic light profile with a different position angle to that of the velocity field.
We have also explored the possibility that $\Delta PA$ is correlated with $\sigma_{obs}$ and that the increased random motions or turbulence within the galaxies are the driving force behind kinematic and morphological misalignment.
There appears to be some suggestion that the two quantities are connected, although we make no attempt to quantify this, and conclude that the elevated velocity dispersions at high redshift could be a contributing factor in the observation of large mean $\Delta PA$.

We do not restrict the sample on the basis of $\Delta PA$ due to the increased uncertainty in determining both $PA_{morph}$ and $PA_{Kin}$ and due to the appearance of galaxies which have misalignment between $PA_{morph}$ and $PA_{Kin}$ at low axis ratios.  

\subsubsection{Observed and Intrinsic Properties}\label{subsubsec:observed_and_intrinsic}
In Figure \ref{fig:observed_and_intrinsic} we investigate correlations between the observed and intrinsic dynamical properties of the resolved KDS galaxies, to highlight the mean beam smearing correction factors, the mean values of the observed and intrinsic distributions and the factors driving the beam smearing corrections.
Figure \ref{fig:observed_and_intrinsic} (a) and (c) contain all resolved KDS galaxies with a valid $V_{C}$ measurement from the modelling sample, whereas (b) contains all resolved KDS galaxies, with $\sigma_{int}$ determined by subtracting the instrumental resolution value from $\sigma_{obs}$ for those galaxies without a valid $V_{C}$ measurement. 

The mean and median $\sigma_{obs}$ to $\sigma_{int}$ correction factors are 0.79 and 0.77 respectively. 
The mean and median $V_{C_{smeared}}$, the extrapolated and inclination corrected observed velocity, to $V_{C}$ correction factors, are 1.30 and 1.21 respectively.
In Figure \ref{fig:observed_and_intrinsic} (a) and (b) we find a strong positive correlation between both the observed and intrinsic velocities and velocity dispersions, which verifies that the beam smearing corrections do not alter the shape of the distributions of $V_{obs}$ and $\sigma_{obs}$ determined from the raw data.

Figure \ref{fig:observed_and_intrinsic} (c) highlights that galaxies with larger $V_{C}$ values tend to have larger $\sigma_{obs}$, which tells us that the broadening of the median velocity dispersion is driven by the strength of the velocity gradient, as assumed throughout the dynamical modelling section.

The most striking observation here is that the mean and median $V_{C}$ values in the KDS sample are substantially lower than reported by \cite{Gnerucci2011} for the AMAZE sample of galaxies at comparable $M_{\star}$.
This will be discussed further in \cref{subsubsec:sigma_and_v} and throughout \cref{subsec:redshift_evolution}.

\subsubsection{$\sigma_{int}$ and $V_{C}$}\label{subsubsec:sigma_and_v}
In figure \ref{fig:v_sig_and_v} (a) we plot the distribution of resolved KDS galaxies with $V_{C}$ measurements in $\sigma_{int}$ vs. log($V_{C}$) parameter space, along with the galaxies from G11 plotted in grey.
We also plot the $V_{C}$ = $\sigma_{int}$ line, which roughly bisects the KDS sample, and indicates the kinematic state of the galaxies as described in the following subsection. 

\begin{figure*}
    \centering \hspace{-1.3cm}
    \begin{subfigure}[h!]{0.5\textwidth}
        \centering
        \includegraphics[height=3.5in]{sigma_versus_v.png}
        %\caption{$\Delta PA$ vs. axis ratio}
    \end{subfigure} \hspace{0.4cm}
    \begin{subfigure}[h!]{0.5\textwidth}
        \centering
        \includegraphics[height=3.5in]{v_over_sigma_versus_v.png}
        %\caption{$\Delta PA$ vs. $\sigma_{obs}$}
    \end{subfigure}
    \caption{In panel (a) the distribution of all KDS galaxies with $V_{C}$ measurements in the $\sigma_{int}$ vs. $V_{C}$ plane is plotted with the red symbols, and AMAZE galaxies from G11 with grey.
    There is no strong correlation between the two parameters in either of these surveys, however several of the $V_{C}$ values in G11 are substantially larger or have $\sigma_{int}$ values consistent with 0kms$^{-1}$.
    This is despite the two samples having similar ranges in stellar mass, and is probably a consequence of the different modelling techniques used to infer the dynamical parameters.
    We assume that $\sigma_{int} = 30kms^{-1}$ for the G11 galaxies with reported $\sigma_{int} = 0kms^{-1}$, which is roughly equal to the spectral resolution of the instrument with which they were observed (SINFONI).
    The $V_{C}/\sigma_{int} = 1$ divides the galaxies classified as rotation dominated and dispersion dominated, with dispersion dominated galaxies sitting above the line. 
    The galaxy s\_sa22a-M38, classified here as a merger but classified in G11 as a rotating galaxy with $V_{C} = 346kms^{-1}$ is marked on the plot.
    In panel (b) the distribution of all KDS galaxies with $V_{C}$ measurements in the $V_{C}$/$\sigma_{int}$ vs. $V_{C}$ plane is shown.
    $V_{C}$/$\sigma_{int}$ increases as a function of $V_{C}$ following the beam smearing corrections applied to both quantities.
    We also mark the $V_{C}$/$\sigma_{int}=1$ line in this panel, finding that half of the KDS galaxies with $V_{C}$ measurements have $V_{C}$/$\sigma_{int} > 1$, despite all showing velocity gradients in the observations.
    We discuss the rotation dominated fraction with respect to the full sample in the text.}
    \label{fig:v_sig_and_v}
\end{figure*}

The KDS galaxies are clustered around a relatively tight region in both $V_{C}$ and $\sigma_{int}$ and there is excellent agreement between KDS and AMAZE for $ \sim 50\%$ of the G11 rotating sample.
The other $ \sim 50\%$ generally have much larger $V_{C}$ values, and three galaxies are reported as having $\sigma_{int}$ consistent with 0kms$^{-1}$.
We note also that one of the KDS galaxies in SSA22-a classified as a merger due to the double peak in the HST morphology, s\_sa22a-m38, is in the G11 rotating sample (highlighted in Figure \ref{fig:v_sig_and_v}), although the authors discuss the possibility that the galaxy could be either a close pair or two clumps embedded within a rotating disk.
This difference highlights the classification ambiguity for galaxies with double peaks in both ionised emission line and HST F160W profiles, as the galaxy could fall into either of the categories described above.
However in the case of s\_sa22a-m38 we also observe a double peak in the spectrum at the flux centre, indicative of two blobs at different redshifts and leading to artificial broadening of $\sigma_{obs}$ towards the centre of the galaxy when the dynamics are measured with single gaussian fits.
Several of the G11 galaxies plotted here also do not have reported errors on the $V_{C}$ values, and due to the lack of tabulated $V_{obs}$ values it is unclear the extent to which these have been extrapolated from the data.
It is these high $V_{C}$ G11 galaxies which drive the difference in mean and median $V_{C}$ between the samples.
Further comparison between the KDS and G11 $\sigma_{int}$ and $V_{C}$ values are made throughout \cref{subsec:redshift_evolution}. 

\subsubsection{Rotation Dominated Fraction and Merger Fraction}\label{subsubsec:rotation_and_mergers}
\begin{table}
    \centering
\begin{tabular}{ c c c c c c c c }

 \hline
 & N$_{Res}$ & N$_{Merg}$ & $\%$ Merg & N$_{Disp}$ & $\%$ Disp & N$_{RD}$ & RDF  \\
 \hline
G1 & 12 & 2 & 17 & 4 & 33 & 6 & 0.50 \\
G2 &  11 & 2 & 19 & 6 & 54 & 3 & 0.27\\
Sa & 9 & 8 & 89 & 1 & 11 & 0 & 0.0 \\
Sb & 11 & 1 & 9 & 6 & 54 & 4 & 0.37 \\
 \hline
\end{tabular}
\caption{The rotation and merger dominated fractions for each pointing}
\label{tab:rdf}
\end{table}

In Figure \ref{fig:v_sig_and_v} (b) we plot the ratio of $V_{C}$/$\sigma_{int}$ against $V_{C}$ for the KDS galaxies, with the former quantity commonly used as an indicator of whether SFGs at different redshifts are rotation dominated $(V_{C}$/$\sigma_{int} > 1)$ or dispersion dominated $(V_{C}$/$\sigma_{int} < 1)$ \citep[e.g.][]{ForsterSchreiber2009,Epinat2012,Wisnioski2015,Stott2016,Harrison2017}.
In the local universe, disky, SFGs are observed to have $V_{C}/\sigma_{int}$ values in excess of 10 (i.e. the disks are `well settled'), due to the decline of velocity dispersion across cosmic time, as gas fractions and disk turbulence decrease \citep[e.g.][]{Epinat2008,Epinat2008a}.
The KDS galaxies above the black, dashed $V_{C}/\sigma_{int} = 1$ line in figure \ref{fig:v_sig_and_v} are classified as rotation dominated, and we plot the full morpho-kinematic grids for these in \cref{app:rotation_dominated}.
The remainder are dispersion dominated and are plotted in \cref{app:dispersion_dominated}.
What is striking about many of the dispersion dominated galaxies is that they show clear velocity gradients across the observed velocity maps, and can generally be well fit with the arctangent model.
This suggests that, for these galaxies, although rotational motions are present within the disks, they are small or comparable with the elevated $\sigma_{int}$ values observed throughout the $z \sim 3.5$ sample.
This suggests in turn that the random motions traced by $\sigma_{int}$ are probably providing some degree of gravitational support for the total mass of the system (see \cref{subsubsec:dynamical_masses}), in contrast to the local universe in which late-type galaxies are supported entirely by ordered rotation.
In table \ref{tab:rdf} we list the merger, dispersion and rotation dominated fractions (RDFs) in each of the 4 pointings across GOODS-S and SSA22.
For all pointings except SSA22-a the detected merger fraction is low, however, as discussed already in \cref{subsec:morpho-kin-class}, almost every spatially resolved galaxy in SSA22-a is taking part in some kind of interaction event.
This observation has important consequences for the subsequent kinematic and morphological evolution of galaxies in high density environments, especially protoclusters, where the brightest cluster galaxies grow hierarchically through the aggregation of satellite mass.
We choose to omit this pointing from the analysis of a total rotation dominated fraction (RDF) since using the numbers from the high density environment would bias the RDF low.
For the other three pointings we combine the number of rotation dominated galaxies and divide by the total number of spatially resolved galaxies (including mergers and those not in the modelling sample) to arrive at the combined RDF of $0.39\pm 0.08$, where the error is the statistical error on the RDF, computed by generating bootstrapped samples, with replacement, and values perturbed by a random number drawn from a gaussian distribution with width given by the error on the original datapoints.
For each of the bootstrapped samples we re-compute the RDF and take the errors on the actual RDF as the 16th and 84th percentiles of the resulting distribution.
We have also calculated upper and lower RDF limits based on the individual $V_{C}/\sigma_{int}$ points and errors, evaluating the maximum and minimum number of galaxies which could scatter above $V_{C}/\sigma_{int} = 1$.
This leads to a larger error bar of $0.39^{+0.34}_{-0.27}$, plotted as a thin grey shaded region in \ref{fig:rdf_and_v_sigma_w_redshift} and the statisical value is represented by the black error bar.
This value is in good agreement with the result of G11, however we note that their result of $RDF \sim 0.33$ could be higher if any of the galaxies which were not passed through their dynamical model were rotation dominated.
The number would also decrease if s\_sa22a-m38 were classified as a merger as described in \cref{subsubsec:sigma_and_v}.
We interpret these results in the context of the cosmic evolution of the RDF in \cref{subsubsec:RDF_evolution}.

\subsubsection{Lack of Dynamical Mass}\label{subsubsec:dynamical_masses}

\begin{figure}
\centering
\includegraphics[width=0.48\textwidth]{rv_squared.png}
\caption{We plot the ratio $M_{vir}/M_{\star}$ computed using only rotational velocities (Eq. 13) for the KDS galaxies with $V_{C}$ measurements vs. $M_{\star}$, with the black line indicating equality between the two quantities.
The red symbols show the galaxies with $V_{C}/\sigma_{int} > 1$ and the clear symbols show the galaxies with $V_{C}/\sigma_{int} < 1$, which these galaxies having on average lower $M_{vir}$.
The majority of the points lie in the unphysical $M_{vir} < M_{\star}$ region.
This highlights the potential for a combination of random motions, traced by $\sigma_{int}$, and ordered rotation to play a role in supporting the total dynamical mass, an expression for which is given by Eq. 14 in the text.}
\label{fig:dyn_masses}
\end{figure}

Further evidence for the provision of gravitational support by random motion traced by $\sigma_{int}$ can be found by comparing the virial mass computed at the convovled 2R$_{1/2}$ radius, $M_{vir}$, with $M_{\star}$.
It is expected that $M_{vir} > M_{\star}$, as the dynamics trace not only the stellar mass, but also the gas mass and dark matter mass within the radius at which $M_{vir}$ is evaluated.
At $z \sim 3.5$ the gas reservoirs are on average a much larger fraction of the total galaxy mass than in the local universe \citep[e.g.][]{Tacconi2013,Saintonge2013,Wisnioski2015} as there has not yet been time to convert this gas mass into stellar mass, and so in principle if ordered rotation is tracing the total mass in these systems, $M_{\star}$ should be clearly below $M_{vir}$.
In figure \ref{fig:dyn_masses}, where we have computed the virial mass from the ordered rotational motions alone, as traced by $V_{C}$, following equation \ref{eq:dyn_mass_v_only}, we find that this is not the case for the modelling sample, with $M_{vir} < M_{\star}$ for the vast majority of galaxies (particularly those with $M_{\star} > 9.5$).

\begin{equation}\label{eq:dyn_mass_v_only}
   M_{vir} = \frac{2R_{1/2}V_{C}^{2}}{G}
\end{equation}

This strongly suggests that rotational motions alone are not enough to provide gravitational support for the mass in the system, and random motions contribute to provide gravitational support for some fraction of the total mass contained within these compact galaxies.
To follow this idea we can compute the contribution to $M_{vir}$ from $\sigma_{int}$ using equation \ref{eq:dyn_mass_sigma} \citep[e.g.][]{Cappellari2006}.

\begin{equation}\label{eq:dyn_mass_sigma}
   M_{vir} = \frac{2R_{1/2}V_{C}^{2}}{G} + \frac{\beta R_{1/2}\sigma_{int}^{2}}{G}
\end{equation}     

In general the amount gravitational suport provided by $\sigma_{int}$ depends on galaxy geometry and structure \citep[e.g.][]{Cappellari2006,Belli2014}, with this dependence contained in the value of $\beta$ in equation \ref{eq:dyn_mass_sigma}.
It is possible to derive a relationship between $\beta$ and \Sers index, a proxy for galaxy structure, as detailed in \cite{Cappellari2006} and shown in equation \ref{eq:beta_n}.

\begin{equation}\label{eq:beta_n}
   \beta(n) = 8.87 - 0.831n + 0.0241n^{2}
\end{equation} 

%Now simply stating that it is possible to find the lower limit on beta to bring the dynamical masses above the stellar masses, BUT that realistically the data do not allow us to figure out the contribution from sigma so just quoted for interest

We have modelled the galaxies throughout the analysis as n = 1 exponential disks, and verified with single \Sers fits in GOODS-S that this is close to the case for most of the galaxies, for which $\beta(n) \sim 8$.
For galaxies below the $M_{vir} = M_{\star}$ line, we can find the minimum value of $\beta$ required to bring the two quantities to equality.
In both the $V_{C}/\sigma_{int} > 1$ and $V_{C}/\sigma_{int} < 1$ samples we find that the mean $\beta_{min}\sim 5$, although we do not attempt to use $M_{vir}$ for any further analysis, merely to demonstrate the need for gravitational support that originates in random motions, as characterised by $\sigma_{int}$.
 
\subsection{The redshift evolution of dynamical properties}\label{subsec:redshift_evolution}
Having observed low rotational velocities and large velocity dispersions in our $z \sim 3.5$ sample, and compared these with the G11 values at similar redshift, we now aim to place the results in cosmological perspective and interpret the data in the context of galaxy evolution.
Throughout the following sections we compare the median dynamical properties of several IFU studies over a wide redshift baseline \citep{Green2014,Wisnioski2015,Epinat2012,ForsterSchreiber2009,Cresci2009,Gnerucci2011}, and also compare with Fabry-Perot data from the GHASP sample \citep{Epinat2008,Epinat2008a} as an indicator of the behaviour of SFGs in the local universe.
The galaxy selection criteria in these surveys, with the exception of \cite{Green2014}, consistently pick out main sequence SFGs, and trace the dynamical properties of these by observing either the [O~{\sc III}]$\lambda$5007 or $H\alpha$ ionised gas emission line.
In \cite{Green2014}, half of the galaxies at $z\sim 0.1$ are selected to have high sSFR representative of SFGs at $z\sim 2$, and are also found to have higher $f_{gas}$ values than locally, providing a local comparison sample to the high redshift universe. 
The galaxies in these surveys have different $M_{\star}$ distributions and we attempt to make clear in the following figures, with the symbols and shaded regions, where the results are directly comparable. 

We have made use of the tabulated data and errors in these studies to derive the datapoint values and errorbars plotted in figure \ref{fig:rdf_and_v_sigma_w_redshift} and \ref{fig:sigma_and_v_sigma_w_redshift} in a consistent way, which is to take the fractional error weighted mean of $V_{C}/\sigma_{int}$, $\sigma_{int}$ and $V_{C}$ in each of the samples (i.e. we do not want the derived values with extremely large errors to dominate the averages).
The errors on these mean values are computed in a statistical sense, by generating bootstrapped samples, with replacement, with size equivalent to the original survey sample size and with values perturbed by a random number drawn from a gaussian distribution with width given by the error on the original point.
In \cref{subsec:3d modelling} we have described the variation in the methodology used to compute $V_{C}$ and $\sigma_{int}$ between different surveys as a way to help disentangle true evolution from offsets between derived dynamical values caused by disparate measurement techniques.
It appears that choice of dynamical model and dynamical parameter extraction method does still play a large role in understanding the evolution of dynamical parameters, and for this reason we have chosen to closely follow the methods described in \cite{Harrison2017} to allow for a consistent comparison between $z \sim 3.5$ and $z \sim 1$. 
In the following sections we will make statements about dynamical evolution by connecting the dots of these different surveys, assuming that on average they are tracing a population of SFGs which evolve across cosmic time. 

\subsubsection{The evolution of the rotation dominated fraction}\label{subsubsec:RDF_evolution}

\begin{figure*}
    \centering \hspace{-1.3cm}
    \begin{subfigure}[h!]{0.5\textwidth}
        \centering
        \includegraphics[height=3.5in]{v_sigma_evolution.png}
        \caption{$V_{C}/\sigma_{int}$ vs. redshift}
    \end{subfigure} \hspace{+0.4cm}
    \begin{subfigure}[h!]{0.5\textwidth}
        \centering
        \includegraphics[height=3.5in]{rdf_evolution.png}
        \caption{RDF vs. redshift}
    \end{subfigure}
    \caption{In panel (a) $V_{C}/\sigma_{int}$ is plotted against redshift for surveys spanning $0 < z < 4$.
    The filled symbols show the sample mean values, with the KDS mean at $z\sim3.5$ plotted with the red circle, with black errorbars representing the statistical error on the mean computed using bootstrap re-sampling as described in the text.
    The sample medians and statistical errors on these values are plotted with the coloured and grey horizontal bars, falling in each case below the mean, suggesting that large individual $V_{C}/\sigma_{int}$ measurements may bias the sample mean high.  
    We also plot the 16th and 84th percentiles of the distribution of individual $V_{C}/\sigma_{int}$ measurements as shaded regions to give an indication of the range of values measured in each survey, and indicate the mean stellar mass of each survey in the legend.
    Lack of tabulated values for KMOS$^{3D}$ and SINS restricted us from computing the $V_{C}/\sigma_{int}$ values in a consistent way, and so we plot only the sample means without errorbars and caution against comparison of these with the other surveys.
    The DYNAMO point and errors are plotted in white to highlight the difference in selection criteria, with the low redshift galaxies chosen to have similar properties to those at $z\sim2$.
    There is a clear decline in the mean $V_{C}/\sigma_{int}$ values with redshift, but with wide ranges in the individual measurements.
    There is also a connection to the mean $M_{\star}$ of the surveys, with larger mean $M_{\star}$ surveys reporting higher mean $V_{C}/\sigma_{int}$.
    In panel (b) we plot the associated RDF, with this deriving from the fraction of galaxies above the $V_{C}/\sigma_{int} = 1$ dashed line plotted in panel (a).
    The narrow shaded regions represent the maximum and miminum RDF, computed as described in the text.
    Again, it was not possible to directly compute the RDF in a consistent way for the KMOS$^{3D}$, SINS and AMAZE galaxies and we simply plot the quoted values for these surveys.
    It appears that the RDF drops from $\sim 100\%$ in the local universe to $60\%$ at intermediate redshifts and then to $\sim 40\%$ at z $\sim 3.5$, again with the trend for high $V_{C}/\sigma_{int}$ values in the larger $M_{\star}$ surveys reflected by higher RDFs.
    Presumably this decline would continue out to the higher redshifts in which galaxies really are assembling their first stellar components are not supported by ordered rotation.
    The dot-dash line is the RDF $\propto z^{-0.2}$ cosmic decline suggested in \protect\cite{Stott2016}.
    In both panels the surveys from which the data are drawn are denoted in the legends.
}
    \label{fig:rdf_and_v_sigma_w_redshift}
\end{figure*}

In figure \ref{fig:rdf_and_v_sigma_w_redshift} (a) we plot $V_{C}/\sigma_{int}$ and in (b) we plot the closely connected RDF both as a function of redshift for the KDS modelling sample (omitting the SSA22-a pointing from the RDF calculation as explained in \cref{subsubsec:rotation_and_mergers}).
The $V_{C}/\sigma_{int}$ values do not change significantly across the individual KDS pointings, and plotting these independently would not change the results.
We measure the mean $V_{C}/\sigma_{int} = 1.13^{+0.18}_{-0.15}$ and median $V_{C}/\sigma_{int} = 0.99^{+0.14}_{-0.11}$ and RDF = $0.39 \pm 0.08$ for the KDS galaxies.
For the samples in which tabulated $V_{C}$ and $\sigma_{int}$ values are available we compute the fractional error weighted mean and median of $V_{C}/\sigma_{int}$, along with associated errors, as described in \cref{subsec:redshift_evolution}.
The samples for which the mean $V_{C}/\sigma_{int}$ and RDF values can be computed in a consistent way to the KDS galaxies are GHASP \citep[E08]{Epinat2008} $log(M_{\star}[M_{\odot}])=10.6$, DYNAMO \citep[G14]{Green2014} $log(M_{\star}[M_{\odot}])=10.3$ (although with the caveat that the selection criteria was designed to pick out SFGs representative of the $z\sim2$ universe), KROSS \citep[H17]{Harrison2017} $log(M_{\star}[M_{\odot}])=9.9$ and MASSIV \citep[E12]{Epinat2012} $log(M_{\star}[M_{\odot}])=10.2$.
For these surveys, in figure \ref{fig:rdf_and_v_sigma_w_redshift} (a) the filled symbols and black errorbars represent the mean and error on the mean, and the horizontal lines show the median and error on the median.   
The shaded regions give the 16th and 84th percentiles of the distribution on individual $V_{C}/\sigma_{int}$ measurements, plotted here to indicate the range of measurements at each redshift slice.
In figure \ref{fig:rdf_and_v_sigma_w_redshift} (b) the filled symbols with errorbars denote the surveys for which the RDF is computed as the fraction of galaxies with $V_{C}/\sigma_{int} > 1$.
We have also computed the maximum and minimum RDF for these surveys by adding and subtracting the $V_{C}/\sigma_{int}$ error values respectively and recomputing the RDF.
These are shown with the thin grey shaded regions, and are intended to give an indication of the limits of the RDF at different redshift slices given the size of the errors on the invidual points.
The filled symbols without errorbars represent results quoted in other works.
In both panels (a) and (b) the KDS mean is plotted as the large red circle at $z\sim3.5$.

There is a further caveat when comparing with the G11 ($log(M_{\star}[M_{\odot}])=9.9$) values; since dynamical properties are only computed for 11/33 rotationally dominated galaxies in their sample, the mean $V_{C}/\sigma_{int}$ values are likely to be biased high when compared with the mean values computed from other surveys which contain a mix of rotation and dispersion dominated galaxies.
The shaded percentile regions and errors for the AMAZE data in figure \ref{fig:rdf_and_v_sigma_w_redshift} (a) are shown for reference, and we note that for their three galaxies with $\sigma_{int} = 0kms^{-1}$ we assume $\sigma_{int} = 30kms^{-1}$ (roughly equal to the SINFONI spectral resolution) in order to compute the $V_{C}/\sigma_{int}$ ratio and mean and median $\sigma_{int}$ values.
For those galaxies in G11 that have no errorbars on $V_{C}$ due to the parameter being unconstrained, we assume a fractional error of 1.
The G11 RDF, taken as 33\% is also difficult to compare against and is a minimum in the sense that some of the galaxies for which dynamical properties have not been computed could be rotation dominated.
We plot the G11 RDF with the clear symbol and without errorbars in \ref{fig:rdf_and_v_sigma_w_redshift} (b). 

Throughout the KMOS$^{3D}$ study of \cite[W15]{Wisnioski2015} two redshift ranges are considered, the first at $z\sim 1$ with $log(M_{\star}[M_{\odot}])=10.65$ and the second at $z\sim 2.2$ with $log(M_{\star}[M_{\odot}])=10.86$.
No tabulated values are provided, so we plot here the $V_{C}/\sigma_{int}$ and RDF presented in W15 without errorbars.
For the SINS sample \citep[FS09]{ForsterSchreiber2009} and \citep[C09]{Cresci2009} ($log(M_{\star}[M_{\odot}])=10.6$), we plot without errorbars the mean $V_{C}/\sigma_{int}$ value in figure \ref{fig:rdf_and_v_sigma_w_redshift} (a) for the full sample presented in W15, since this cannot be directly computed from the tabulated integrated velocity dispersions in FS09.
We plot also in figure \ref{fig:rdf_and_v_sigma_w_redshift} (b) RDF for the full sample, computed in FS09 section 9.5.1, which classifies rotation dominated galaxies as those where $V_{obs}/(2\sigma_{tot}) > 0.4$, with $\sigma_{tot}$ equal to the velocity dispersion measured from the galaxy integrated spectrum.
In a recent work by \cite[R16]{Rodrigues2016}, a subset of the KMOS$^{3D}$ $z \sim 1$ sample has been re-analysed and a lower RDF of 69\% was recovered, despite the use of identical classification criteria.
The recovery of this lower fraction is driven mainly by the presence of peaks in the $\sigma_{int}$ maps which do not coincide with the rotation centre, a criteria which we do not impose throughout this work, and spatial offsets between the rotation centre and the centre of stellar mass.
We also do not impose the coincidence of $PA_{kin}$ and $PA_{morph}$ as described in \cref{subsubsec:kin_misalignment} as a rotation dominated criteria due to the increased uncertainties in measuring both of these quantities, although we note here that our KDS RDF would drop from 38\% to 16\% due to the loss of 7 galaxies from the rotation dominated sample if we require $\Delta PA < 30^{\circ}$. 
The KROSS $z\sim1$ RDF presented in H17 lies between the two extremes of R16 and W15 computed for the KMOS$^{3D}$ sample.

Despite the systematic variations in measurement technique and classification criteria, it is clear that the RDF, as traced simply by the $V_{C}/\sigma_{int}$ inferred from ionised gas emission lines, drops from $\sim 100\%$ of galaxies in the local universe to $\sim 2/3$ of galaxies in the $z \sim 1-2$ universe and to $\sim 1/3$ above $z \sim 3$, albeit with individual surveys reporting wide $V_{C}/\sigma_{int}$ distributions at each redshift slice.
This is highlighted in figure \ref{fig:rdf_and_v_sigma_w_redshift} (b) by overplotting the RDF $\propto z^{-0.2}$ line which appears to roughly follow the cosmic RDF decline as described in \cite{Stott2016}. 
Scatter above and below the RDF $\propto z^{-0.2}$ evolution line may be attributed in part to the different mean stellar masses of the surveys, with those that have larger stellar masses tending to scatter above this line.
Higher $M_{\star}$ values suggest that the galaxies are more evolved, with more gas having been converted into stars for the bulk of the sample, which provides stability for a rotating disk and pushes the sample RDF higher.
The width of the $V_{C}/\sigma_{int}$ distributions indicates the galaxy diversity at each redshift slice, driven by the collection of galaxy masses, sSFRs, gas fractions, morphologies and evolutionary states which comprise each sample.
Explanations for the mean decline of $V_{C}/\sigma_{int}$ with cosmic time have been discussed at length in other surveys \citep[e.g][]{ForsterSchreiber2009,Law2009,Wisnioski2015} and we revisit this in the following subsection.

\subsubsection{The evolution of velocity dispersion}\label{subsubsec:sigma_evolution}

\begin{figure*}
    \centering \hspace{-1.3cm}
    \begin{subfigure}[h!]{0.5\textwidth}
        \centering
        \includegraphics[height=3.5in]{sigma_evolution.png}
        \caption{$\sigma_{int}$ vs. redshift}
    \end{subfigure} \hspace{0.4cm}
    \begin{subfigure}[h!]{0.5\textwidth}
        \centering
        \includegraphics[height=3.5in]{v_evolution.png}
        \caption{$V_{C}$ vs. redshift}
    \end{subfigure}
    \caption{Panel (a) presents a compilation of literature $\sigma_{int}$ values plotted against redshift for surveys spanning $0 < z < 4$.
    The filled symbols show the sample mean values, with the KDS mean at $z\sim3.5$ plotted with the red circle, with black errorbars representing the statistical error on the mean computed using bootstrap re-sampling as described in the text.
    The sample medians and statistical errors on these values are plotted with the coloured and grey horizontal bars, falling in each case below the mean, suggesting that large individual $\sigma_{int}$ measurements may bias the sample mean high.  
    We also plot the 16th and 84th percentiles of the distribution of individual $\sigma_{int}$ measurements as shaded regions to give an indication of the range of values measured in each survey, and indicate the mean stellar mass of each survey in the legend.
    Lack of tabulated values for KMOS$^{3D}$ restricted us from computing the $\sigma_{int}$ values in a consistent way, and so we plot only the sample means without errorbars and caution against comparison of these with the other surveys.
    The DYNAMO point and errors are plotted in white to highlight the difference in selection criteria, with the low redshift galaxies chosen to have similar properties to those at $z\sim2$.
    The $\sigma_{int}(z)$ scaling relation from W15, the form of which is shown in the top right equation, is plotted for three different $M_{\star}$ values, with the solid line showing $log(M_{\star}[M_{\odot}])=10.0$, the dashed line $log(M_{\star}[M_{\odot}])=10.3$ and the dot-dashed line $log(M_{\star}[M_{\odot}])=10.6$.
    In each case the scaling relation is plotted for $V_{C} = 150kms^{-1}$.
    These are intended as indicators of the way in which $\sigma_{int}$ evolves in galaxies in different mean $M_{\star}$, where this on average gives an indication of the extent to which the stellar population has been accumulated, which may provide increased stability for an extended rotating disk.
    The KDS datapoint appears to be consistent with the scenario proposed in previous work, whereby the mean $\sigma_{int}$ increases over cosmic time, with large scatter between individual galaxies, as a result of increased gas fractions and more efficient accretion of cold gas.
    The vertical location of the mean datapoints in this plane are mediated by the mean $M_{\star}$ of each survey, which is itself connected to the mean gas fraction. 
    We note also that the Law 2009 datapoint has not been corrected for beam smearing and so represents an upper limit on $\sigma_{int}$ for that survey. 
    Panel (b) presents the mean, median and distributions of $V_{C}$ measurements of these surveys between $0 < z < 4$, with the large red circle representing the KDS measurement.
    We again present the KMOS$^{3D}$ mean points without errors and distributions.} 
    \label{fig:sigma_and_v_sigma_w_redshift}
\end{figure*}

The observation that $\sigma_{int}$ becomes increasingly large towards high redshift has now been firmly established in several studies, including all of those from which data has been taken to create figure \ref{fig:sigma_and_v_sigma_w_redshift}.
To complement the surveys already described throughout \cref{subsubsec:RDF_evolution}, we add data from \cite{Law2009} to explore the evolution of $\sigma_{int}$ values across cosmic time.
In this study, resolved data are collected with the OSIRIS instrument for 13 SFGs with $log(M_{\star}[M_{\odot}])=10.0$.
We note that the $\sigma_{mean}$ values and errors tabulated in this paper are the flux weighted mean and standard deviation of the measurements in individual spaxels in each galaxy, not corrected for beam smearing in any way.
These measurements are expected to be larger than the $\sigma_{int}$ measurements at similar redshift which have beam beam smearing corrected, or extracted at the galaxy outskirts where the beam smearing is much smaller.
\cite{Law2009} attribute the cosmic increase in $\sigma_{int}$ to much larger gas fractions at high redshift, with the gas accumulated through rapid cold-mode accretion along cosmic filaments.
Systems such as these form roughly spherical stellar distributions with large internal motions at early times, before more gradual accretion from the galactic halo forms an extended, stable, gaseous disk.
Following this, $\sigma_{int}$ decreases over time as ordered rotations begin to dominate.
\cite{Wisnioski2015} interpret the increase of $\sigma_{int}$ towards high redshift in terms of the equilibrium model, which describes the rough balance between galactic accretion, star-formation and outflows \citep[e.g.][]{Lilly2013}, deriving a scaling relation for $\sigma_{int}$ with redshift appropriate for galaxies which have formed a quasi-stable disk (see their section 5.2).
This relation gives a prediction for $\sigma_{int}$ in terms of the key observable properties $f_{gas}$, $t_{depl}$ and sSFR, and using observed constraints for these properties leads to $\sigma_{int}$ predictions with redshift that match the data remarkably well (see \cite{Wisnioski2015} figures 10 and 11).
In both viewpoints the increased $\sigma_{int}$ values observed at high redshift are primarily the result of the prevalence and efficiency of gas inflow onto galaxies in this regime.
In figure \ref{fig:sigma_and_v_sigma_w_redshift} (a) we plot the fractional error weighted mean, median and distribution of $\sigma_{int}$ from tabulated data, as described in detail in the figure caption.
For the KMOS$^{3D}$ points at $z\sim 1$ and $z\sim 2.2$, there are no tabulated values and so we again plot the mean $\sigma_{int}$ values listed in W15 without errorbars.
The mean KDS datapoint is shown with the large red symbol with $\sigma_{int} = 70.8^{+3.3}_{-3.1} kms^{-1}$ (median $\sigma_{int} = 67.0^{+5.0}_{-4.8} kms^{-1}$) at z = 3.5, which is in agreement with the $\sigma_{int}$ scaling relation presented in W15, plotted with the grey lines for $V_{C} = 150kms^{-1}$ and $log(M_{\star}[M_{\odot}])=10.0,10.3,10.6$, rather than the $log(M_{\star}[M_{\odot}])=10.5$ value used in \cite{Wisnioski2015} (note we have also not shifted the datapoints to conform with a single $M_{\star}$ value as per \cite{Wisnioski2015}).
The lower $M_{\star}$ tracks suggest a flatter evolution of $\sigma_{int}$ with redshift, and indeed the surveys with lower average $M_{\star}$ (MASSIV, DYNAMO, LAW 2009, KROSS, AMAZE) have higher $\sigma_{int}$ than the surveys with larger average $M_{\star}$ (GHASP, KMOS$^{3D}$, SINS), although the exact values are sensitive to the choice of $Q_{crit}$ and $V_{C}$ in the equation shown in the top right of figure \ref{fig:sigma_and_v_sigma_w_redshift} (a).

The same caveats apply as described above surrounding the comparison of values derived from different measurement techniques, however it appears that the evolution of $\sigma_{int}$ can be well described in the context of the equilibrium model by the evolution of the gas fraction with redshift \citep{Wisnioski2015}.
The $M_{\star}$ value of a galaxy is linked to the size of the gas reservoir, since gas is depleted to build the stellar population.
The location of mean $\sigma_{int}$ values is therefore related to the mean $M_{\star}$ of each survey, with this being an indicator of the accumulation of a stellar population and hence the evolutionary state of the galaxies.
The tracks plotted in figure \ref{fig:sigma_and_v_sigma_w_redshift} give a rough indication of how $\sigma_{int}$ evolves in galaxies of a particular stellar mass in the context of the equilibrium model, and the KDS galaxies provide a strong observational constraint on the internal dynamics of main sequence galaxies between $z=3-3.7$.

\subsubsection{The evolution of maximum circular velocity}\label{subsubsec:v_evolution}
In figure \ref{fig:sigma_and_v_sigma_w_redshift} (b) we plot the fractional error weighted mean, median and distributions of $V_{C}$ values reported in the different surveys against redshift, as explained in the caption of figure \ref{fig:sigma_and_v_sigma_w_redshift} to highlight two observations:
\begin{itemize}
    \item The KDS $V_{C}$ values appear lower than reported in lower redshift studies with similar mean $M_{\star}$ values, as a result of averaging over a sample with a higher dispersion dominated fraction
    \item Different modelling and measurement techniques affect the inferred $V_{C}$ values
\end{itemize}

In a rotationally supported system, the magnitude of $V_{C}$ is set by the requirement to support the mass, $M_{tot}$, in the system, i.e. the combination of $M_{gas}$, $M_{\star}$ and $M_{DM}$, against gravitational collapse.
$V_{C}$ increases as $M_{tot}$ increases and so we would expect the systems with larger average $M_{\star}$, assuming that these are indicative of systems with greater $M_{tot}$, to have larger $V_{C}$, which appears generally to be the case in figure \ref{fig:sigma_and_v_sigma_w_redshift} (b) where the KMOS$^{3D}$, GHASP and SINS galaxies have the largest inferred $V_{C}$ values.
Using the average $M_{\star}$ values and velocity extraction radius in the KDS sample, $log(M_{\star}[M_{\odot}])=10.0$ and $2R_{1/2} = 3.2kpc$ respectively, and assuming simple, circular Keplerian orbits where the rotational motions alone support the mass of the system, we can calculate a rough lower limit of $\left<V_{C}\right>$ $>$ $125kms^{-1}$ for the baryonic material in the galaxies (since the inclusion of $M_{gas}$ and $M_{DM}$ would raise this estimate).
Indeed this estimate is similar to the mean value found in the KROSS sample for $log(M_{\star}[M_{\odot}])=10.0$ galaxies at $z\sim1$.
In figure \ref{fig:sigma_and_v_sigma_w_redshift} (b) we plot the mean and median values and percentiles of the $V_{C}$ distribution for the different surveys, finding wide distributions which reflect the mass range and the mixture of rotation/dispersion dominated galaxies which constitute the samples.
The KDS datapoint, represented by the large red circle, has mean $V_{C} = 78.8^{+5.4}_{-5.1}kms^{-1}$ and median $V_{C} = 60.0^{+7.5}_{-5.9}kms^{-1}$, both of which are below the $\left<V_{C}\right> = 125kms^{-1}$ limit.
This could either be a result of the measurement and modelling technique, which carries significant uncertainties due to modelling assumptions, the S/N of the [O~{\sc III}]$\lambda5007$ emission line even in very long integrations at this redshift, the spatial coverage of [O~{\sc III}]$\lambda5007$ may be more compact that $H\alpha$ and consequently may not probe out to the same galactic radii and the sparsity of resolution elements, or could be the result or some real physical process in the galaxies which lowers $V_{C}$ as traced by the ionised gas dynamics.
Also, since there is a higher fraction of dispersion dominated galaxies at $z\sim3.5$, the mean $V_{C}$ value of the complete sample is likely to be biased low.
This is easier to see in \cref{subsec:TF} where we revisit the rotational velocities in the context of the inverted stellar mass `Tully-Fisher Relation' \citep[smTFR;][]{Tully1977,Bell2000a}.

It seems likely that the measurement of low $V_{C}$ values in the KDS could be at least in part a consequence of the KMOS data not having the depth to trace the outskirts of the galaxies where the rotation curve begins to flatten.
Using the 2D arctangent function to extrapolate through the datapoints to $2R_{1/2}$ could underestimate the true $V_{C}$ if the data do not trace close to where the rotation curve begins to flatten.
We have also assumed that the galaxies are disky in order to compute an inclination angle in each case, which is then used to correct $V_{obs}$ for viewing angle.
At this redshift and mean $M_{\star}$, the disk assumption may not be correct, as galaxies could have more complicated structures which have not yet stabilised into rotating disks.
However the distribution of inclination angles is consistent with observing a selection of galaxies with inclinations randomly distributed as $0^{\circ} < i < 90^{\circ}$, with the KDS mean inclination of $59.5^{\circ}$ agreeing well with the theoretical mean of $57.3^{\circ}$ (for a simple derivation of this value see e.g. the appendix in \cite{Law2009}).
There are some galaxies where the observed velocity field has obvious rotational structure and clearly reaches flattening (e.g. lbg\_24 with $log(M_{\star}[M_{\odot}])=9.75$), yet the inferred $V_{C}$ is smaller than $\sigma_{int}$.
This, in combination with figure \ref{fig:dyn_masses}, suggests that the velocity dispersion could be responsible for providing partial dynamical support in these systems.
\cite{Law2009} observe galaxies with similarly low mean velocity shear and high velocity dispersions (the velocities are not plotted here since the tabulated values are not inclination or beam smearing corrected) and attribute this to instabilities which occur when the cold gas becomes dynamically dominant following efficient accretion and in line with the high gas fractions observed at high redshift. 
We caution that at high redshift these $V_{C}$ measurements are challenging for the reasons stated above, however our data appear to support a scenario in which low $M_{\star}$ galaxies experience elevated random motions related to the rapid accretion of cold gas, which is partially supported gravitationally by these random motions and may contribute to the observation of low $V_{C}$ values.
At higher $M_{\star}$ the accumulation of a stellar population provides stability, $\sigma_{int}$ decreases and $V_{C}$ increases to provide the additional gravitational support.
These processes are mediated across redshift by the cosmic decline of gas fractions, sSFRs and accretion. 

The second point is that it is difficult to assess whether the $V_{C}$ values reported by different surveys (the modelling and extraction techniques differ and are described in \cref{subsubsec:param_extraction}) are directly comparable.
The MASSIV, KROSS, KMOS$^{3D}$ and KDS samples should in principle yield comparable $V_{C}$ values which rely on extracting from the data at a fiducial radius or using a velocity shear value, but the comparison to the dynamical models of SINS and AMAZE is not so clear.
In particular the $V_{C}$ parameter reported in G11 is unconstrained for several galaxies, requires large extrapolation from the data and leads to very large rotational velocities for 5/11 of their galaxies.
The remaining 6/11 galaxies have $V_{C}$ values close to those reported for the KDS sample.
This is a consequence of the current state of the data quality at high redshifts, and it is important to keep in mind that both $\sigma_{int}$ and $V_{C}$ are still dependent on modelling technique.

\subsection{The z = 3.5 Tully-Fisher Relation}\label{subsec:TF}

\begin{figure*}
    \centering \hspace{-1.3cm}
    \begin{subfigure}[h!]{0.5\textwidth}
        \centering
        \includegraphics[height=3.5in]{tully_fisher_standard_log.png}
        \caption{smTFR}
    \end{subfigure} \hspace{0.4cm}
    \begin{subfigure}[h!]{0.5\textwidth}
        \centering
        \includegraphics[height=3.5in]{tully_fisher_s0_log.png}
        \caption{S$_{0.5}$ Relation}
    \end{subfigure}
    \caption{In panel (a) we plot log($V_{C}$) vs. $M_{\star}$, i.e. the inverted stellar mass Tully-Fisher relation for the KDS galaxies with $V_{C}/\sigma_{int} > 1$ in red and the galaxies with $V_{C}/\sigma_{int} < 1$ with clear symbols.
    Also plotted with the solid green is the $z\sim0.9$ relation recovered from a fit to $\sim400$ rotation dominated galaxies from the KROSS survey, with the shaded region showing the associated $\sim$0.2 dex error on the fit.
    This $z\sim0.9$ relation shows no signifcant evolution from the fit to a local sample of spiral galaxies with $z < 0.1$, presented in Reyes 2011.
    The $V_{C}/\sigma_{int} > 1$ KDS galaxies also show no evolution from this relation within the errorbars, with the $V_{C}/\sigma_{int} < 1$ galaxies scattering below the relation.
    This highlights the need for careful sample selection when constructing this relation, and explains the low KDS mean $V_{C}$ value as a consequence of averaging over a sample with a high fraction of dispersion dominated galaxies.
    In panel (b) we plot the $S_{0.5}$ vs. $M_{\star}$ relation, which accounts for a contribution to the total dynamical mass from non-circular motions as traced by $\sigma_{int}$.
    The symbol convention is the same as panel (a) and we include also the KDS galaxies with only $\sigma_{int}$ measurements as the black symbols.
    We plot also the K07 relation with the black dashed line, and find that the KDS galaxies agree with this, showing much less separation between the rotation and dispersion dominated galaxies and with less scatter than in panel (a).
    We caution against over-interpretation of this result, since it is unclear what the physical origins of the random motions traced by $\sigma_{int}$ are, and the extent to which they are contributing to support dynamical mass.}
    \label{fig:tf_relation}
\end{figure*}

In the $\Lambda$CDM paradigm the smTFR relates the angular momentum of the dark matter halo within which disk galaxies reside to the $M_{\star}$ content of these galaxies.
Angular momentum in the halo, which is in turn affected by accretion of gas from the IGM, is transferred to the disk \citep{Fall1983}, where the gas then cools radiatively over time, forming the stellar population of the galaxy.
Observing the smTFR at different epochs thus offers constraints for galaxy formation and evolution models seeking to explain simultaneously the properties of dark matter halos and the dynamical properties of disk galaxies.
In principle, studying whether the smTFR evolves with redshift gives us information about how galaxies accumulate their stellar mass and rotational velocities, with offsets from the local relation towards lower $M_{\star}$ values at fixed rotational velocity suggesting that the dynamical mass is in place, but the stellar mass is yet to be formed.
The main question arising from observations in support of and against evolution of the smTFR with redshift is: Do galaxies evolve along a smTFR which is fixed with redshift (i.e. increasing in $V_{C}$ as $M_{\star}$ increases) or is there a shift in the zero-point and/or slope with redshift (i.e. as $M_{\star}$ accumulates, $V_{C}$ remains fixed).
Essentially we'd like to know how an individual, unperturbed point, representing a typical rotation-dominated SFG allowed to evolve without mergers, would move around the $V_{C}$ vs. $M_{\star}$ plane over 12Gyr of purely secular evolution.
Connected to this also is the idea of progenitor bias, in that the SFGs we are selecting and comparing between $z\sim0-3.5$ may not form an evolutionary cohort, and certainly there is much evidence for the quenching of SF at a particular mass scale \citep[e.g.][]{Keres2005,Dekel2006,Peng2010}, leading to morphological and dynamical transformation. 

The smTFR has been studied numerous times between $0.5 < z < 2$, giving results consistent with zero evolution of the slope and zero-point \citep[e.g.][]{Flores2006,Miller2011,Kassin2012,Miller2012,Vergani2012,Miller2014,Contini2015a,Molina2016,Pelliccia2016,DiTeodoro2016,Harrison2017}.
Others have reported that as redshift increases, the smTFR shifts towards lower masses at fixed rotational velocity, e.g. in \cite{Puech2008,Puech2010} where a -0.34 dex log$(M_{\star})$ offset is observed at $z\sim0.6$, in the \cite{Tiley2016a} KROSS sample at $z\sim0.9$ with a -0.41 dex shift (although in this work a stricter cut than in H17 of $V_{C}/\sigma_{int} > 3$ is applied), in \cite{Cresci2009} a -0.41 dex offset is observed at $z\sim2$ (although the galaxies used to compute this were the highest fidelity rotators from the SINS sample with mean $V_{C}/\sigma_{int} = 5)$ and in G11 an offset of -1.29 dex is claimed, although there is large degree of scatter between the individual galaxies, with 50\% of the sample consistent with the $z\sim0$ relation.

To demonstrate the relationship between $V_{C}$ and $M_{\star}$ in individual KDS galaxies from the modelling sample, we plot in figure \ref{fig:tf_relation} (a) the inverted smTFR.
In this plane, $V_{C}$, a tracer of the dynamical mass of the system, is plotted as a function of the luminous mass for the KDS modelling sample.
The rotationally dominated galaxies with $V_{C}/\sigma_{int} > 1$ are shown with red circles, whereas the dispersion dominated $V_{C}/\sigma_{int} < 1$  galaxies are shown with the open circles.
The black errorbars on the individual points reflect the upper and lower errors on the individual $V_{C}$ values, whereas the black errorbar in the bottom left gives the representative 0.2 dex error in measuring $M_{\star}$ for each galaxy from the SED fits.
In H17, the smTFR relation is fit to $\sim400$ SFGs at $z\sim0.9$, measured to be rotationally dominated with $V_{C}/\sigma_{int} > 1$ and with $9.5 < log(M_{\star}[M_{\odot}]) < 10.5$, similar to the KDS galaxies.
The data are used to constrain parameters `a' and `b' in the equation $logV_{C}=b+a[logM_{\star} - 10.0]$ (see their section 4.2) finding $a = 0.33 \pm 0.11$ and $b = 2.12 \pm 0.04$.
We plot this best fit relation with the green line in figure \ref{fig:tf_relation} (a), with the green shaded region giving the typical uncertainty on the fit of 0.2 dex along the velocity axis.
These parameters are consistent with the values $a_{z=0} = 0.278 \pm 0.010$ and $b_{z=0} = 2.142 \pm 0.004$, measured in \cite{Reyes2011} for a sample of $z<0.1$ disks traced with H$\alpha$, in a similar mass range and where $V_{C}$ has been extracted at the same radius, and we plot this as the dashed black line in figure \ref{fig:tf_relation} (a).
The agreement between these parameters derived at $z\sim0$ and $z\sim0.9$, where the galaxies span a similar $M_{\star}$ range and the $V_{C}$ values have been computed following a consistent methodology, suggests that there is no evolution in the smTFR between these limits.
The rotation dominated KDS galaxies with $V_{C}/\sigma_{int} > 1$ and $9.5 < log(M_{\star}[M_{\odot}]) < 10.5$ are broadly consistent within the errorbars with the H17 fit, showing a tendency to lie below the relation.
The dispersion dominated KDS galaxies sit below the H17 relation, bringing significant scatter to lower $V_{C}$ values at fixed $M_{\star}$ as is also observed in H17 and for early-type galaxies in the local universe \citep[e.g.][]{Romanowsky2012}.  
We do not attempt to fit through our data, due to the relatively narrow mass range spanned by the points, and conclude that there does not appear to be evolution in the zero-point of the smTFR between $z\sim1-3.5$ for rotation-dominated galaxies as traced by KROSS and KDS.
This result is a strong function of sample selection; in \cite{Tiley2016a} where a stricter $V_{C}/\sigma_{int} > 3$ cut is applied, evolution to lower masses at fixed rotational velocity is observed as a result of omitting rotationally dominated galaxies with lower $V_{C}$ values (as is also the case in \cite{Cresci2009} with the high fidelity disk sample).
Conversely, fitting the smTFR through a full sample including dispersion dominated galaxies would shift the zero-point in the opposite sense.
The conclusion from this work is that the smTFR appears to show no evolution across the bulk of cosmic time when the measurement of dynamical properties and selection of rotation dominated galaxies are consistent.
This comes with the caveat that it is not clear whether main sequence SFGs at difference redshift slices can be considered to form an evolutionary sequence.

For completeness we also include the $S_{0.5} = \sqrt{0.5V_{C}^{2} + \sigma_{int}^{2}}$ vs. $M_{\star}$ relation described throughout \cite[K07][]{Kassin2007} in figure \ref{fig:tf_relation} (b) (which is consistent with the absorption-line-based $M_{\star}$ Faber-Jackson relation for nearby elliptical galaxies), where we plot $S_{0.5}$ vs. $M_{\star}$. 
This is connected to the idea suggested in figure \ref{fig:dyn_masses}, that the rotational velocities alone may not be a good tracer of the total dynamical mass, with $S_{0.5}$ accounting for the contribution to the dynamical mass from non-circular motions.
In K07, using $S_{0.5}$ reduced the scatter in the TFR out to $z\sim1.2$, which they attribute to the inclusion of galaxies with peculiar kinematics in the sample with reduced rotational velocities in relation to their stellar mass.
$S_{0.5}$ has been studied in both C09 and G11, with both reporting an evolution of the K07 trend towards lower $M_{\star}$ at fixed $S_{0.5}$, which is taken as further evidence for the evolution of the smTFR.
In figure \ref{fig:tf_relation} (b) the KDS galaxies agree with the K07 relation within the errorbars, with reduced scatter in comparison to the smTFR in panel (a), suggesting that their is no evolution in $S_{0.5}-M_{\star}$ relation.
The symbol convention is the same as in panel (a), and the division now between the rotation and dispersion dominated galaxies is no longer clear. 
Considering the increase in observed $\sigma_{int}$ values over cosmic time (figure \ref{fig:sigma_and_v_sigma_w_redshift}), this suggests that there is an interplay between the $V_{C}$ values and $\sigma_{int}$ traced by ionised gas emission lines as discussed in \cite{Kassin2012}. 
We caution against over-interpretation of this result, since the origin of $\sigma_{int}$ and the extent to which this quantity traces dynamical mass is unclear, however this trend adds to the evidence that $V_{C}$ alone is an insufficient tracer of the total dynamical mass at this redshift. 

The smTFR derives directly from the virial theorem, in which changes to the dynamical mass are driven by changes in both the sizes and velocities of galaxies, prompting authors recently \citep[e.g.][]{Cortese2016,Contini2015a,Burkert2016a,Harrison2017,Swinbank2017} to study the connection between $M_{\star}$ and specific angular momentum, $j_{s} \propto R_{1/2}V_{C}$, with this possibly a more fundamental quantity to describe galaxy evolution.
We briefly discuss the specific angular momentum of the KDS galaxies in the following subsection.

\subsection{Angular Momentum}\label{subsec:ang_mom}

\begin{figure*}
\centering
\includegraphics[width=0.8\textwidth]{angular_momentum.png}
\caption{We plot the specific angular momentum $j_{s} \propto R_{1/2}V_{C}$ against stellar mass, $M_{\star}$, for the KDS galaxies with $V_{C}/\sigma_{int} > 1$ in red and with clear symbols when $V_{C}/\sigma_{int} < 1$.
Included also are the $j_{s} \propto M_{\star}^{\alpha}$ fits (with $\alpha = 0.6\pm0.2$) reported in H17 for the KROSS full sample in solid green and for the rotation and dispersion dominated subsamples with the dashed green lines.
The black solid line is the fit presented in \protect\cite{Romanowsky2012} to a sample of $z=0$ disk galaxies ($\alpha = 0.51\pm0.06$), to which the KROSS relation is offset in $j_{s}$ by $\sim0.2-0.3$dex.
The KDS rotation dominated galaxies are offset from the KROSS relation by a further $\sim0.3-0.4$dex, with the dispersion dominated galaxies scattering to lower $j_{s}$, with both KDS subsets following a roughly similar slope to KROSS and the $z=0$ sample.
Further, we include the $j_{s} \propto (1+z)^{-1}$ evolution reported in \protect\cite{Swinbank2017} with the red solid line, using the local slope and normalisation, which appears to pass through the KDS rotation dominated galaxies.
These results are in line with a scenario where the cosmic evolution of $j_{s}$ is driven mainly by the size evolution of galaxies as they accrete cold material to form stars from the IGM, adding to the outer components of the galaxy disks.
The split between rotation dominated and dispersion dominated galaxies may indicate that these systems are the progenitors of local late-type and early-type galaxies respectively.}
\label{fig:ang_mom}
\end{figure*}

In figure \ref{fig:ang_mom} we plot the specific angular momentum, j$_{\star} \equiv J/M_{\star}$ where J is the total angular momentum \citep{Fall1983}, as a function of stellar mass M$_{\star}$.
This quantity has been described in detail in \cite{Harrison2017} section 4.3 and refer to this for more details about the use of $j_{s}$ as measured from ionised gas kinematics as a tracer of the specific angular momentum content of SFGs.

\begin{equation}\label{eq:ang_mom}
   j_{n} = k_{n}C_{i}v_{s}R_{1/2}
\end{equation}

In equation \ref{eq:ang_mom} we list the approximate estimator for specific angular momentum described in \cite{Romanowsky2012} (their equation 6), with k$_{n}$ a numerical coefficient which depends on the \Sers index of the galaxy, n, and is approximated in equation \ref{eq:kn} (see \cite{Romanowsky2012}).

\begin{equation}\label{eq:kn}
   k_{n} = 1.15 + 0.029n + 0.062n^{2}
\end{equation}

Due to varying HST coverage and quality, it has not been possible to estimate n for all KDS galaxies.
However we have verified with both floating \Sers index fits, fixed n = 1 \Sers fits and fixed n = 4 \Sers fits throughout \cref{subsubsec:galfitting} that the vast majority of GOODS-S galaxies ($\sim70\%$ of the KDS modelling sample) are well described by the n = 1 exponential profile (also assumed throughout the modelling procedure in \cref{sec:Kinematic_Properties}).
Setting n = 1 in equation \ref{eq:kn} leaves $k_{n} = 1.19$ and we continue under this assumption analogous to in \cite{Harrison2017} (although we note that as stated in \cite{Harrison2017} setting n = 1 results in only a $\sim20\%$ difference in the derived $j_{s}$ values).
In equation \ref{eq:ang_mom} $C_{i}$ is the de-projection factor, assumed to be $1/sin(i)$, and $v_{s}$ is the observed velocity at $2R_{1/2}$.
The combination $v_{s}C_{i} \equiv V_{C}$ and we can use our measured rotation velocities to leave \ref{eq:ang_mom_final} as the final expression for $j_{s}$ used to calculate the values in figure \ref{fig:ang_mom}. 

\begin{equation}\label{eq:ang_mom_final}
   j_{n=1} \equiv j_{s} = 1.19V_{C}R_{1/2}
\end{equation}

In figure \ref{fig:ang_mom} we plot the fit to all KROSS galaxies with the solid green line, and also the individual fits to the rotation and dispersion dominated KROSS galaxies with the dashed green and light green lines respectively.
The fits to the full sample and rotation dominated KROSS galaxies (with fitting relation defined by log$j_{s} =  \beta + \alpha[logM_{\star} - 10.0]$) lie $\sim0.2-0.3$ dex beneath the $z\sim0$ relation from \cite{Romanowsky2012}, in contrast to the smTFR, with the dispersion dominated galaxies scattering around much lower $j_{s}$ values.
This offset is explained to be a consequence mainly of the size evolution of the disks of SFGs over the redshift range $0-0.9$.
The local and KROSS relations have comparable slopes, with $\alpha_{z=0}=0.51\pm0.06$ and $\alpha_{z=0.9}=0.6\pm0.2$, in agreement with other studies of the $j_{s}-M_{\star}$ relation \citep{Cortese2016,Contini2015a,Burkert2016a,Swinbank2017}, and also in agreement with predictions for the slope of the relationship between dark matter specific angular momentum and the halo mass (i.e. $j_{halo} \propto M_{halo}^{2/3}$ \citep[e.g.][]{Barnes1987}).
This is widely regarded as a consequence of the baryonic retention of angular momentum following a period in the formation of the proto-galaxy when the dark matter and baryons were well mixed.

The rotation dominated KDS galaxies form a sequence $\sim 0.3-0.4$ dex beneath the KROSS relation, and with similar slope, however the slope is not well constrained due to the scatter and sparsity of the dataset.
The dispersion dominated KDS galaxies form a sequence with similar slope but offset to lower still $j_{s}$ values.
We plot also the $\sim j_{s} \propto (1+z)^{-1}$ reported in \cite{Swinbank2017} (using the local slope and normalisation from \cite{Romanowsky2012}), which the rotation dominated KDS galaxies appear to roughly follow.
These results suggest that there is a continual drop in the specific angular momentum of SFGs with increasing redshift, as governed by the evolution of disk sizes \citep[e.g.][]{Trujillo2007} (see figure \ref{fig:morpho-distributions}) and that the relationship between the angular momentum of the baryons and the halos is constant over time.
Conversely, the angular momentum of SFGs will increase with decreasing redshift as accretion of cold gas from the IGM which fuels star formation and tidal disruption from satellite galaxies or major mergers conspire to grow disk sizes over time \citep{Trujillo2007,Buitrago2008,VanderWel2014a}.
Additonally since we observe a bimodal distribution of the rotation and dispersion dominated KDS galaxies in the $j_{s}-M_{\star}$ plane, we are plausibly observing the progenitors of late-type and early-type galaxies at $z\sim3.5$ as revealed by their kinematic signatures, consistent with the idea that early-type galaxies began their lives with less total angular momentum available to them.  

\section{CONCLUSIONS}\label{sec:conclusion}
We have presented new measurements of the dynamical properties of $\sim 45$ spatially resolved SFGs at $z\sim3.5$ as part of the KDS, based on IFU data observed with KMOS.
These measurements push back the frontier of IFU observations in the early universe and provide more robust constraints on the internal and rotational dynamics of $9.5 < log(M_{\star}[M_{\odot}])< 10.5$ typical main sequence galaxies at these redshifts.
By using a combined morpho-kinematic classification based on broad-band HST imaging and our IFU data, we have separated probable interacting galaxies from the sample and made beam-smearing corrected measurements of $V_{C}$ and $\sigma_{int}$ for those remaining.
The main conclusions of this work are summarised as follows:

\begin{itemize}
    \item We observe an increase in $\sigma_{int}$ values in comparison with $z\sim0$ galaxies, with a sample mean of $\sigma_{int} = 70.8^{+3.3}_{-3.1} kms^{-1}$.
    We plot mean $\sigma_{int}$ values computed in a consistent way for SFG samples spanning a wide redshift baseline, finding a sharp increase in $\sigma_{int}$ values between $z=0-1$ and a fairly shallow increase thereafter, mediated by the mean $M_{star}$ of the galaxy samples (figure \ref{fig:sigma_and_v_sigma_w_redshift}).
    This is in line with a simple equilibrium model prescription in which the gas fractions and impact of several physical mechanisms such as accretion of gas from the IGM, stellar feedback and turbulence increase with redshift and combine to boost random motions within high redshift galaxies.
    \item As a result of the increase in $\sigma_{int}$, most of the resolved galaxies are dispersion dominated, with velocity dispersions greater than their measured rotation velocities.
    We measure a rotation dominated percentage, $V_{C}/\sigma_{int} > 1$, of $39 \pm 8 \%$ in our sample, substantially lower than surveys at lower redshift, although there is significant diversity amongst the individual $V_{C}/\sigma_{int}$ measurements at each redshift slice (figure \ref{fig:rdf_and_v_sigma_w_redshift}).
    \item The mean rotational velocity value of $V_{C} = 78.8^{+5.4}_{-5.1}kms^{-1}$ appears low in comparison to surveys at lower redshift, although there is again significant diversity in $V_{C}$ measurements in each survey (figure \ref{fig:sigma_and_v_sigma_w_redshift}).
    This appears to be mainly the result of averaging over a sample with a much higher percentage of dispersion dominated galaxies.
    \item When the $V_{C}$ values are viewed as a function of mass in the inverse stellar mass Tully-Fisher Relation, the rotation dominated galaxies are correlated with mass and lie within the errors on the same relation as derived for rotation dominated galaxies with $9.0 < log(M_{\star}[M_{\odot}]) < 11.0$ at $z\sim0.9$ in the KROSS survey (figure \ref{fig:tf_relation}). %something else here when finished writing the TF section
    This relation is consistent with SFGs at $z=0$ and so we report no significant evolution in the slope or zero-point of the smTFR between $z=0-3.5$.
    Constistency of sample selection and measurement techniques are crucial factors when determining evolution in the $V_{C}$ vs. $M_{\star}$ plane by comparing samples at different redshifts.
    The dispersion dominated galaxies scatter below the trend.
    \item The virial mass computed from $V_{C}$ values alone is lower than the stellar mass for the majority of KDS galaxies.
    We compute an additional component of the virial mass contributed by the velocity dispersion, shifting the virial mass above the stellar mass, and use this as a qualitative indicator that $\sigma_{int}$ is responsible for supporting some fraction of the total mass of the system (figure \ref{fig:dyn_masses}).
    This is clearly over-simplified, as the lack of correlation between $\sigma_{int}$ and $M_{\star}$ suggests that not all of the random motions support mass, and may originate from turbulence.
    \item To investigate the possible contribution of $\sigma_{int}$ to the total dynamical mass further, we plot $S_{0.5} = \sqrt{0.5V^{2} + \sigma_{int}^{2}}$ as a function of $M_{\star}$, finding a tighter relationship than in the case of the inverse smTFR, which shows no evidence for evolution from the $z=0$ relation from \cite{Kassin2007} (figure \ref{fig:tf_relation}).
    We caution again though that there is no clear relationship between $\sigma_{int}$ and $M_{\star}$ and this result should not be over-interpreted.  
    \item The specific angular momentum (angular momentum divided by stellar mass $M_{\star}$; $j_{s}$) of the KDS galaxies is $\sim$0.3-0.4 dex lower than KROSS galaxies at $z\sim0.9$ at fixed $M_{\star}$, and the slope of the $j_{s}-M_{\star}$ relation appears comparable between the two samples (figure \ref{fig:ang_mom}).
    This supports a scenario in which the derease in $j_{s} \propto R_{D}V_{C}$ with increasing redshift between $z=0-3.5$ is driven by the gradual size evolution of galaxies over the same redshift interval (figure \ref{fig:morpho-distributions}).
    \item Almost all of the resolved KDS galaxies in the $z\sim3.1$ high density protocluster pointing SSA22-a are involved in interactions, which has important consequences for the dynamics of and accumulation of stellar populations within these cluster galaxies which sit at the peak of the cosmological density field.
    The merger fraction computed across the other, lower density pointings is 15\%.

\end{itemize}

\section*{Acknowledgements}

This work is based on observations taken by the CANDELS Multi-Cycle Treasury Program with the NASA/ESA HST, which is operated by the Association of Universities for Research in Astronomy, Inc., under NASA contract NAS5-26555.
This work is based on observations taken by the 3D-HST Treasury Program (GO 12177 and 12328) with the NASA/ESA HST, which is operated by the Association of Universities for Research in Astronomy, Inc., under NASA contract NAS5-26555.
Based on data obtained with the European Southern Observatory Very Large Telescope, Paranal, Chile, under Large Program 185.A-0791, and made available by the VUDS team at the CESAM data center, Laboratoire d'Astrophysique de Marseille, France.
The HST data matched to the VUDS-DR1 are described in \cite{Grogin2011} and \cite{Koekemoer2011} for CANDELS and include data from the ERS \cite{Windhorst2011}.

%%%%%%%%%%%%%%%%%%%%%%%%%%%%%%%%%%%%%%%%%%%%%%%%%%

%%%%%%%%%%%%%%%%%%%% REFERENCES %%%%%%%%%%%%%%%%%%

% The best way to enter references is to use BibTeX:

%\bibliographystyle{mnras}
%\bibliography{example} % if your bibtex file is called example.bib


% Alternatively you could enter them by hand, like this:
% This method is tedious and prone to error if you have lots of references
\clearpage 
\bibliographystyle{apj.bst}
%\bibliography{/usr/local/texlive/texmf-local/bibtex/bib/ojt.bib}
\bibliography{/Users/owenturner/Documents/PhD/KMOS/Latex/Bibtex/library.bib}

\clearpage
%%%%%%%%%%%%%%%%%%%%%%%%%%%%%%%%%%%%%%%%%%%%%%%%%%

%%%%%%%%%%%%%%%%% APPENDICES %%%%%%%%%%%%%%%%%%%%%

\appendix

\section{Kinematics Plots}\label{app:kinematics_plots}
\subsection{Rotation Dominated ($V_{C}/\sigma_{int} > 1$)}\label{app:rotation_dominated}

\begin{figure*}
\centering
\includegraphics[width=\textwidth]{GOODS_ROTATION/combine_sci_reconstructed_bs006516_grid_paper.png}
\includegraphics[width=\textwidth]{GOODS_ROTATION/combine_sci_reconstructed_bs008543_grid_paper.png}
\includegraphics[width=\textwidth]{GOODS_ROTATION/combine_sci_reconstructed_bs016759_grid_paper.png}
\includegraphics[width=\textwidth]{GOODS_ROTATION/combine_sci_reconstructed_cdfs_lbg_20_grid_paper.png}
\includegraphics[width=\textwidth]{GOODS_ROTATION/combine_sci_reconstructed_cdfs_lbg_32_grid_paper.png}
\label{fig:goods_rot_1}
\end{figure*}

\begin{figure*}
\centering
\includegraphics[width=\textwidth]{GOODS_ROTATION/combine_sci_reconstructed_cdfs_lbg_38_grid_paper.png}
\label{fig:goods_rot_2}
\end{figure*}


\begin{figure*}
\centering
\includegraphics[width=\textwidth]{GOODS_ROTATION/combine_sci_reconstructed_cdfs_lbg_91_grid_paper.png}
\includegraphics[width=\textwidth]{GOODS_ROTATION/combine_sci_reconstructed_bs009818_grid_paper.png}
\includegraphics[width=\textwidth]{GOODS_ROTATION/combine_sci_reconstructed_b012141_012208_grid_paper.png}
\label{fig:goods_rot_3}
\end{figure*}

\begin{figure*}
\centering
\includegraphics[width=\textwidth]{SSA_ROTATION/combine_sci_reconstructed_s_sa22a-d3_grid_paper.png}
\includegraphics[width=\textwidth]{SSA_ROTATION/combine_sci_reconstructed_n3_009_grid_paper.png}
\includegraphics[width=\textwidth]{SSA_ROTATION/combine_sci_reconstructed_s_sa22b-c20_grid_paper.png}
\includegraphics[width=\textwidth]{SSA_ROTATION/combine_sci_reconstructed_s_sa22b-d5_grid_paper.png}
\label{fig:SSA_rot_1}
\end{figure*}

\subsection{Dispersion Dominated ($V_{C}/\sigma_{int} < 1$)}\label{app:dispersion_dominated}

\begin{figure*}
\centering
\includegraphics[width=\textwidth]{GOODS_DISPERSION/combine_sci_reconstructed_bs006541_grid_paper.png}
\includegraphics[width=\textwidth]{GOODS_DISPERSION/combine_sci_reconstructed_b15573_grid_paper.png}
\includegraphics[width=\textwidth]{GOODS_DISPERSION/combine_sci_reconstructed_cdfs_lbg_24_grid_paper.png}
\label{fig:goods_disp_1}
\end{figure*}

\begin{figure*}
\centering
\includegraphics[width=\textwidth]{GOODS_DISPERSION/combine_sci_reconstructed_bs014828_grid_paper.png}
\includegraphics[width=\textwidth]{GOODS_DISPERSION/combine_sci_reconstructed_cdfs_lbg_25_grid_paper.png}
\includegraphics[width=\textwidth]{GOODS_DISPERSION/combine_sci_reconstructed_cdfs_lbg_94_grid_paper.png}
\label{fig:goods_disp_2}
\end{figure*}


\begin{figure*}
\centering
\includegraphics[width=\textwidth]{GOODS_DISPERSION/combine_sci_reconstructed_cdfs_lbg_105_grid_paper.png}
\includegraphics[width=\textwidth]{GOODS_DISPERSION/combine_sci_reconstructed_cdfs_lbg_109_grid_paper.png}
\includegraphics[width=\textwidth]{GOODS_DISPERSION/combine_sci_reconstructed_cdfs_lbg_111_grid_paper.png}
\label{fig:goods_disp_3}
\end{figure*}

\begin{figure*}
\centering
\includegraphics[width=\textwidth]{GOODS_DISPERSION/combine_sci_reconstructed_cdfs_lbg_113_grid_paper.png}
\includegraphics[width=\textwidth]{SSA_DISPERSION/combine_sci_reconstructed_lab7_top_grid_paper.png}
\label{fig:goods_disp_4}
\end{figure*}


\begin{figure*}
\centering
\includegraphics[width=\textwidth]{SSA_DISPERSION/combine_sci_reconstructed_s_sa22b-md25_grid_paper.png}
\includegraphics[width=\textwidth]{SSA_DISPERSION/combine_sci_reconstructed_n3_006_grid_paper.png}
\includegraphics[width=\textwidth]{SSA_DISPERSION/combine_sci_reconstructed_lab25_grid_paper.png}
\label{fig:SSA_disp_2}
\end{figure*}


\begin{figure*}
\centering
\includegraphics[width=\textwidth]{SSA_DISPERSION/combine_sci_reconstructed_lab18_grid_paper.png}
\includegraphics[width=\textwidth]{SSA_DISPERSION/combine_sci_reconstructed_s_sa22b-d9_grid_paper.png}
\includegraphics[width=\textwidth]{SSA_DISPERSION/combine_sci_reconstructed_n_c3_grid_paper.png}
\label{fig:SSA_disp_1}
\end{figure*}

\subsection{Mergers}\label{app:mergers}

\begin{figure*}
\centering
\includegraphics[width=\textwidth]{SSA_MERGERS/combine_sci_reconstructed_s_sa22b-c10_grid_paper.png}
\includegraphics[width=\textwidth]{SSA_MERGERS/combine_sci_reconstructed_s_sa22a-md23_grid_paper.png}
\includegraphics[width=\textwidth]{SSA_MERGERS/combine_sci_reconstructed_s_sa22a-md46_grid_paper.png}
\caption{ssa mergers}
\label{app:ssa_mergers_1}
\end{figure*}

\begin{figure*}
\centering
\includegraphics[width=\textwidth]{SSA_MERGERS/combine_sci_reconstructed_s_sa22a-m38_grid_paper.png}
\includegraphics[width=\textwidth]{SSA_MERGERS/combine_sci_reconstructed_s_sa22a-c35_grid_paper.png}
\includegraphics[width=\textwidth]{SSA_MERGERS/combine_sci_reconstructed_s_sa22a-c30_grid_paper.png}
\caption{ssa mergers}
\label{app:ssa_mergers_2}
\end{figure*}

\begin{figure*}
\centering
\includegraphics[width=\textwidth]{SSA_MERGERS/combine_sci_reconstructed_s_sa22a-aug96m16_grid_paper.png}
\includegraphics[width=\textwidth]{SSA_MERGERS/combine_sci_reconstructed_n_c49_grid_paper.png}
\includegraphics[width=\textwidth]{SSA_MERGERS/combine_sci_reconstructed_n_c47_grid_paper.png}
\caption{ssa mergers}
\label{app:ssa_mergers_3}
\end{figure*}

\begin{figure*}
\centering
\includegraphics[width=\textwidth]{GOODS_MERGERS/combine_sci_reconstructed_cdfs_lbg_14_grid_paper.png}
\includegraphics[width=\textwidth]{GOODS_MERGERS/combine_sci_reconstructed_cdfs_lbg_23_grid_paper.png}
\includegraphics[width=\textwidth]{GOODS_MERGERS/combine_sci_reconstructed_bs010545_grid_paper.png}
\includegraphics[width=\textwidth]{GOODS_MERGERS/combine_sci_reconstructed_cdfs_lbg_93_grid_paper.png}
\caption{High resolution imaging is important not only for reliably recovering the morphological parameters, but also to explain peculiar features in the observed velocity and velocity dispersion fields.
In the two examples above, multiple components in the HST imaging clearly correspond to marginally different [O~{\sc III}]$\lambda$5007 emission wavelengths, which manifest as velocity shifts in individual spaxels and broadens the observed dispersion pattern where the emission lines at different wavelengths merge.
Overplotted onto the models are the data values and uncertainties extracted along $PA_{kin}$.
These figures also show the HST imaging for each galaxy, the best fit galfit model and residuals, 2D maps of the [O~{\sc III}]$\lambda$5007 flux, observed velocity, observed dispersion, best fitting smeared model velocity, data - model residuals, intrinsic velocity dispersion (observed dispersion corrected for beam smearing and instrumental resolution as described in \cref{subsubsec:param_extraction}) as well as a plot of 1D extractions along the observed, intrinsic and beam smeared dispersion maps.}
\label{app:goods_mergers_1}
\end{figure*}


\section{Table of measurements from different surveys}
\begin{table*}
\centering
\caption{We present here a reference of the mean and median kinematic properties used throughout figures 7 and 8 for the different surveys.
The errors on the mean represent the statistical errors from bootstrap resampling and the lower and upper errors on the medians are the 16th and 84th percentiles of the distribution respectively.}
\label{tab:evolution_numbers}
\begin{tabular}{c c c c c c c c c c }

 \hline
Survey & $\left< z \right> $ & $\left< log\left(\frac{M_{\star}}{M_{\odot}}\right)\right>$ & $\left< \frac{V_{C}}{\sigma_{int}} \right>$ & med$\left(\frac{V_{C}}{\sigma_{int}}\right)$ & RDF & $ \left< \sigma_{int} \right>$ & med$\left(\sigma_{int}\right)$ & $\left<V_{C}\right>$ & med$\left(V_{C}\right)$  \\
 \hline
 GHASP & 0.001 & 10.6 & $12.91^{+0.51}_{-0.44}$ & $12.45^{+5.15}_{-5.05}$ & $1.0^{+0.00}_{-0.024}$ & $13.0^{+0.5}_{-0.5}$ & $13.0^{+6.0}_{-4.0}$ & $189.0^{+3.5}_{-3.0}$ & $159.4^{+113.3}_{-79.2}$ \\
 DYNAMO & 0.1 & 10.3 & $5.06^{+0.18}_{-0.15}$ & $4.55^{+3.06}_{-2.22}$ & $0.96^{+0.02}_{-0.015}$ & $45.9^{+0.3}_{-0.3}$ & $39.0^{+18.2}_{-21.9}$ & $183.0^{+1.0}_{-1.0}$ & $164.0^{+76.0}_{-50.0}$ \\
 KROSS & 0.9 & 9.9 & $3.05^{+0.17}_{-0.17}$ & $2.36^{+2.64}_{-1.46}$ & $0.81^{+0.05}_{-0.05}$ & - & - & $117.0^{+4.0}_{-4.0}$ & $109^{+77.0}_{-66.0}$ \\
 KMOS$^{3D}$ & 1.0 & 10.7 & 5.5 & - & 0.93 & 25 & - & 170 & - \\
 MASSIV & 1.2 & 10.2 & $2.41^{+1.4}_{-0.9}$ & $1.95^{+2.40}_{-1.31}$ & $0.67^{+0.06}_{-0.06}$ & $61.8^{+3.8}_{-4.2}$ & $52.0^{+23.2}_{-20.7}$ & $132.13^{+10.4}_{-8.2}$ & $103.0^{+99.0}_{-61.2}$ \\
 SINS(C09) & 2.0 & 10.6 & $5.0^{+0.93}_{-0.97}$ & $4.74^{+1.11}_{-0.18}$ & 1.0 & $51.2^{+8.0}_{-7.9}$ & $42.5^{+14.5}_{-3.5}$ & $232.0^{+12.8}_{-12.7}$ & $240.0^{+31.0}_{-60.2}$ \\
 SINS(F09) & 2.0 & 10.6 & 2.6 & - & 0.60 & - & - & $201.3^{+4.3}_{-4.0}$ & $174.0^{+83.0}_{-58.0}$ \\
 KMOS$^{3D}$ & 2.2 & 10.9 & 2.6 & - & 0.73 & 55 & - & 170 & - \\
 AMAZE & 3.0 & 10.0 & $3.59^{+1.45}_{-1.1}$ & $2.1^{+4.7}_{-1.15}$ & 0.33 & $85.9^{+1.5}_{-1.4}$ & $78.0^{+27.0}_{-48.0}$ & $217^{+59.1}_{-40.2}$ & $129^{+166.0}_{-50.2}$ \\
 KDS & 3.5 & 10.0 & $1.125^{+0.18}_{-0.15}$ & $0.99^{+0.59}_{-0.34}$ & $0.39^{+0.08}_{-0.08}$ & $70.8^{+3.3}_{-3.1}$ & $67.0^{+18.4}_{-19.0}$ & $78.8^{+5.4}_{-5.1}$ & $60.0^{+49.1}_{-18.2}$ \\
 \hline
\end{tabular}
\end{table*}


\section{Kinematic Parameters Error Estimates}\label{app:kin_error_estimates}

\subsection{3D modelling kinematic parameter error estimates}\label{appsubsec:model_errors}
As mentioned above, the MCMC sampling provides distributions for each of the model parameters, from which we can extract the 84th and 16th percentile values as the +/- 1-$\sigma$ errors.
During the following procedure $PA_{kin}$ is fixed to the maximum likelihood value for all model evaluations.
We also fold in the uncertainty on measuring the inclination angle at this point, assuming a conservative GALFIT measurement error of $\Delta$i = 10$\%$ following the analysis of EXAMPLE. 
The beam smeared and intrinsic models are reconstructed using the 16th and 84th percentile parameters, with the lower velocity 16th percentile model using the inclination value closer to edge-on, and the 84th percentile model using the inclination closer to face-on.
To clearly distinguish the 1-$\sigma$ error regions for both the beam smeared and intrinsic model the region between the 16th and 84th percentile evaluations are shaded blue and red respectively in the figures shown in Appendix A.
The +/- 1-$\sigma$ error values for both $v_{2.2}$ and $v_{3.0}$ are then calculated using the the equations below, also taking account the measurement errors with $\bar{\sigma}_{v_{obs}}$ equal to the average observational uncertainty extracted along $PA_{kin}$.
The subscripts `16th' and `84th' denote the model from which the velocity has been extracted.

\begin{equation}\label{eq:v2.2_plus}
   \delta v_{2.2}+ = \sqrt{(v_{2.2_{84}} - v_{2.2})^{2} + (\bar{\delta}v_{obs})^{2}}
\end{equation}

\begin{equation}\label{eq:v2.2_minus}
   \delta v_{2.2}- = \sqrt{(v_{2.2} - v_{2.2_{16}})^{2}  + (\bar{\delta}v_{obs})^{2}}
\end{equation}

\begin{equation}\label{eq:v3.0_plus}
   \delta v_{3.0}+ = \sqrt{(v_{3.0_{84}} - v_{3.0})^{2} + (\bar{\delta}v_{obs})^{2}}
\end{equation}

\begin{equation}\label{eq:v3.0_minus}
   \delta v_{3.0}- = \sqrt{(v_{3.0} - v_{3.0_{16}})^{2}  + (\bar{\delta}v_{obs})^{2}}
\end{equation}

The upper and lower errors on the mean velocity dispersion are calculated in a similar fashion.
To encorporate the uncertainty introduced in the modelling by assuming a fixed value of $\sigma_{int}=50kms^{-1}$, we make two further model evaluations using both the 16th and 84th percentile parameters with $\sigma_{int} = 40kms^{-1}$ and $\sigma_{int} = 80kms^{-1}$, with these representing reasonable minimum and maximum mean instrinsic sigma values.
When assuming the broader $\sigma_{int} = 80km^{-1}$ value in each spaxel the beam smearing correction decreases, as the impact of convolution with other lines that have a shifted velocity centre is less severe.
Conversely, assuming $\sigma_{int} = 40km^{-1}$ increases the beam smearing.
We calculate the minimum and maxmimum intrinsic velocity dispersion, $\sigma_{o-84-40}$ and $\sigma_{o-16-80}$, using equations \ref{eq:sig_84_40} and \ref{eq:sig_16_80} respectively:

\begin{equation}\label{eq:sig_84_40}
   \sigma_{o-84-40} = \sqrt{(\sigma_{obs} - \sigma_{bs-84-40})^{2} - \sigma_{sky}^{2}}
\end{equation}

\begin{equation}\label{eq:sig_16_80}
   \sigma_{o-16-80} = \sqrt{(\sigma_{obs} - \sigma_{bs-16-80})^{2} - \sigma_{sky}^{2}}
\end{equation}

and then the lower and upper errors on $\sigma_{o}$ are given by the following:

\begin{equation}\label{eq:sig_plus_error}
   \delta\sigma_{o}+ = \sqrt{(\sigma_{o} - \sigma_{o-84-40})^{2} + (\bar{\sigma}_{\sigma_{obs}})^{2}}
\end{equation}

\begin{equation}\label{eq:sig_minus_error}
   \delta\sigma_{o}- = \sqrt{(\sigma_{o-16-80} - \sigma_{o})^{2} + (\bar{\sigma}_{\sigma_{obs}})^{2}}
\end{equation}

with $\bar{\sigma}_{\sigma_{obs}}$ equal to the average of the velocity dispersion measurement errors.
Once these quantities have been measured, the upper and lower errors on the ratio $\frac{v_{2.2}}{\sigma{o}}$ can be computed using equations \ref{eq:v_over_sig_plus} and \ref{eq:v_over_sig_minus}, and the same prescription is followed to find the errors on $\frac{v_{3.0}}{\sigma{o}}$.

\begin{equation}\label{eq:v_over_sig_plus}
   \delta\frac{v_{2.2}}{\sigma_{o}}+ = \frac{v_{2.2}}{\sigma_{o}}\sqrt{\left(\frac{\delta v_{2.2}+}{v_{2.2}}\right)^{2} + \left(\frac{\delta\sigma_{o}+}{\sigma_{o}}\right)^{2}}
\end{equation}

\begin{equation}\label{eq:v_over_sig_minus}
   \delta\frac{v_{2.2}}{\sigma_{o}}- = \frac{v_{2.2}}{\sigma_{o}}\sqrt{\left(\frac{\delta v_{2.2}-}{v_{2.2}}\right)^{2} + \left(\frac{\delta\sigma_{o}+-}{\sigma_{o}}\right)^{2}}
\end{equation}

%%%%%%%%%%%%%%%%%%%%%%%%%%%%%%%%%%%%%%%%%%%%%%%%%%


% Don't change these lines
\bsp    % typesetting comment
\label{lastpage}
\end{document}

% End of mnras_template.tex